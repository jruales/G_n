
\section{Proof that $G_0$ is not left-orderable}
\label{section:G_0}
We start by proving that $G_0$ is not left-orderable, as the proof uses a different approach than the general case $n\geq1$
\begin{lemma} $G_0$ is isomorphic to $\langle x,\; y\mid{}(x^{-2}y^{2})^{3}=x^{5}=y^{5} \rangle$.
\end{lemma}
\begin{proof} When $n=0$, we have:
\begin{align*}
G_0=\langle a,\; b,\; c,\; d \mid{} d^{-1}a^{2},\; b^{-2}c,\; (d^{-1}c)^{3}c^{-1}bc^{-2},\; d^{2}a^{-1}d(c^{-1}d)^{3} \rangle.
\end{align*}
The first two relations show that $a^{2}=d$ and $b^{2}=c$, thus we can rewrite the presentation using only $a$ and $b$ as generators:
\begin{align*}
G_0&=\langle a,\;b\mid{} (a^{-2}b^{2})^{3}b^{-2}bb^{-4},\; a^{4}a^{-1}a^{2}(b^{-2}a^{2})^{3} \rangle\\
G_0&=\langle a,\;b\mid{} (b^{-2}a^{2})^{3}=b^{-5},\; a^{5}(b^{-2}a^{2})^{3} \rangle\\
G_0&=\langle a,\;b\mid{} (b^{-2}a^{2})^{3}=b^{-5},\; a^{5}b^{-5} \rangle\\
G_0&=\langle a,\;b\mid{} (a^{-2}b^{2})^{3}=a^{5}=b^{5} \rangle.\qedhere
\end{align*}
\end{proof}

\begin{lemma} Both $x^{5}$ and $y^{5}$ commute with all elements in $G_0$.
\label{lemma:G_0:center}
\end{lemma}
\begin{proof} Since we can change $x^5$ to $y^5$ and back as necessary, it is clear that both $x^5$ and $y^5$ commute with $x$, $y$, $x^{-1}$, and $y^{-1}$ and therefore with any element of $G_0$.

%Let $w\in{}G_0$ be any word. We will show that $wx^{5}=x^{5}w$ and similarly with $y^{5}$. The elements $x$ and $y$ generate $G_0$, so $w$ can be expressed as some word in $x$ and $y$, that is:
%\begin{align*}
%w=x^{n_0}y^{n_1}x^{n_2}y^{n_3}\cdots{}x^{n_{N-2}}y^{n_{N-1}}x^{n_N}.
%\end{align*}
%Where all $n_i$ are non-zero integers except $n_0$ and $n_N$ which may be zero. Then:
%\begin{align*}
%wx^{5}&=x^{n_0}y^{n_1}x^{n_2}y^{n_3}\cdots{}x^{n_{N-2}}y^{n_{N-1}}x^{n_N}x^{5}\hfil\\
%&=x^{n_0}y^{n_1}x^{n_2}y^{n_3}\cdots{}x^{n_{N-2}}y^{n_{N-1}}x^{n_N+5}\hfil\\
%&=x^{n_0}y^{n_1}x^{n_2}y^{n_3}\cdots{}x^{n_{N-2}}y^{n_{N-1}}x^{5}x^{n_N}\hfil\\
%&=x^{n_0}y^{n_1}x^{n_2}y^{n_3}\cdots{}x^{n_{N-2}}y^{n_{N-1}}y^{5}x^{n_N}\hfil\\
%&=x^{n_0}y^{n_1}x^{n_2}y^{n_3}\cdots{}x^{n_{N-2}}y^{n_{N-1}+5}x^{n_N}\hfil\\
%&=x^{n_0}y^{n_1}x^{n_2}y^{n_3}\cdots{}x^{n_{N-2}}y^{5}y^{n_{N-1}}x^{n_N}\hfil\\
%&=x^{n_0}y^{n_1}x^{n_2}y^{n_3}\cdots{}x^{n_{N-2}}x^{5}y^{n_{N-1}}x^{n_N}\hfil\\
%\intertext{\center $\vdots{}$}
%\setbox0\hbox{$\cdots{}$}\mathrel{\makebox[\wd0]{\hfil\vdots\hfil}}\\
%&\hspace{80pt} \vdots{}\\
%&=x^{n_0}x^{5}y^{n_1}x^{n_2}y^{n_3}\cdots{}x^{n_{N-2}}y^{n_{N-1}}x^{n_N}\hfil\\
%&=x^{n_0+5}y^{n_1}x^{n_2}y^{n_3}\cdots{}x^{n_{N-2}}y^{n_{N-1}}x^{n_N}\hfil\\
%&=x^{5}x^{n_0}y^{n_1}x^{n_2}y^{n_3}\cdots{}x^{n_{N-2}}y^{n_{N-1}}x^{n_N}\hfil\\
%wx^{5}&=x^{5}w.
%\end{align*}
%A similar proof is clear for $y^{5}$.
\end{proof}

\begin{lemma} If $G_0$ is left-orderable, then $wx^{n}w^{-1}$ has the same sign as $x$ for any $w\in{}G_0$ and for any $n\geq{}1$. Similarly, $wy^{n}w^{-1}$ has the same sign as $y$ for any $w\in{}G_0$ and for any $n\geq{}1$.
\label{lemma:G_0:conjugates}
\end{lemma}
\begin{proof} Suppose $G_0$ is left-orderable. We know by Lemma~\ref{lemma:G_0:center} that for any $w\in{}G_0$:
\begin{align*}
wx^{5}w^{-1}=ww^{-1}x^{5}=x^{5}.
\end{align*}
By Proposition~\ref{proposition:pospowers}, $x^5$ has the same sign as $x$, and thus $wx^{5}w^{-1}$ has the same sign as $x$. But $wx^{5}w^{-1}=(wxw^{-1})^{5}$, and so by Proposition~\ref{proposition:pospowers} $wxw^{-1}$ has the same sign as $wx^{5}w^{-1}$ and therefore has the same sign as $x$. A similar proof works for $y$.
\end{proof}

\begin{lemma} If $G_0$ is left-orderable and $x>1$, then $x^{-2}y^{2}>1$.
\label{lemma:G_0:XXyy}
\end{lemma}
\begin{proof}
Suppose $G_0$ is left-orderable and suppose $x>1$. Then $x^5>1$ and $(x^{-2}y^2)^{3}>1$ since $(x^{-2}y^2)^{3}=x^5$. By Proposition~\ref{proposition:pospowers} this shows that $x^{-2}y^2>1$.
\end{proof}

\begin{proposition}
The group $G_0=\langle x,\; y \mid{} (x^{-2}y^{2})^{3}=x^{5}=y^{5} \rangle$ is not left-orderable.
\label{propG0}
\end{proposition}
\begin{proof} Suppose (for contradiction) that $G_0$ is left-orderable. First note that if $x=1$, then $y^5=1$. Then either $y=1$ as well and $G_0$ is trivial, or $y\neq{}1$ and $G_0$ has torsion and is therefore not left-orderable by Fact~\ref{fact:torsion}, a contradiction. Thus by Fact~\ref{fact:WLOG} we can assume without loss of generality that $x>1$. Note that $x>1$ implies $x^5=y^5>1$ which implies $y>1$ by Proposition~\ref{proposition:pospowers}. Starting with the group relation, we have:
\begin{align}
x^{-2}y^{2}x^{-2}y^{2}x^{-2}y^{2}&=y^{5}\nonumber{}\\
x^{-2}y^{2}x^{-2}y^{2}x^{-2}&=y^{3}\nonumber{}\\
x^{-2}y^{2}x^{-2}y^{2}x^{2}&=y^{3}x^{4}\nonumber{}\\
x^{-2}y^{2}x^{-2}y^{2}x^{2}&=x^{5}y^{3}x^{-1}
\label{x5center}\\
y^{2}x^{-2}y^{2}x^{2} & =x^{7}y^{3}x^{-1}\nonumber{}\\
y^{-2}x^{-2}y^{2}x^{2} & =y^{-4}x^{7}y^{3}x^{-1}\nonumber{}\\
[x^{-2}y^{-2}]x^{-2}[y^{2}x^{2}] & =x^{-2}y^{-4}x^{5}x^{2}y^{3}x^{-1}\nonumber{}\\
& =x^{-2}yx^{2}y^{3}x^{-1}
\label{x5=y5}\\
[(x^{-1})x^{-2}y^{-2}]x^{-2}[y^{2}x^{2}(x)] & =(x^{-1})x^{-2}yx^{2}y^{3}x^{-1}(x)\nonumber{}\\
[(y^{3})x^{-3}y^{-2}]x^{-2}[y^{2}x^{3}(y^{-3})] & =(y^{3})x^{-3}yx^{2}(y^{3}y^{-3})\nonumber{}\\
[(x^{-2})y^{3}x^{-3}y^{-2}]x^{-2}[y^{2}x^{3}y^{-3}(x^{2})] & =(x^{-2})y^{3}x^{-3}yx^{2}(x^{2})\nonumber{}\\
[x^{-2}y^{3}x^{-3}y^{-2}] x^{-2} [x^{-2}y^{3}x^{-3}y^{-2}]^{-1} & =[x^{-2}y^{2}]y[(x^{-3})y(x^{3})]x.
\label{PFcontraG0}
\end{align}
Where for (\ref{x5center}) we have used the fact (shown in Lemma~\ref{lemma:G_0:center}) that $x^5$ commutes with any element of $G_0$, for (\ref{x5=y5}) we have used the group relation $x^{5}=y^{5}$. Now in (\ref{PFcontraG0}), the right expression must be positive since $x>1$, $y>1$, $x^{-2}y^{2}>1$ by Lemma~\ref{lemma:G_0:XXyy}, and $(x^{-3})y(x^{3})>1$ by Lemma~\ref{lemma:G_0:conjugates}. However, the expression on the left is negative by Lemma~\ref{lemma:G_0:conjugates} since it is of the form $(wx^{-1}w^{-1})^{2}$ for some $w\in{}G_0$. This is a contradiction.
\end{proof}

\begin{remark}
An alternative proof of Proposition \ref{propG0} follows from the fact that $K_0$ is a Montesino knot, and hence $\Sigma{}(K_0)$ is a Seifert fibered space. Proposition \ref{propG0} then follows from \cite[Theorem 4]{BoyerGordonWatson}.
\end{remark}

\section{Non-triviality and distinctness of the four generators}
\label{section:identityProofs}

\noindent{}Recall that
\begin{align*}
G_n=\langle a,\; b,\; c,\; d \mid{} &(a^{-1}b)^{10n}d^{-1}a^{2},\; b^{-2}c(b^{-1}a)^{10n},\\
&(d^{-1}c)^{10n+3}c^{-1}bc^{-2},\; d^{2}a^{-1}d(c^{-1}d)^{10n+3} \rangle
\end{align*}
for integers $n\geq{}0$.

\vspace{10pt}

\begin{proposition}
The four generators of $G_n$ are all distinct and non-trivial.
\label{proposition:non-trivial}
\end{proposition}
\begin{proof}
Greene and Watson \cite[Section 4.2]{GreeneWatson} show that the abelianization of $G_n$ provides a homomorphism:
\begin{align*}
\phi:G_n\rightarrow \mathbb{Z}_{25}
\end{align*}
with $\phi(a)=13$, $\phi(b)=3$, $\phi(c)=6$, and $\phi(d)=1$. Because each of the generators maps to a distinct, non-identity element of $\mathbb{Z}_{25}$, it follows that the four generators are distinct and nontrivial.
\end{proof}

\begin{corollary}
Both $bc^{-3}$ and $ad^{-3}$ are non-trivial.
\label{corollary:non-trivial2}
\end{corollary}
\begin{proof}
By the third group relation, we have:
\begin{align*}
(d^{-1}c)^{10n+3}c^{-1}bc^{-2}&=1\\
\Rightarrow{}(cd^{-1})^{10n+3}bc^{-3}&=1\\
\Rightarrow{}bc^{-3}&=(dc^{-1})^{10n+3}.
\end{align*}
By Proposition~\ref{proposition:non-trivial}, $dc^{-1}\ne1$, therefore $bc^{-3}\ne1$.\\

\noindent{}By the fourth group relation, we have:
\begin{align*}
(d^{-1}c)^{10n+3}d^{-1}ad^{-2}&=1\\
\Rightarrow{}(cd^{-1})^{10n+3}ad^{-3}&=1\\
\Rightarrow{}ad^{-3}&=(dc^{-1})^{10n+3}.
\end{align*}
By Proposition~\ref{proposition:non-trivial}, $dc^{-1}\ne1$, therefore $ad^{-3}\ne1$.
\end{proof}

\noindent{}We will assume $G_n$ is left-orderable and then reach a contradiction. By Fact~\ref{fact:WLOG}, we can assume without loss of generality that $b^{-1}a\geq1$. Because of Proposition~\ref{proposition:non-trivial}, we can assume (without loss of generality) that $b^{-1}a>1$.\\

\noindent{}With four (non-trivial) group generators there are 16 possible cases for the signs of each of the generators that must be considered (see Table~\ref{table:casesAll}). We will disprove each of these cases.



\begin{table}[ht]
\begin{center}
\begin{tabular}{l | l | l | l | l}
Case\hspace{10 pt} & $a$\hspace{10 pt} & $b$\hspace{10 pt} & $c$\hspace{10 pt} & $d$\hspace{10 pt} \\\hline\hline
1 & $+$ & $+$ & $+$ & $+$ \\\hline
2 & $+$ & $+$ & $+$ & $-$ \\\hline
3 & $+$ & $+$ & $-$ & $+$ \\\hline
4 & $+$ & $+$ & $-$ & $-$ \\\hline
5 & $+$ & $-$ & $+$ & $+$ \\\hline
6 & $+$ & $-$ & $+$ & $-$ \\\hline
7 & $+$ & $-$ & $-$ & $+$ \\\hline
8 & $+$ & $-$ & $-$ & $-$ \\\hline
9 & $-$ & $+$ & $+$ & $+$ \\\hline
10 & $-$ & $+$ & $+$ & $-$ \\\hline
11 & $-$ & $+$ & $-$ & $+$ \\\hline
12 & $-$ & $+$ & $-$ & $-$ \\\hline
13 & $-$ & $-$ & $+$ & $+$ \\\hline
14 & $-$ & $-$ & $+$ & $-$ \\\hline
15 & $-$ & $-$ & $-$ & $+$ \\\hline
16 & $-$ & $-$ & $-$ & $-$ 
\end{tabular}
\end{center}
\caption{The 16 possible cases with 4 generators}
\label{table:casesAll}
\end{table}

\section{General Lemmas}
\label{section:generalLemmas}

\noindent{}First we prove some lemmas that will be used later. These lemmas are true for all cases listed in Table~\ref{table:casesAll}.

\begin{lemma} In $G_n$, $d^{-1}a^{2}=c^{-1}b^{2}$.
\label{lemma:eq5}
\end{lemma}
\begin{proof} The first group relation for $G_n$ can be rearranged as follows:
\begin{align}
(a^{-1}b)^{10n}d^{-1}a^{2}&=1\nonumber{}\\
\Rightarrow{}(b^{-1}a)^{10n}&=d^{-1}a^{2},\label{eq1:2}
\end{align}
and the second group relation can be rearranged as follows:
\begin{align}
b^{-2}c(b^{-1}a)^{10n}&=1\nonumber{}\\
\Rightarrow{}(b^{-1}a)^{10n}&=c^{-1}b^{2}.\label{eq2:2}
\end{align}
This shows that:
\begin{align*}
d^{-1}a^{2}&=c^{-1}b^{2}.\qedhere
\end{align*}
\end{proof}

\begin{corollary} In $G_n$, $a^{2}b^{-2}=dc^{-1}$.
\label{corollary:eq6}
\end{corollary}
\begin{proof}By Lemma~\ref{lemma:eq5} we know:
\begin{align*}
d^{-1}a^{2}&=c^{-1}b^{2}\\
\Rightarrow{}a^{2}b^{-2}&=dc^{-1}.\qedhere
\end{align*}
\end{proof}

\begin{lemma} In $G_n$, $d^{2}a^{-1}d=c^{2}b^{-1}c$.
\label{lemma:eq7}
\end{lemma}
\begin{proof}
The third group relation for $G_n$ can be rearranged as follows:
\begin{align*}
(d^{-1}c)^{10n+3}c^{-1}bc^{-2}&=1\\
\Rightarrow{}(d^{-1}c)^{10n+3}&=c^{2}b^{-1}c,
\end{align*}
and the fourth group relation can be rearranged as follows:
\begin{align*}
d^{2}a^{-1}d(c^{-1}d)^{10n+3}&=1\\
\Rightarrow{}(d^{-1}c)^{10n+3}&=d^{2}a^{-1}d.
\end{align*}
Showing that:
\begin{align}
\label{eq7}d^{2}a^{-1}d&=c^{2}b^{-1}c.
\end{align}
\end{proof}

\begin{lemma} In $G_n$, $d^{2}a=c^{2}b$.
\label{lemma:eq8}
\end{lemma}
\begin{proof}
Starting from Lemma~\ref{lemma:eq5}, we have:
\begin{align}
d^{-1}a^{2}&=c^{-1}b^{2}\nonumber{}\\
\Rightarrow{}c&=b^{2}a^{-2}d.\label{eq5:2}
\end{align}
Using (\ref{eq5:2}), we can substitute for the right-most $c$ in (\ref{eq7}). We find:
\begin{align*}
d^{2}a^{-1}d&=c^{2}b^{-1}b^{2}a^{-2}d\\
\Rightarrow{}d^{2}a^{-1}d&=c^{2}ba^{-2}d\\
\Rightarrow{}d^{2}a&=c^{2}b.\qedhere
\end{align*}
\end{proof}

\begin{lemma} In $G_n$, $ad^{-1}a=bc^{-1}b$.
\label{lemma:eq16}
\end{lemma}
\begin{proof} By the first group relation, we have:
\begin{align}
(a^{-1}b)^{10n}d^{-1}a^{2}&=1\nonumber{}\\
\Rightarrow{}(a^{-1})(ba^{-1})^{10n}ad^{-1}a^{2}&=1\nonumber{}\\
\Rightarrow{}(ba^{-1})^{10n}ad^{-1}a&=1.\label{eq1:3}
\end{align}
By the second group relation, we have:
\begin{align}
b^{-2}c(b^{-1}a)^{10n}&=1\nonumber{}\\
\Rightarrow{} b^{-2}cb^{-1}(ab^{-1})^{10n}b&=1\nonumber{}\\
\Rightarrow{}b^{-1}cb^{-1}(ab^{-1})^{10n}&=1\nonumber{}\\
\Rightarrow{}b^{-1}cb^{-1}&=(ba^{-1})^{10n}.\label{eq2:3}
\end{align}
Substituting (\ref{eq2:3}) into (\ref{eq1:3}), we see:
\begin{align*}
b^{-1}cb^{-1}ad^{-1}a&=1\\
\Rightarrow{}ad^{-1}a&=bc^{-1}b.\qedhere
\end{align*}
\end{proof}

\begin{lemma} In $G_n$, $c^{-3}d^{3}=b^{-1}a$.
\label{lemma:eq9}
\end{lemma}
\begin{proof} By the third group relation, we have:
\begin{align}
(d^{-1}c)^{10n+3}c^{-1}bc^{-2}&=1\nonumber{}\\
\Rightarrow{}c^{-1}(cd^{-1})^{10n+3}bc^{-2}&=1\nonumber{}\\
\Rightarrow{}(cd^{-1})^{10n+3}bc^{-3}&=1.\label{eq3:33}
\end{align}
By the fourth group relation, we have:
\begin{align}
d^{2}a^{-1}d(c^{-1}d)^{10n+3}&=1\nonumber{}\\
\Rightarrow{}d^{2}a^{-1}(dc^{-1})^{10n+3}d&=1\nonumber{}\\
\Rightarrow{}d^{3}a^{-1}(dc^{-1})^{10n+3}&=1\nonumber{}\\
\Rightarrow{}d^{3}a^{-1}&=(cd^{-1})^{10n+3}.\label{eq4:33}
\end{align}
Substituting (\ref{eq4:33}) into (\ref{eq3:33}), we see:
\begin{align*}
d^{3}a^{-1}bc^{-3}&=1\\
\Rightarrow{}c^{-3}d^{3}&=b^{-1}a.\qedhere
\end{align*}
\end{proof}

\begin{corollary} If $G_n$ is left-orderable, then $c^{-3}d^{3}>1$.
\label{corollary:inEq5}
\end{corollary}
\begin{proof}This is an immediate consequence of Lemma~\ref{lemma:eq9} and the general assumption that $b^{-1}a>1$.
\end{proof}

\begin{lemma} If $G_n$ is left-orderable, then $d^{-1}a^{2}>1$.
\label{lemma:inEq3}
\end{lemma}
\begin{proof}By (\ref{eq1:2}) we have:
\begin{align*}
(b^{-1}a)^{10n}&=d^{-1}a^{2},
\end{align*}
and since we are assuming $b^{-1}a>1$, this shows that $d^{-1}a^{2}>1$.
\end{proof}

\begin{lemma} If $G_n$ is left-orderable, then $d^{-2}c^{2}b^{-1}cd^{-1}b<1$.
\label{lemma:inEq7}
\end{lemma}
\begin{proof} By  Lemma~\ref{lemma:eq7} we have:
\begin{align}
d^{2}a^{-1}d&=c^{2}b^{-1}c\nonumber{}\\
\Rightarrow{}d^{2}&=c^{2}b^{-1}cd^{-1}a.\label{eq7:2}
\end{align}
By rearranging the result of Lemma~\ref{lemma:eq9} we have:
\begin{align*}
c^{3}=d^{3}a^{-1}b=d(c^{2}b^{-1}cd^{-1}a)a^{-1}b.
\end{align*}
Note that the last equality follows by substituting for $d^{2}$ using (\ref{eq7:2}). This expression can be rearranged to give:
\begin{align}
d&=c^{3}b^{-1}dc^{-1}bc^{-2}\nonumber{}\\
\Rightarrow{}c^{-1}d&=c^2b^{-1}dc^{-1}bc^{-2}.\label{eq13}
\end{align}
Now (\ref{eq13}) can be rearranged to yield:
\begin{align}
d^{-1}c^3b^{-1}dc^{-1}bc^{-2}&=1\nonumber{}\\
\Rightarrow{}(d^{-3}c^3)(b^{-1}dc^{-1}bc^{-2}d^{2})&=1.\label{lemma:inEq7:contradiction}
\end{align}
By Corollary~\ref{corollary:inEq5}, $d^{-3}c^3<1$, therefore (\ref{lemma:inEq7:contradiction}) shows that:
\begin{align*}
b^{-1}dc^{-1}bc^{-2}d^{2}&>1\\
\Rightarrow{}d^{-2}c^{2}b^{-1}cd^{-1}b&<1.\qedhere
\end{align*}
\end{proof}

\begin{lemma} If $G_n$ is left-orderable, then $c^{-1}ba^{-1}d>1$.
\label{lemma:inEq:CbAd}
\end{lemma}
\begin{proof} By Lemma~\ref{lemma:eq16}, we have:
\begin{align}
a^{-1}da^{-1}bc^{-1}b&= 1\nonumber{}\\
\Rightarrow{}c^{-1}ba^{-1}da^{-1}b &=1\nonumber{}\\
\Rightarrow{}(b^{-1}a)(d^{-1}ab^{-1}c) &= 1.\label{lemma:inEq:CbAd:contradiction}
\end{align}
We know $b^{-1}a>1$, so (\ref{lemma:inEq:CbAd:contradiction}) shows that:
\begin{align*}
d^{-1}ab^{-1}c&<1\\
\Rightarrow{}c^{-1}ba^{-1}d&>1.\qedhere
\end{align*}
\end{proof}

\begin{lemma} If $G_n$ is left-orderable, then $b<1$ implies $c<1$.
\label{genImp1}
\end{lemma}
\begin{proof}By Lemma~\ref{lemma:inEq3}, $d^{-1}a^{2}>1$, and hence $a^{-2}d<1$. By Lemma~\ref{lemma:eq5}, we have:
\begin{align*}
d^{-1}a^{2}&=c^{-1}b^{2}\\
\Rightarrow{}c&=b^{2}a^{-2}d,
\end{align*}
and it is now easy to see that:
\begin{align*}
b<1\Rightarrow{}b^{2}<1\Rightarrow{}b^{2}a^{-2}d&<1\Rightarrow{}c<1.\qedhere
\end{align*}
\end{proof}

\section{Cases 2, 5, 6, 7, 9, 10, 11, 12, 13, 14, and 15}
\label{section:manyCases}

\begin{proposition} If $G_n$ is left-orderable, then Cases 9, 10, 11, and 12 are impossible.\label{proposition:case9,10,11,12}
\end{proposition}
\begin{proof} Suppose $G_n$ is left-orderable. In Cases 9, 10, 11, and 12, $a<1$ and $b>1$ and so $a<b$, but we have taken $b^{-1}a$ to be positive, telling us that $a>b$. Therefore, Cases 9, 10, 11, and 12 are not possible.
\end{proof}

\begin{proposition} If $G_n$ is left-orderable, then Cases 2 and 15 are impossible.\label{proposition:case2,15}
\end{proposition}
\begin{proof} Suppose $G_n$ is left-orderable. The fourth group relation for $G_n$ tells us that:
\begin{align}
d^{2}a^{-1}d(c^{-1}d)^{10n+3}&=1.\label{eq4}
\end{align}
Suppose (for contradiction) that the signs of the generators are as in Case 2. Then, $d<1$, while $a>1$ implies $a^{-1}<1$ and $c>1$ implies $c^{-1}<1$. These statements show that:
\begin{align*}
d^{2}a^{-1}d(c^{-1}d)^{10n+3}<1.
\end{align*}
This contradicts (\ref{eq4}); therefore, Case 2 is impossible.\\
\noindent{}Suppose now (for contradiction) that the signs of the generators are as in Case 15. Then, $d>1$, while $a<1$ implies $a^{-1}>1$, and $c<1$ implies $c^{-1}>1$. These statements show that:
\begin{align*}
d^{2}a^{-1}d(c^{-1}d)^{10n+3}>1.
\end{align*}
This contradicts (\ref{eq4}); therefore, Case 15 is impossible.
\end{proof}

\begin{proposition} If $G_n$ is left-orderable, then Case 7 is impossible.
\label{proposition:case7}
\end{proposition}
\begin{proof} Suppose that $G_n$ is left-orderable. By Lemma~\ref{lemma:eq8} we have:
\begin{align*}
d^{2}a&=c^{2}b.
\end{align*}
This shows that if $a$ and $d$ are both positive then $b$ and $c$ cannot both be negative, eliminating Case 7 as a possibility.
\end{proof}

\begin{proposition} If $G_n$ is left-orderable, then Cases 5, 6, 13, and 14 are impossible.\label{proposition:case5,6,13,14}
\end{proposition}
\begin{proof} Suppose that $G_n$ is left-orderable. By Lemma~\ref{genImp1}, Cases 5, 6, 13, and 14 are not possible, since in these cases $b<1$ but $c>1$.
\end{proof}
\vspace{20 pt}
\noindent{}To summarize, the remaining cases are 1, 3, 4, 8, and 16. They are shown in Table~\ref{table:casesPart1}.

\begin{table}[ht]
\begin{center}
\begin{tabular}{l | l | l | l | l}
Case\hspace{10 pt} & $a$\hspace{10 pt} & $b$\hspace{10 pt} & $c$\hspace{10 pt} & $d$\hspace{10 pt} \\\hline\hline
1 & $+$ & $+$ & $+$ & $+$ \\\hline
3 & $+$ & $+$ & $-$ & $+$ \\\hline
4 & $+$ & $+$ & $-$ & $-$ \\\hline
8 & $+$ & $-$ & $-$ & $-$ \\\hline
16 & $-$ & $-$ & $-$ & $-$
\end{tabular}
\end{center}
\caption{The five cases that remain after considering Propositions~\ref{proposition:case9,10,11,12},~\ref{proposition:case2,15},~\ref{proposition:case7}, and~\ref{proposition:case5,6,13,14}.}
\label{table:casesPart1}
\end{table}

%end{part1}
%begin{case3}

\section{Case 3}
\label{section:case3}

\noindent{}We will now show that if $G_n$ is left-orderable then the four generators cannot have the signs shown in Case 3 (see Table~\ref{table:case3}). To accomplish this we will assume that $G_n$ is left-orderable and that the signs of the generators are as in Case 3 and reach a contradiction. 

\begin{table}[ht]
\begin{center}
\begin{tabular}{l | l | l | l | l}
Case\hspace{10 pt} & $a$\hspace{10 pt} & $b$\hspace{10 pt} & $c$\hspace{10 pt} & $d$\hspace{10 pt} \\\hline\hline
3 & $+$ & $+$ & $-$ & $+$ 
\end{tabular}
\end{center}
\caption{The signs of the four generators in Case 3}
\label{table:case3}
\end{table}

\begin{lemma} In Case 3, $ba^{-1}>1$.
\label{lemma:inEq3.1}
\end{lemma}
\begin{proof}By Lemma~\ref{lemma:eq8}, we have:
\begin{align}
c^{2}b&=d^{2}a\nonumber{}\\
\Rightarrow{}(c^{-2}d^{2})(ab^{-1})&=1.\label{eq8:2}
\end{align}
In Case 3, we assume that $c<1$ and $d>1$, thus $c^{-2}d^{2}>1$, and thus by (\ref{eq8:2}) we see that:
\begin{align*}
ab^{-1}&<1\\
\Rightarrow{}ba^{-1}&>1.\qedhere
\end{align*}
\end{proof}

\begin{proposition}
If $G_n$ is left-orderable, then Case 3 ($a>1$, $b>1$, $c<1$, and $d>1$) is impossible.
\end{proposition}
\begin{proof}Suppose that $G_n$ is left-orderable, and suppose (for contradiction) that $a>1$, $b>1$, $c<1$, and $d>1$. By the second group relation, we have:
\begin{align}
b^{-2}c(b^{-1}a)^{10n}&=1\nonumber{}\\
\Rightarrow{}b(a^{-1}b)^{10n}c^{-1}b&=1\nonumber{}\\
\Rightarrow{}(ba^{-1})^{10n}(bc^{-1}b)&=1.\label{proposition:case3:contradiction}
\end{align}
Now by Lemma~\ref{lemma:inEq3.1}, $(ba^{-1})^{10n}>1$. Further, we are assuming that $b>1$ and $c<1$ so $bc^{-1}b>1$. Therefore, (\ref{proposition:case3:contradiction}) states that a product of positive elements equals the identity, a contradiction.
\end{proof}

%end{case3}
%begin{case4}

\section{Case 4}
\label{section:case4}

\noindent{}We will now show that if $G_n$ is left-orderable then the four generators cannot have the signs shown in Table~\ref{table:case4}. To accomplish this we will assume that $G_n$ is left-orderable and that the signs of the generators are as in Case 4 and reach a contradiction.

\begin{table}[ht]
\begin{center}
\begin{tabular}{l | l | l | l | l}
Case\hspace{10 pt} & $a$\hspace{10 pt} & $b$\hspace{10 pt} & $c$\hspace{10 pt} & $d$\hspace{10 pt} \\\hline\hline
4 & $+$ & $+$ & $-$ & $-$ 
\end{tabular}
\end{center}
\caption{The signs of the four generators in Case 4.}
\label{table:case4}
\end{table}

\begin{lemma} In Case 4, $c^{-1}d>1$.
\label{inEq4.1}
\end{lemma}
\begin{proof} By the third group relation, we have:
\begin{align}
(d^{-1}c)^{10n+3}(c^{-1}bc^{-2})&=1.\label{lemma:inEq4.1:contradiction}
\end{align}
In Case 4, $b>1$, and $c^{-1}>1$, thus (\ref{lemma:inEq4.1:contradiction}) shows that $d^{-1}c<1$ or equivalently $c^{-1}d>1$.
\end{proof}

\begin{lemma} In Case 4, $ab^{-1}>1$.
\label{inEq4.2}
\end{lemma}
\begin{proof} By the first group relation, we have:
\begin{align}
(a^{-1}b)^{10n}d^{-1}a^2&=1\nonumber{}\\
\Rightarrow a^{-1}(ba^{-1})^{10n}ad^{-1}a^{2}&=1\nonumber{}\\
\Rightarrow(ba^{-1})^{10n}(a)(d^{-1})(a)&=1.\label{eq1:4}
\end{align}
In Case 4, $a>1$ and $d^{-1}>1$, thus (\ref{eq1:4}) shows that $ba^{-1}<1$, or equivalently $ab^{-1}>1$.
\end{proof}

\begin{lemma} In Case 4, $c^{-2}d^{2}>1$.\label{inEq4.3}
\end{lemma}
\begin{proof} By the third group relation, we have:
\begin{align}
(d^{-1}c)^{10n+3}c^{-1}bc^{-2}&=1\nonumber{}\\
\Rightarrow(bc^{-2})(d^{-1}c)^{10n+3}(c^{-1})&=1\nonumber{}\\
\Rightarrow(bc^{-2})(cc^{-1})((dd^{-1})d^{-1}c)^{10n+3}(cc^{-1})(c^{-1})&=1\nonumber{}\\
\Rightarrow(bc^{-2})c(c^{-1}dd^{-2}cc)^{10n+3}c^{-1}(c^{-1})&=1\nonumber{}\\
\Rightarrow(b)(c^{-1})([c^{-1}d][d^{-2}c^2])^{10n+3}c^{-2}&=1.\label{lemma:inEq4.3:contradiction}
\end{align}
In Case 4, $b>1$, $c^{-1}>1$, and $c^{-1}d>1$ by Lemma~\ref{inEq4.1}. Therefore, (\ref{lemma:inEq4.3:contradiction}) shows that $d^{-2}c^2<1$, or equivalently, $c^{-2}d^2>1$.
\end{proof}

\begin{proposition} If $G_n$ is left-orderable, then Case 4 ($a>1$, $b>1$, $c<1$, and $d<1$) is impossible.
\end{proposition}
\begin{proof} Suppose $G_n$ is left-orderable. By Lemma~\ref{lemma:eq8}, we have:
\begin{align}
d^2a&=c^2b\nonumber{}\\
\Rightarrow b^{-1}c^{-2}d^2a&=1\nonumber{}\\
\Rightarrow (c^{-2}d^2)(ab^{-1})&=1.\label{proposition:case4:contradiction}
\end{align}
But $ab^{-1}>1$ by Lemma~\ref{inEq4.2} and $c^{-2}d^2>1$ by Lemma~\ref{inEq4.3}, thus (\ref{proposition:case4:contradiction}) is a contradiction.
\end{proof}

%end{case4}
%begin{case8}

\section{Case 8}
\label{section:case8}

\noindent{}We will now show that if $G_n$ is left-orderable then the four generators cannot have the signs shown in Table~\ref{table:case8}. To accomplish this we will assume that $G_n$ is left-orderable and that the signs of the generators are as in Case 8 and reach a contradiction. 

\begin{table}[ht]
\begin{center}
\begin{tabular}{l | l | l | l | l}
Case\hspace{10 pt} & $a$\hspace{10 pt} & $b$\hspace{10 pt} & $c$\hspace{10 pt} & $d$\hspace{10 pt} \\\hline\hline
8 & $+$ & $-$ & $-$ & $-$ \\
\end{tabular}
\end{center}
\caption{The signs of the four generators in Case 8.}
\label{table:case8}
\end{table}

\begin{lemma} In Case 8, $bc^{-1}>1$.
\label{lemma:inEq8.2}
\end{lemma}
\begin{proof} In Case 8, we have:
\begin{align}
ab^{-1}>1,\label{inEq8.1}
\end{align} 
since $a>1$ and $b^{-1}>1$. The second group relation of $G_n$ tells us that:
\begin{align}
b^{-2}c(b^{-1}a)^{10n}&=1\nonumber{}\\
\Rightarrow{}b^{-1}c(b^{-1}a)^{10n}b^{-1}&=1\nonumber{}\\
\Rightarrow{}(b^{-1})(cb^{-1})(ab^{-1})^{10n}&=1.\label{lemma:inEq8.2:contradiction}
\end{align}
In Case 8, $b^{-1}>1$ and $ab^{-1}>1$ by (\ref{inEq8.1}), therefore (\ref{lemma:inEq8.2:contradiction}) tells us that:
\begin{align*}
cb^{-1}&<1\\
\Rightarrow{}bc^{-1}&>1.\qedhere
\end{align*}
\end{proof}

\begin{lemma} In Case 8, $c^{-1}b>1$.
\label{lemma:inEq8.3}
\end{lemma}
\begin{proof} The second group relation of $G_n$ tells us that:
\begin{align}
b^{-2}c(b^{-1}a)^{10n}&=1\nonumber{}\\
\Rightarrow{}(b^{-1}c)(b^{-1}a)^{10n}b^{-1}&=1.\label{lemma:inEq8.3:contradiction}
\end{align}
In Case 8, $b<1\Rightarrow{}b^{-1}>1$ and we have assumed in general that $b^{-1}a>1$, thus (\ref{lemma:inEq8.3:contradiction}) tells us that:
\begin{align*}
b^{-1}c&<1\\
\Rightarrow{}c^{-1}b&>1.\qedhere
\end{align*}
\end{proof}

\begin{lemma} In Case 8, $c^{-1}d>1$.
\label{lemma:inEq8.4}
\end{lemma}
\begin{proof} In Case 8, $c<1$ so $c^{-2}>1$, and Lemma~\ref{lemma:inEq8.3} states $c^{-1}b>1$. Therefore, we have:
\begin{align}
c^{-1}bc^{-2}>1.\label{inEq8.5}
\end{align}
The third group relation tells us that:
\begin{align}
(d^{-1}c)^{10n+3}c^{-1}bc^{-2}&=1\nonumber{}\\
\Rightarrow{}(d^{-1}c)^{10n+3}(c^{-1}bc^{-2})&=1.\label{lemma:inEq8.4:contradiction}
\end{align}
Together, (\ref{inEq8.5}) and (\ref{lemma:inEq8.4:contradiction}) show that:
\begin{align*}
d^{-1}c&<1\\
\Rightarrow{}c^{-1}d&>1.\qedhere
\end{align*}
\end{proof}

\begin{lemma} In Case 8, $d^{-2}c^{2}>1$.
\label{lemma:inEq8.6} 
\end{lemma}
\begin{proof} By Lemma~\ref{lemma:eq8}, we have:
\begin{align}
d^{2}a&=c^{2}b\nonumber{}\\
\Rightarrow{}(ab^{-1})(c^{-2}d^{2})&=1.\label{lemma:inEq8.6:contradiction}
\end{align}
In Case 8, $b<1$ and $a>1$ so $ab^{-1}>1$. Therefore, (\ref{lemma:inEq8.6:contradiction}) shows that:
\begin{align*}
c^{-2}d^{2}&<1\\
\Rightarrow{}d^{-2}c^{2}&>1.\qedhere
\end{align*}
\end{proof}

\begin{proposition} If $G_n$ is left-orderable, then Case 8 ($a>1$, $b<1$, $c<1$, and $d<1$) is impossible.
\label{proposition:case8}
\end{proposition}
\begin{proof} Suppose that $G_n$ is left-orderable, and suppose (for contradiction), that $a>1$, $b<1$, $c<1$, and $d<1$.
By the third group relation, we have: 
\begin{align}
(d^{-1}c)^{10n+3}c^{-1}bc^{-2}&=1\nonumber{}\\
\Rightarrow{}(bc^{-2})((d^{-1}c)^{10n+3})(c^{-1})&=1\nonumber{}\\
\Rightarrow{}(bc^{-2})(cc^{-1})(((dd^{-1})d^{-1}c)^{10n+3})(cc^{-1})(c^{-1})&=1\nonumber{}\\
\Rightarrow{}(bc^{-2}c)(c^{-1}dd^{-2}cc)^{10n+3}(c^{-1})(c^{-1})&=1\nonumber{}\\
\Rightarrow{}(bc^{-1})([c^{-1}d][d^{-2}c^{2}])^{10n+3}(c^{-2})&=1.\label{proposition:case8:contradiction}
\end{align}
By Lemma~\ref{lemma:inEq8.2}, $bc^{-1}>1$, by Lemma~\ref{lemma:inEq8.4}, $c^{-1}d>1$, and by Lemma~\ref{lemma:inEq8.6}, $d^{-2}c^{2}>1$. Furthermore, in Case 8 $c<1$ so $c^{-2}>1$. Therefore, (\ref{proposition:case8:contradiction}) states that a product of positives is the identity, a contradiction.
\end{proof}

%end{case8}
%begin{case1}

\section{Case 1}
\label{section:case1}

\noindent{}In order to show Case 1 ($a,b,c,d>1$) is not possible if $G_n$ is left-orderable, we consider sub-cases (see Table~\ref{table:case1.-}).

\begin{table}[ht]
\begin{center}
\begin{tabular}{l | l | l | l }
Case \hspace{10 pt} & $a^{-1}d$\hspace{10 pt} & $b^{-1}d$\hspace{10 pt} & $b^{-1}c$\hspace{10 pt} \\\hline\hline
\case{1}{1}{} & $+$ & $+$ & $+$ \\\hline
\case{1}{2}{} & $+$ & $+$ & $-$ \\\hline
\case{1}{3}{} & $+$ & $-$ & $+$ \\\hline
\case{1}{4}{} & $+$ & $-$ & $-$ \\\hline
\case{1}{5}{} & $-$ & $+$ & $+$ \\\hline
\case{1}{6}{} & $-$ & $+$ & $-$ \\\hline
\case{1}{7}{} & $-$ & $-$ & $+$ \\\hline
\case{1}{8}{} & $-$ & $-$ & $-$ 
\end{tabular}
\end{center}
\caption{Eight sub-cases of Case 1, considering the signs of $a^{-1}d$, $b^{-1}d$, and $b^{-1}c$.}
\label{table:case1.-}
\end{table}

\noindent{}In order to justify that these sub-cases represent all possibilities if $G_n$ is left-orderable, we must first show that $a^{-1}d\neq{}1$, $b^{-1}d\neq{}1$, and $b^{-1}c\neq{}1$.

\subsection{Proof that $a^{-1}d\neq{}1$ in Case 1}
\noindent{}Note: unless otherwise stated, all lemmas in this section assume $G_n$ is left-orderable and assume signs as in Case 1.

\begin{lemma} If $a^{-1}d=1$, then $b^{-1}d>1$.
\label{lemma:id:Ad:Bd}
\end{lemma}
\begin{proof} Suppose $a^{-1}d=1$. We know $b^{-1}a>1$, and
\begin{align*}
(d^{-1}b)(b^{-1}a)=d^{-1}a=1.
\end{align*}
This shows that $d^{-1}b<1$, or equivalently that $b^{-1}d>1$.
\end{proof}

\begin{lemma} If $a^{-1}d=1$, then $d^{-1}c>1$.
\label{lemma:id:Ad:Dc}
\end{lemma}
\begin{proof} Suppose $a^{-1}d=1$. Starting from the third group relation, we have:
\begin{align*}
(d^{-1}c)^{10n+3}c^{-1}bc^{-2}&=1\\
\Rightarrow{}c^{2}b^{-1}c(c^{-1}d)^{10n+3}&=1\\
\Rightarrow{}c^{2}(b^{-1}d)(c^{-1}d)^{10n+2}&=1.
\end{align*}
But $b^{-1}d>1$ by Lemma~\ref{lemma:id:Ad:Bd} and $c>1$ by assumption in Case 1. Therefore, $c^{-1}d$ must be negative, showing that $d^{-1}c>1$.
\end{proof}

\begin{lemma} If $a^{-1}d=1$, then $a^{-1}c>1$.
\label{lemma:id:Ad:Ac}
\end{lemma}
\begin{proof} Suppose $a^{-1}d=1$, then:
\begin{align*}
(d^{-1}c)(c^{-1}a)=d^{-1}a=1.
\end{align*}
This shows that $c^{-1}a<1$ since $d^{-1}c>1$ by Lemma~\ref{lemma:id:Ad:Dc}, therefore $a^{-1}c>1$.
\end{proof}

\begin{proposition} If $G_n$ is left-orderable, then $a^{-1}d\neq{}1$ in Case 1.
\label{proposition:id:Ad}
\label{proposition:id:dA}
\end{proposition}
\begin{proof} Suppose $G_n$ is left-orderable and suppose (for contradiction) that $a^{-1}d=d^{-1}a=1$. By Lemma~\ref{lemma:eq16}, we have:
\begin{align}
ad^{-1}a&=bc^{-1}b\nonumber{}\\
\Rightarrow{}b^{-1}a(d^{-1}a)b^{-1}c&=1\nonumber{}\\
\Rightarrow{}b^{-1}a(1)b^{-1}c&=1\\
\Rightarrow{}(b^{-1}a)b^{-1}(aa^{-1})c&=1\nonumber{}\\
\Rightarrow{}(b^{-1}a)(b^{-1}a)(a^{-1}c)&=1.\label{proposition:id:Ad:contradiction}
\end{align}
Now $b^{-1}a>1$ by assumption and $a^{-1}c>1$ by Lemma~\ref{lemma:id:Ad:Ac}, thus (\ref{proposition:id:Ad:contradiction}) is a contradiction.
\end{proof}

\subsection{Proof that $b^{-1}d\neq{}1$ in Case 1}
\noindent{}Note: unless otherwise stated, all lemmas in this section assume $G_n$ is left-orderable and assume signs as in Case 1.

\begin{lemma} If $b^{-1}d=1$, then $d^{-1}a>1$.
\label{lemma:id:Bd:Da}
\end{lemma}
\begin{proof} Suppose $b^{-1}d=1$, then:
\begin{align*}
(b^{-1}a)(a^{-1}d)=b^{-1}d=1.
\end{align*}
This shows that $a^{-1}d<1$ since $b^{-1}a>1$ by general assumption. Therefore, $d^{-1}a>1$.
\end{proof}

\begin{lemma} If $b^{-1}d=1$, then $c^{-1}b>1$.
\label{lemma:id:Bd:Cb}
\end{lemma}
\begin{proof} Suppose $b^{-1}d=1$. By Lemma~\ref{lemma:eq8}, we have
\begin{align}
c^{2}b&=d^{2}a\nonumber{}\\
\Rightarrow{}b^{-1}c^{-2}d^{2}a&=1\nonumber{}\\
\Rightarrow{}b^{-1}(cc^{-1})c^{-2}d^{2}(dd^{-1})a&=1\nonumber{}\\
\Rightarrow{}(b^{-1}c)(c^{-3}d^{3})(d^{-1}a)&=1.\label{lemma:id:Bd:Cb:contradiction}
\end{align}
Now $d^{-1}a>1$ by Lemma~\ref{lemma:id:Bd:Da} and $c^{-3}d^{3}>1$ by Corollary~\ref{corollary:inEq5}. Therefore, (\ref{lemma:id:Bd:Cb:contradiction}) shows that
\begin{align*}
b^{-1}c&<1\\
\Rightarrow{}c^{-1}b&>1.\qedhere
\end{align*}
\end{proof}

\begin{lemma} If $b^{-1}d=1$, then $c^{-1}d>1$.
\label{lemma:id:Bd:Cd}
\end{lemma}
\begin{proof} Suppose $b^{-1}d=1$, then
\begin{align*}
(d^{-1}c)(c^{-1}b)&=d^{-1}b=1\\
\Rightarrow{}c^{-1}d&=c^{-1}b>1.
\end{align*}
Where the last inequality follows from Lemma~\ref{lemma:id:Bd:Cb}.
\end{proof}

\begin{lemma} If $b^{-1}d=1$, then $ab^{-1}>1$.
\label{lemma:id:Bd:aB}
\end{lemma}
\begin{proof} Suppose $b^{-1}d=1$. Starting from the first group relation, we have:
\begin{align*}
(a^{-1}b)^{10n}d^{-1}a^{2}&=1\\
\Rightarrow{}(a^{-1}b)^{10n-1}(a^{-1}b)d^{-1}a^{2}&=1\\
\Rightarrow{}a^{-1}(ba^{-1})^{10n-1}bd^{-1}a^{2}&=1\\
\Rightarrow{}(ba^{-1})^{10n-1}b(d^{-1}a)&=1.
\end{align*}
This shows $ba^{-1}<1$ since $b>1$ by assumption in Case 1 and  $d^{-1}a>1$ by Lemma~\ref{lemma:id:Bd:Da}. Therefore, $ab^{-1}>1$.
\end{proof}

\begin{lemma} If $b^{-1}d=1$, then $da^{-1}b>1$.
\label{lemma:id:Bd:dAb}
\end{lemma}
\begin{proof}  Suppose $b^{-1}d=1$. By Lemma~\ref{lemma:eq16}, we have:
\begin{align*}
ad^{-1}a&=bc^{-1}b\\
\Rightarrow{}cb^{-1}ad^{-1}ab^{-1}&=1\\
\Rightarrow{}(c)(b^{-1}ad^{-1})(ab^{-1})&=1.
\end{align*}
This shows that $b^{-1}ad^{-1}<1$, since $c>1$ in Case 1 by assumption and $ab^{-1}>1$ by Lemma~\ref{lemma:id:Bd:aB}. Therefore, $da^{-1}b>1$.
\end{proof}

\begin{lemma} If $b^{-1}d=1$, then $da^{-1}d>1$.
\label{lemma:id:Bd:dAd}
\end{lemma}
\begin{proof}  Suppose $b^{-1}d=1$. Then
\begin{align*}
(d^{-1}ad^{-1})(da^{-1}b)&=d^{-1}b=1\\
\Rightarrow{}da^{-1}d&=da^{-1}b>1.
\end{align*}
This final inequality follows from Lemma~\ref{lemma:id:Bd:dAb}.
\end{proof}

\begin{proposition} If $G_n$ is left-orderable, then $b^{-1}d\neq{}1$ in Case 1.
\label{proposition:id:Bd}
\end{proposition}
\begin{proof} Suppose $G_n$ is left-orderable, and suppose (for contradiction) that $b^{-1}d=1$. Starting from the fourth group relation, we have:
\begin{align*}
d^{2}a^{-1}d(c^{-1}d)^{10n+3}&=1\\
\Rightarrow{}(c^{-1}d)^{10n+3}d^{2}a^{-1}d&=1\\
\Rightarrow{}(c^{-1}d)^{10n+2}(c^{-1}d)d^{2}a^{-1}d&=1\\
\Rightarrow{}(c^{-1}d)^{10n+2}c^{-1}(bb^{-1})d(cc^{-1})d^{2}a^{-1}d&=1\\
\Rightarrow{}(c^{-1}d)^{10n+2}(c^{-1}b)(b^{-1}dc)(c^{-1}d)(da^{-1}d)&=1\\
\Rightarrow{}(c^{-1}d)^{10n+2}(c^{-1}b)(c)(c^{-1}d)(da^{-1}d)&=1.
\end{align*}
We use the fact that $b^{-1}d=1$ for the last implication. This is a contradiction, since $c^{-1}d>1$ by Lemma~\ref{lemma:id:Bd:Cd}, $c^{-1}b>1$ by Lemma~\ref{lemma:id:Bd:Cb}, $c>1$ by assumption in Case 1, and $da^{-1}d>1$ by Lemma~\ref{lemma:id:Bd:dAd}.
\end{proof}

\subsection{Proof that $b^{-1}c\neq{}1$ in Case 1}
\noindent{}Note: unless otherwise stated, all lemmas in this section assume $G_n$ is left-orderable and assume signs as in Case 1.

\begin{lemma} If $b^{-1}c=1$, then $c^{-1}a>1$.
\label{lemma:id:Bc:Ca}
\end{lemma}
\begin{proof} Suppose that $b^{-1}c=1$, then:
\begin{align*}
(b^{-1}a)(a^{-1}c)&=b^{-1}c=1\\
\Rightarrow{}c^{-1}a&=b^{-1}a>1.\qedhere
\end{align*}
\end{proof}

\begin{lemma} If $b^{-1}c=1$, then $d^{-1}a>1$.
\label{lemma:id:Bc:Da}
\end{lemma}
\begin{proof} Suppose that $b^{-1}c=1$. Starting from the third group relation, we have:
\begin{align*}
(d^{-1}c)^{10n+3}c^{-1}bc^{-2}&=1\\
\Rightarrow{}c^{2}b^{-1}c(c^{-1}d)^{10n+3}&=1\\
\Rightarrow{}c^{2}b^{-1}d(c^{-1}d)^{10n+2}&=1\\
\Rightarrow{}c^{2}b^{-1}(aa^{-1})d(c^{-1}(aa^{-1})d)^{10n+2}&=1\\
\Rightarrow{}c^{2}(b^{-1}a)(a^{-1}d)([c^{-1}a][a^{-1}d])^{10n+2}&=1.
\end{align*}
This shows that $a^{-1}d<1$ since $b^{-1}a>1$ by assumption, $c>1$ by assumption in Case 1, and $c^{-1}a>1$ by Lemma~\ref{lemma:id:Bc:Ca}. Therefore, $d^{-1}a>1$.
\end{proof}

\begin{proposition} If $G_n$ is left-orderable, then $b^{-1}c\neq{}1$ in Case 1.
\label{proposition:id:Bc}
\label{proposition:id:cB}
\end{proposition}
\begin{proof} Suppose $G_n$ is left-orderable, and suppose (for contradiction) that $b^{-1}c=1$. By Lemma~\ref{lemma:eq16}, we have:
\begin{align*}
ad^{-1}a&=bc^{-1}b\\
\Rightarrow{}(b^{-1}c)(b^{-1}a)(d^{-1}a)&=1\\
\Rightarrow{}(b^{-1}a)(d^{-1}a)&=1,\\
\end{align*}
where the last implication uses the fact that $b^{-1}c=1$. This is a contradiction, since $b^{-1}a>1$ by assumption and $d^{-1}a>1$ by Lemma~\ref{lemma:id:Bc:Da}.
\end{proof}
$\;$
\subsection{\Case{1}{2}{}-\case{1}{7}{}}

Now that we have justified the totality of our cases, we continue by disproving left-orderability in each case. We start by showing that if $G_n$ is left-orderable, then \Cases{1}{2}{}-\case{1}{7}{} are not possible.

%begin{case1.ii-VII}

\begin{proposition} If $G_n$ is left-orderable, then \Cases{1}{3}{} and \case{1}{4}{} are impossible.
\end{proposition}
\begin{proof} $\;$ Suppose that $G_n$ is left-orderable, and suppose (for contradiction) that $a^{-1}d>1$ and $b^{-1}d<1$, then:
\begin{align*}
(a^{-1}d)(d^{-1}b)=a^{-1}b&>1\\
\Rightarrow{}b^{-1}a&<1,
\end{align*}
which contradicts the general assumption $b^{-1}a>1$.
\end{proof}

\begin{proposition} If $G_n$ is left-orderable, then \Cases{1}{5}{} and \case{1}{7}{} are impossible.
\end{proposition}
\begin{proof} $\;$ Suppose that $G_n$ is left-orderable, and suppose (for contradiction) that $b^{-1}c>1$ and $a^{-1}d<1$, then:
\begin{align*}
(c^{-1}b)(a^{-1}d)=b^{-1}(bc^{-1}b)a^{-1}d=b^{-1}(bc^{-1}b)(a^{-1}da^{-1})a=b^{-1}a.
\end{align*}
Where the last equality follows from Lemma~\ref{lemma:eq16}. This shows that \Cases{1}{5}{} and \case{1}{7}{} are impossible since $b^{-1}a>1$ by assumption, but $c^{-1}b<1$ and $a^{-1}d<1$ in \Cases{1}{5}{} and \case{1}{7}{}.
\end{proof}

\begin{proposition} If $G_n$ is left-orderable, then \Case{1}{2}{} is impossible.
\end{proposition}
\begin{proof} $\;$ Suppose that $G_n$ is left-orderable. By the third group relation, we have:
\begin{align}
(c^{-1}d)^{10n+3}c^2b^{-1}c&=1\nonumber{}\\
\Rightarrow{}c^2b^{-1}c(c^{-1}d)^{10n+3}&=1\nonumber{}\\
\Rightarrow{}c^2b^{-1}(dc^{-1})^{10n+3}c&=1\label{eq3:2}\\
\Rightarrow{}c^2b^{-1}(dc^{-1})^{10n+2}d&=1\nonumber{}\\
\Rightarrow{}c^2[b^{-1}(dc^{-1})b]^{10n+2}(b^{-1}d)&=1\label{eq3:3}\\
\Rightarrow{}(c^2)([b^{-1}a][a^{-1}d][c^{-1}b])^{10n+2}(b^{-1}a)(a^{-1}d)&=1.\nonumber{}
\end{align}
This shows that \Case{1}{2}{} is impossible, since $a^{-1}d>1$, $b^{-1}d>1$, $b^{-1}c<1$, and $c>1$ in \Case{1}{2}{}.
\end{proof}

\begin{proposition} If $G_n$ is left-orderable, then \Case{1}{6}{} is impossible.
\end{proposition}
\begin{proof} $\;$ Assume that $G_n$ is left-orderable. By (\ref{eq3:3}) we have:
\begin{align*}
c^2[b^{-1}(dc^{-1})b]^{10n+2}(b^{-1}d)&=1\\
\Rightarrow{}c^2[(b^{-1}d)(c^{-1}b)]^{10n+2}(b^{-1}d)&=1.
\end{align*}
This shows that \Case{1}{6}{} is impossible, since $b^{-1}d>1$, $b^{-1}c<1$, and $c>1$ in \Case{1}{6}{}.
\end{proof}
%end{case1.ii-VII}
%begin{case1.i}
%begin{case1.i.2-7}

\subsection{\Case{1}{1}{}}
\noindent{}We now show that if $G_n$ is left-orderable, then \Case{1}{1}{} is impossible. To accomplish this, we consider eight new sub-cases (see Table~\ref{table:case1.i.-}).

\begin{table}[ht]
\begin{center}
\begin{tabular}{l | l | l | l }
Case\hspace{10 pt} & $ca^{-1}$\hspace{10 pt} & $da^{-1}$\hspace{10 pt} & $cb^{-1}$\hspace{10 pt} \\\hline\hline
\case{1}{1}{1} & $+$ & $+$ & $+$ \\\hline
\case{1}{1}{2} & $+$ & $+$ & $-$ \\\hline
\case{1}{1}{3} & $+$ & $-$ & $+$ \\\hline
\case{1}{1}{4} & $+$ & $-$ & $-$ \\\hline
\case{1}{1}{5} & $-$ & $+$ & $+$ \\\hline
\case{1}{1}{6} & $-$ & $+$ & $-$ \\\hline
\case{1}{1}{7} & $-$ & $-$ & $+$ \\\hline
\case{1}{1}{8} & $-$ & $-$ & $-$ 
\end{tabular}
\end{center}
\caption{Eight sub-cases of \Case{1}{1}{}, considering the signs of $ca^{-1}$, $da^{-1}$, and $cb^{-1}$.}
\label{table:case1.i.-}
\end{table}

\noindent{}Since we are working in a sub-case of \Case{1}{1}{}, we also know $a>1$, $b>1$, $c>1$, $d>1$, and the following:
\begin{align} a^{-1}d>1\label{case1.i:inEq:Ad}\\
b^{-1}d>1\label{case1.i:inEq:Bd}\\
b^{-1}c>1
\end{align}

\noindent{}As before, we must first justify that the cases shown in Table~\ref{table:case1.i.-} represent all possibilities if $G_n$ is left-orderable.

\subsubsection{Proof that $ca^{-1}\neq{}1$, $da^{-1}\neq{}1$, and $cb^{-1}\neq{}1$ in \Case{1}{1}{}}

\noindent{}Note: unless otherwise stated, all lemmas in this section assume $G_n$ is left-orderable and assume signs as in \Case{1}{1}{}.

\begin{lemma} If $ca^{-1}=1$, then $d^{-1}c>1$.
\label{lemma:id:cA:Dc}
\end{lemma}
\begin{proof} Suppose $ca^{-1}=1$. Starting from the third group relation, we have:
\begin{align*}
(d^{-1}c)^{10n+3}c^{-1}bc^{-2}&=1\\
\Rightarrow{}c^{2}b^{-1}c(c^{-1}d)^{10n+3}&=1\\
\Rightarrow{}c^{2}b^{-1}d(c^{-1}d)^{10n+2}&=1\\
\Rightarrow{}c^{2}b^{-1}(aa^{-1})d(c^{-1}d)^{10n+2}&=1\\
\Rightarrow{}c^{2}(b^{-1}a)(a^{-1}d)(c^{-1}d)^{10n+2}&=1.
\end{align*}
This shows that $c^{-1}d<1$ since $b^{-1}a>1$ by assumption, $c>1$ by assumption in Case 1, and $a^{-1}d>1$ by assumption in \Case{1}{1}{}. Therefore, $d^{-1}c>1$.
\end{proof}

\begin{proposition} If $G_n$ is left-orderable, then $ca^{-1}\neq{}1$ in \Case{1}{1}{}.
\label{proposition:id:cA}
\end{proposition}
\begin{proof} Suppose $G_n$ is left-orderable, and suppose (for contradiction) that $ca^{-1}=1$, then:
\begin{align*}
(a^{-1}d)(d^{-1}c)=a^{-1}c=1.
\end{align*}
This is a contradiction, since $a^{-1}d>1$ by assumption in \Case{1}{1}{}, and $d^{-1}c>1$ by Lemma~\ref{lemma:id:cA:Dc}.
\end{proof}

We also know that $a^{-1}d\neq{}1$ by Proposition (\ref{proposition:id:Ad}) and $cb^{-1}\neq{}1$ by Proposition (\ref{proposition:id:Bc}).

\subsubsection{\Case{1}{1}{2}-\case{1}{1}{7}}
\noindent{}Now that we have justified the totality of our cases, we continue by disproving left-orderability in each case. We start by showing that if $G_n$ is left-orderable, then \Cases{1}{1}{2}-\case{1}{1}{7} are impossible.

\begin{proposition} If $G_n$ is left-orderable, then \Case{1}{1}{2} is impossible.
\end{proposition}
\begin{proof} $\;$ Suppose that $G_n$ is left-orderable, and suppose (for contradiction) that $ca^{-1}>1$, $da^{-1}>1$ and $bc^{-1}>1$, then: 
\begin{align*}
(ca^{-1})(da^{-1})(bc^{-1})(bc^{-1})>1\\
\Rightarrow{}c(a^{-1}da^{-1})(bc^{-1}b)c^{-1}>1.
\end{align*}
But by Lemma~\ref{lemma:eq16}, we have:
\begin{align*}
c(a^{-1}da^{-1})(bc^{-1}b)c^{-1}=cc^{-1}=1,
\end{align*}
a contradiction.
\end{proof}

\begin{lemma} In \Case{1}{1}{}, $d^{-1}c>1$.\label{case1.1:Dc}
\end{lemma}
\begin{proof} By the fourth group relation, we have:
\begin{align}
d^{2}a^{-1}d(c^{-1}d)^{10n+3}=1\nonumber{}\\
\Rightarrow{}(d^{2})(a^{-1}d)(c^{-1}d)^{10n+3}=1.\label{case1:1:Dc}
\end{align}
Now, $d^{2}>1$ by Case 1, and $a^{-1}d>1$ by \Case{1}{1}{}. Therefore (\ref{case1:1:Dc}) shows that:
\begin{align*}
c^{-1}d&<1\\
\Rightarrow{}d^{-1}c&>1.\qedhere
\end{align*}
\end{proof}

\begin{proposition} If $G_n$ is left-orderable, then \Case{1}{1}{3} is impossible.
\end{proposition}
\begin{proof} Suppose that $G_n$ is left-orderable, and suppose (for contradiction) that $ca^{-1}>1$, $da^{-1}<1$ and $cb^{-1}>1$. Then by Lemma \ref{case1.1:Dc}, $d^{-1}c>1$ and:
\begin{align}
(ad^{-1})(d^{-1}c)(cb^{-1})(ca^{-1})(ad^{-1})&>1\nonumber{}\\
\Rightarrow{}(a)(d^{-2}c^{2})(b^{-1})(cd^{-1})&>1\nonumber{}\\
\Rightarrow{}(a)(ab^{-1})(b^{-1})(cd^{-1})=(a^{2}b^{-2})(cd^{-1})&>1,\label{proposition:case1.i.3:contradiction}
\end{align}
where the last implication follows from Lemma~\ref{lemma:eq8}, which tells us that $d^{2}a=c^{2}b$, implying $d^{-2}c^{2}=ab^{-1}$. Now by Corollary~\ref{corollary:eq6} we know:
\begin{align*}
a^{2}b^{-2}&=dc^{-1}\\
\Rightarrow{}(a^{2}b^{-2})(cd^{-1})&=1.
\end{align*}
This contradicts (\ref{proposition:case1.i.3:contradiction}).
\end{proof}

\begin{proposition} If $G_n$ is left-orderable, then \Case{1}{1}{4} is impossible.
\end{proposition}
\begin{proof} Suppose $G_n$ is left-orderable, and suppose (for contradiction) that $ca^{-1}>1$, $da^{-1}<1$ and $cb^{-1}<1$, then
\begin{align*}
[(bc^{-1})(ca^{-1})]^{10n-1}(bc^{-1})(ca^{-1})(ad^{-1})(a)&>1\\
\Rightarrow{}[(ba^{-1})]^{10n}bd^{-1})&>1\\
\Rightarrow{}aa^{-1}(ba^{-1})^{10n}d^{-1}a&>1\\
\Rightarrow{}aa^{-1}b(a^{-1}b)^{10n-1}a^{-1}bd^{-1}a&>1\\
\Rightarrow{}a(a^{-1}b)^{10n}d^{-1}a>1\\
\end{align*}
this contradicts the first group relation, which says $(a^{-1}b)^{10n}d^{-1}a^{2}=1$ or equivalently $a(a^{-1}b)^{10n}d^{-1}a=1$.
\end{proof}

\begin{proposition} If $G_n$ is left-orderable, then \Case{1}{1}{5} is impossible.
\end{proposition}
\begin{proof} Suppose that $G_n$ is left-orderable, and suppose (for contradiction) that $ca^{-1}<1$, $da^{-1}>1$ and $cb^{-1}>1$, then:
\begin{align*}
(c)(cb^{-1})[(da^{-1})(ac^{-1})]^{10n+2}(d)(c^{-1}c)&>1\\
\Rightarrow{}(c^{2}b^{-1})[(dc^{-1})]^{10n+2}(dc^{-1})(c)&>1\\
\Rightarrow{}c^{2}b^{-1}(dc^{-1})^{10n+3}c&>1.
\end{align*}
This contradicts (\ref{eq3:2}), which says that
\begin{align*}
c^{2}b^{-1}(dc^{-1})^{10n+3}c&=1\qedhere
\end{align*}
\end{proof}

\begin{proposition} If $G_n$ is left-orderable, then \Case{1}{1}{6} is impossible.
\end{proposition}
\begin{proof} Suppose $G_n$ is left-orderable, and suppose (for contradiction) that $ca^{-1}<1$ and $da^{-1}>1$, then
\begin{align}
[(da^{-1})(ac^{-1})]^{10n+3}(d^{2})(da^{-1})&>1\nonumber{}\\
\Rightarrow{}(dc^{-1})^{10n+3}d^{3}a^{-1}&>1,\label{proposition:case1.i.6:contradiction}
\end{align}
but by the fourth group relation, we have:
\begin{align}
d^{2}a^{-1}d(c^{-1}d)^{10n+3}&=1\nonumber{}\\
\Rightarrow{}d^{2}a^{-1}(dc^{-1})^{10n+3}d&=1\nonumber{}\\
\Rightarrow{}(dc^{-1})^{10n+3}d^{3}a^{-1}&=1,\label{eq4:2}
\end{align}
this contradicts (\ref{proposition:case1.i.6:contradiction}).
\end{proof}

\begin{proposition} If $G_n$ is left-orderable, then \Case{1}{1}{7} is impossible.
\end{proposition}
\begin{proof} Suppose that $G_n$ is left-orderable, and suppose (for contradiction) that $ca^{-1}<1$, $da^{-1}<1$, and $cb^{-1}>1$, then: 
\begin{align*}
(cb^{-1})(cb^{-1})(ad^{-1})(ac^{-1})&>1\\
\Rightarrow{}(c)(b^{-1}cb^{-1})(ad^{-1}a)(c^{-1})&>1\\
\Rightarrow{}cc^{-1}=1&>1,
\end{align*}
where the last implication follows from Lemma~\ref{lemma:eq16}.
\end{proof}

%end{case1.i.2-7}
%begin{case1.i.1}

\subsubsection{\protect\Case{1}{1}{1}}

\noindent{}We now show that if $G_n$ is left-orderable, then \Case{1}{1}{1} (see Table~\ref{table:case1.i.-}) is impossible.

\begin{lemma} In \Case{1}{1}{1}, $b^{-1}cd^{-1}b>1$.\label{case1.i.1:inEq:BcDb}
\end{lemma}
\begin{proof} By the third group relation, we have:
\begin{align}
(d^{-1}c)^{10n+3}c^{-1}bc^{-2}&=1\nonumber{}\\
\Rightarrow{}(cd^{-1})^{10n+3}bc^{-3}=(cd^{-1})^{10n+3}(bc^{-1})c^{-2}&=1\nonumber{}\\
\Rightarrow{}cd^{-1}&>1,\label{case1.i.1:inEq:cD}
\end{align}
where the last implication follows from the fact that $bc^{-1}<1$ and $c^{-1}<1$ in \Case{1}{1}{1}. Now by the third group relation, we have:
\begin{align}
(d^{-1}c)^{10n+3}c^{-1}bc^{-2}&=1\nonumber{}\\
\Rightarrow{}c^{2}b^{-1}c(c^{-1}d)^{10n+3}&=1\nonumber{}\\
\Rightarrow{}c^{2}b^{-1}cd^{-1}(dc^{-1})^{10n+3}d&=1\nonumber{}\\
\Rightarrow{}c(cb^{-1})(cd^{-1})b(b^{-1}dc^{-1}b)^{10n+3}(b^{-1}d)&=1.\label{lemmaBcDb:contradiction}
\end{align}
By (\ref{case1.i:inEq:Bd}) and (\ref{case1.i.1:inEq:cD}) it is easy to see that all expressions in parentheses in (\ref{lemmaBcDb:contradiction}) are positive except for $(b^{-1}dc^{-1}b)^{10n+3}$. This tells us that:
\begin{align*}
b^{-1}dc^{-1}b&<1\\
\Rightarrow{}b^{-1}cd^{-1}b&>1.\qedhere
\end{align*}
\end{proof}

\begin{lemma} In \Case{1}{1}{1}, $d^{-1}c^{-1}d^{2}>1$.\label{lemma:inEq:DCdd}
\end{lemma}
\begin{proof} $\;$ By (\ref{eq4:2}), we have:
\begin{align*}
(d^{-1}a)d^{-2}(d^{-1}c)^{10n+3}=1.
\end{align*}
However, by (\ref{case1.i:inEq:Ad}), $d^{-1}a<1$ (as is $d^{-2}$), so (\ref{eq4:2}) shows that:
\begin{align}
d^{-1}c>1.\label{case1.i:inEq:Dc}
\end{align}
Now consider:
\begin{align}
(d^{-2}cd)(d^{-1}c)(b^{-1}cd^{-1}b)&=d^{-2}c^{2}b^{-1}cd^{-1}b<1,\label{lemma:DCdd:contradiction}
\end{align}
where the last inequality follows from Lemma~\ref{lemma:inEq7}. By (\ref{case1.i:inEq:Dc}) and Lemma~\ref{case1.i.1:inEq:BcDb} we see that $(d^{-1}c)(b^{-1}cd^{-1}b)>1$, therefore (\ref{lemma:DCdd:contradiction}) shows that:
\begin{align*}
d^{-2}cd<1\\
\Rightarrow{}d^{-1}c^{-1}d^{2}>1.\qedhere
\end{align*}
\end{proof}

\begin{corollary} In \Case{1}{1}{1}, $d^{-1}c^{-1}dc>1$ and $b^{-1}c^{-1}d^{2}>1$.\label{corollary:inEq:DCdd}
\end{corollary}
\begin{proof} $\;$ These are immediate consequences of Lemma~\ref{lemma:inEq:DCdd} since $d^{-1}c>1$ in \Case{1}{1}{} (by (\ref{case1.i:inEq:Dc})) and $b^{-1}d>1$ in \Case{1}{1}{} (by (\ref{case1.i:inEq:Bd})).
\end{proof}

\begin{proposition} If $G_n$ is left-orderable, then \Case{1}{1}{1} ($ca^{-1}>1$, $da^{-1}>1$, and $cb^{-1}>1$) is impossible.
\end{proposition}
\begin{proof} $\;$ Suppose $G_n$ is left-orderable, and suppose (for contradiction) that $ca^{-1}>1$, $da^{-1}>1$, and $cb^{-1}>1$. By the third group relation we have:
\begin{align}
(d^{-1}c)^{10n+3}c^{-1}bc^{-2}&=1\nonumber{}\\
\Rightarrow{}cb^{-1}c(c^{-1}d)^{10n+3}c&=1\nonumber{}\\
\Rightarrow{}c(b^{-1}d)(c^{-1}d)^{10n+2}c&=1\nonumber{}\\
\Rightarrow{}(cb^{-1})d(c^{-1}d)(c^{-1}d)^{10n}(c^{-1}d)c&=1\nonumber{}\\
\Rightarrow{}(cb^{-1})d(bb^{-1})c^{-1}d(dd^{-1})(c^{-1}d)^{10n}(dd^{-1})c^{-1}dc&=1\nonumber{}\\
\Rightarrow{}(cb^{-1})(db)(b^{-1}c^{-1}d^{2})d^{-1}(c^{-1}d)^{10n}d(d^{-1}c^{-1}dc)&=1\nonumber{}\\
\Rightarrow{}(cb^{-1})(db)(b^{-1}c^{-1}d^{2})(d^{-1}c^{-1}d^{2})^{10n}(d^{-1}c^{-1}dc)&=1.\label{proposition:1.i.1:contradiction}
\end{align}
Now $cb^{-1}>1$ by assumption in \Case{1}{1}{1}. Similarly, $d>1$ and $b>1$ by assumption in Case 1, thus $db>1$. The remaining terms in parentheses in (\ref{proposition:1.i.1:contradiction}) are positive by Lemma~\ref{lemma:inEq:DCdd} and Corollary~\ref{corollary:inEq:DCdd}. We have therefore reached a contradiction, proving that if $G_n$ is left-orderable, then \Case{1}{1}{1} is impossible.
\end{proof}

%end{case1.i.1}
%begin{case1.i.8}

\subsubsection{\Case{1}{1}{8}}

\noindent{}Next we show that if $G_n$ is left-orderable, then \Case{1}{1}{8} is impossible. After Proposition~\ref{proposition:case1.i.8}, all sub-cases of \Case{1}{1}{} will have been eliminated, showing that \Case{1}{1}{} is impossible if $G_n$ is left-orderable.

\begin{lemma} In \Case{1}{1}{8}, $ab^{-1}>1$.
\label{eq8iL}
\end{lemma}
\begin{proof} By the second group relation, we have:
\begin{align*}
b^{-2}c(b^{-1}a)^{10n}&=1\\
\Rightarrow{}(b^{-1})(cb^{-1})(ab^{-1})^{10n}&=1\\
\Rightarrow{}ab^{-1}&>1,
\end{align*}
where the last implication follows from $b^{-1}<1$ (in Case 1), and $cb^{-1}<1$ (in \Case{1}{1}{8}).
\end{proof}

\begin{proposition} If $G_n$ is left-orderable, then \Case{1}{1}{8} ($ca^{-1}<1$, $da^{-1}<1$, and $cb^{-1}<1$) is impossible.\label{proposition:case1.i.8}
\end{proposition}
\begin{proof} $\;$Suppose $G_n$ is left-orderable and suppose (for contradiction), that $ca^{-1}<1$, $da^{-1}<1$, and $cb^{-1}<1$. By Corollary~\ref{corollary:eq6}, we have:
\begin{align}
a^{2}b^{-2}&=dc^{-1}\nonumber{}\\
\Rightarrow{}(b^{-1})(cd^{-1})(a^{2}b^{-1})&=1\nonumber{}\\
\Rightarrow{}(b^{-1}a)(a^{-1}c)(d^{-1}a)(ab^{-1})&=1\nonumber{}\\
\Rightarrow{}(a^{-1}c)(d^{-1}a)&<1,\label{eqproposition8I}
\end{align}
where the last implication follows from the general assumption $(b^{-1}a)>1$, and since $ab^{-1}>1$ by Lemma~\ref{eq8iL}. Nevertheless, by (\ref{eq3:2}):
\begin{align*}
(c^{2}b^{-1})(dc^{-1})^{10n+3}(c)&=1\\
\Rightarrow{}(c^{2})(b^{-1}a)(a^{-1}dc^{-1}a)^{10n+2}(a^{-1}d)(c^{-1}c)&=1\\
\Rightarrow{}(a^{-1}d)(c^{-1}a)&<1,
\end{align*}
where the last implication follows from $b^{-1}a>1$, $c^{2}>1$ (in Case 1), and $a^{-1}d>1$ (in \Case{1}{1}{}), i.e. $(a^{-1}c)(d^{-1}a)>1$, which contradicts (\ref{eqproposition8I}). Therefore if $G_n$ is left-orderable \Case{1}{1}{8} is impossible.
\end{proof}

%end{case1.i.8}
%end{case1.i}
%begin{case1.viii}

\subsection{\Case{1}{8}{}}

\noindent{}We will now show that if $G_n$ is left-orderable, then \Case{1}{8}{} ($d^{-1}a>1$, $d^{-1}b>1$, $c^{-1}b>1$)is impossible.

\begin{lemma} In \Case{1}{8}{}, $c^{-1}d > 1$.
\label{lemma:inEq:Cd}
\end{lemma}

\begin{proof} Starting from the first group relation, we have:
\begin{align}
(a^{-1}b)^{10n}d^{-1}a^{2}&=1\nonumber{}\\
\Rightarrow{}a(a^{-1}b)^{10n}d^{-1}a&=1\nonumber{}\\
\Rightarrow{}(ba^{-1})^{10n-1}bd^{-1}a&=1\label{eq3:4}\\
\Rightarrow{}(ba^{-1})^{10n-2}ba^{-1}bd^{-1}a&=1\nonumber{}\\
\Rightarrow{}(cc^{-1})(ba^{-1}(dd^{-1}))^{10n-2}(cc^{-1})ba^{-1}(dd^{-1})bd^{-1}a&=1\nonumber{}\\
\Rightarrow{}c([c^{-1}ba^{-1}d][d^{-1}c])^{10n-2}(c^{-1}ba^{-1}d)(d^{-1}b)(d^{-1}a)&=1,\label{lemma:inEq:Cd:contradiction}
\end{align}
but $d^{-1}b>1$, $d^{-1}a>1$, and $c>1$ in \Case{1}{8}{}. Further, we know by Lemma~\ref{lemma:inEq:CbAd} that $c^{-1}ba^{-1}d>1$. Therefore, (\ref{lemma:inEq:Cd:contradiction}) shows that:
\begin{align*}
d^{-1}c&<1\\
\Rightarrow{}c^{-1}d&>1.\qedhere
\end{align*}
\end{proof}

\begin{lemma} In \Case{1}{8}{}, $ab^{-1} > 1$.
\label{lemma:inEq:aB}
\end{lemma}

\begin{proof} By (\ref{eq3:4}):
\begin{align}
(ba^{-1})^{10n-1}b(a^{-1}a)(d^{-1}a)&=1\nonumber{}\\
(ba^{-1})^{10n}a(d^{-1}a)&=1,\label{lemma:inEq:aB:contradiction}
\end{align}
but $d^{-1}a>1$ in \Case{1}{8}{} and $a>1$ in Case 1, so (\ref{lemma:inEq:aB:contradiction}) shows that:
\begin{align*}
ba^{-1}<1\\
\Rightarrow{}ab^{-1}>1.\qedhere
\end{align*}
\end{proof}

\begin{lemma} In \Case{1}{8}{}, $a^{-1}dc^{-1}a > 1$. \label{lemma:inEq:AdCa}
\end{lemma}
\begin{proof} By Corollary~\ref{corollary:eq6}, we have:
\begin{align}
a^{2}b^{-2}&=dc^{-1}\nonumber{}\\
\Rightarrow{}b^{-2}cd^{-1}a^{2} &= 1\nonumber{}\\
\Rightarrow{}b^{-1}cd^{-1}a^{2}b^{-1} &= 1\nonumber{}\\
\Rightarrow{}(b^{-1}a)(a^{-1}cd^{-1}a)(ab^{-1}) &= 1.\label{lemma:inEq:AdCa:contradiction}
\end{align}
But $b^{-1}a>1$ by assumption and $ab^{-1}>1$ in \Case{1}{8}{} by Lemma~\ref{lemma:inEq:aB} so (\ref{lemma:inEq:AdCa:contradiction}) shows that:
\begin{align*}
a^{-1}cd^{-1}a&<1\\
\Rightarrow{}a^{-1}dc^{-1}a&>1.\qedhere
\end{align*}
\end{proof}

\begin{lemma} In \Case{1}{8}{}, $c^{-1}d^{-2}a>1$.
\label{lemma:inEq:CDDa}
\end{lemma}
\begin{proof} Starting from the fourth group relation:
\begin{align}
d^{2}a^{-1}d(c^{-1}d)^{10n+3} &= 1\nonumber{}\\
\Rightarrow{}a^{-1}d(c^{-1}d)^{10n+3}d^{2} &= 1\nonumber{}\\
\Rightarrow{}a^{-1}(dc^{-1})^{10n+3}d^{3}&=1\nonumber{}\\
\Rightarrow{}a^{-1}(dc^{-1})^{10n+3}(aa^{-1})d^{2}(cc^{-1})d&=1\nonumber{}\\
\Rightarrow{}(a^{-1}dc^{-1}a)^{10n+3}(a^{-1}d^{2}c)(c^{-1}d)&=1.\label{lemma:inEq:CDDa:contradiction}
\end{align}
But $a^{-1}dc^{-1}a>1$ in \Case{1}{8}{} by Lemma~\ref{lemma:inEq:AdCa} and $c^{-1}d>1$ in \Case{1}{8}{} by Lemma~\ref{lemma:inEq:Cd}, so (\ref{lemma:inEq:CDDa:contradiction}) shows that:
\begin{align*}
a^{-1}d^{2}c&<1\\
\Rightarrow{}c^{-1}d^{-2}a&>1.\qedhere
\end{align*}
\end{proof}

\begin{lemma} In \Case{1}{8}{}, $d^{-1}ad^{-1}>1$.
\label{lemma:inEq:DaD}
\end{lemma}
\begin{proof}
Starting from the fourth group relation, we have:
\begin{align}
(c^{-1}d)^{10n+3}d^{2}a^{-1}d&=1\nonumber{}\\
\Rightarrow{}(c^{-1}d)^{10n+3}(cc^{-1})d^{2}a^{-1}d&=1\nonumber{}\\
\Rightarrow{}(c^{-1}d)^{10n+3}(c)(c^{-1}d)(da^{-1}d)&=1.\label{lemma:inEq:DaD:contradiction}
\end{align}
We know that $c^{-1}d>1$ in \Case{1}{8}{} by Lemma~\ref{lemma:inEq:Cd} and we know that $c>1$ in \Case{1}{8}{}, so (\ref{lemma:inEq:DaD:contradiction}) shows that:
\begin{align*}
da^{-1}d&<1\\
\Rightarrow{}d^{-1}ad^{-1}&>1.\qedhere
\end{align*}
\end{proof}

\begin{lemma} In \Case{1}{8}{}, $cba^{-1}>1$.
\label{lemma:inEq:cbA}
\end{lemma}

\begin{proof} By Lemma~\ref{lemma:eq8}, we have:
\begin{align}
c^{2}b&=d^{2}a\nonumber{}\\
\Rightarrow{}c^{-1}d^{2}ab^{-1}c^{-1}&= 1\nonumber{}\\
\Rightarrow{}(c^{-1}d)d(ab^{-1}c^{-1})&=1.\label{lemma:inEq:cbA:contradiction}
\end{align}
We know that $c^{-1}d>1$ in \Case{1}{8}{} by Lemma~\ref{lemma:inEq:Cd} and we know that $d>1$ in \Case{1}{8}{}, so (\ref{lemma:inEq:cbA:contradiction}) shows that:
\begin{align*}
ab^{-1}c^{-1}&<1\\
\Rightarrow{}cba^{-1}&>1.\qedhere
\end{align*}
\end{proof}

\begin{lemma} In \Case{1}{8}{}, $da^{-1}bc^{-1}>1$.
\label{lemma:inEq:dAbC}
\end{lemma}
\begin{proof} By Lemma~\ref{lemma:eq16}, we have:
\begin{align*}
b^{-1}cb^{-1}ad^{-1}a &= 1\\
\Rightarrow{}(cb^{-1}ad^{-1})(ab^{-1}) &=1\\
\Rightarrow{}ab^{-1}&=da^{-1}bc^{-1}\\
\Rightarrow{}da^{-1}bc^{-1}&>1,
\end{align*}
where the last implication follows from Lemma~\ref{lemma:inEq:aB}.
\end{proof}

\begin{lemma} In \Case{1}{8}{}, $cb^{-1}ac^{-1}>1$.
\label{lemma:inEq:cBaC}
\end{lemma}
\begin{proof} By (\ref{eq3:4}), we have:
\begin{align}
(ba^{-1})^{10n}bd^{-1}a&=1\nonumber{}\\
\Rightarrow{}b(a^{-1}b)^{10n}d^{-1}a&=1\nonumber{}\\
\Rightarrow{}b(a^{-1}b)(a^{-1}b)^{10n-1}d^{-1}a&=1\nonumber{}\\
\Rightarrow{}(ba^{-1})(bc^{-1})(ca^{-1}bc^{-1})^{10n-1}cd^{-1}a&=1\nonumber{}\\
\Rightarrow{}(cc^{-1})(ba^{-1})(dd^{-1})(ad^{-1}da^{-1})(bc^{-1})(ca^{-1}bc^{-1})^{10n-1}cd^{-1}a&=1\nonumber{}\\
\Rightarrow{}c(c^{-1}ba^{-1}d)(d^{-1}ad^{-1})(da^{-1}bc^{-1})(ca^{-1}bc^{-1})^{10n-1}c(d^{-1}a)&=1.\label{lemma:inEq:cBaC:contradiction}
\end{align}
We know $c^{-1}ba^{-1}d>1$ in \Case{1}{8}{} by Lemma~\ref{lemma:inEq:CbAd}; we know $d^{-1}ad^{-1}>1$ in \Case{1}{8}{} by Lemma~\ref{lemma:inEq:DaD}; and we know $da^{-1}bc^{-1}>1$ by Lemma~\ref{lemma:inEq:dAbC}. Furthermore, we know $d^{-1}a>1$ and $c>1$ by assumption in \Case{1}{8}{}. Therefore, (\ref{lemma:inEq:cBaC:contradiction}) shows that:
\begin{align*}
ca^{-1}bc^{-1}&<1\\
\Rightarrow{}cb^{-1}ac^{-1}&>1.\qedhere
\end{align*}
\end{proof}

\begin{lemma} In \Case{1}{8}{}, $cba^{-1}c^{-1}>1$.
\label{lemma:inEq:cbAC}
\end{lemma}
\begin{proof} By Lemma~\ref{lemma:eq8}, we have:
\begin{align}
c^{2}b&=d^{2}a\nonumber{}\\
\Rightarrow{}c^{-1}d^{2}ab^{-1}c^{-1}&=1\nonumber{}\\
\Rightarrow{}c^{-1}d^{2}a^{-1}a^{2}b^{-1}c^{-1}&=1\nonumber{}\\
\Rightarrow{}c^{-1}d^{2}a^{-1}(bc^{-1}cb^{-1})a(c^{-1}c)ab^{-1}c^{-1}&=1\nonumber{}\\
\Rightarrow{}(c^{-1}d)(da^{-1}bc^{-1})(cb^{-1}ac^{-1})(cab^{-1}c^{-1})&=1.\label{lemma:inEq:cbAC:contradiction}
\end{align}
We know that $c^{-1}d>1$ in \Case{1}{8}{} by Lemma~\ref{lemma:inEq:Cd}; we know that $da^{-1}bc^{-1}>1$ in \Case{1}{8}{} by Lemma~\ref{lemma:inEq:dAbC}; and we know that $cb^{-1}ac^{-1}>1$ in \Case{1}{8}{} by Lemma~\ref{lemma:inEq:cBaC}. Therefore, (\ref{lemma:inEq:cbAC:contradiction}) shows that:
\begin{align*}
cab^{-1}c^{-1}&<1\\
\Rightarrow{}cba^{-1}c^{-1}&>1.\qedhere
\end{align*}
\end{proof}

\begin{proposition}
If $G_{n}$ is left-orderable then \Case{1}{8}{} ($d^{-1}a > 1$, $d^{-1}b>1$, and $c^{-1}b>1$) is not possible.
\label{proposition:case1.viii}
\end{proposition}
\begin{proof} By (\ref{eq3:4}), we have:
\begin{align}
(ba^{-1})^{10n}bd^{-1}a&=1\nonumber{}\\
\Rightarrow{}(ba^{-1})(bd^{-1}a)(ba^{-1})^{10n-2}(ba^{-1})&=1\nonumber{}\\
\Rightarrow{}(ba^{-1})(bd^{-1}a)(c^{-1}c)(ba^{-1})^{10n-2}(c^{-1}c)(ba^{-1})&=1\nonumber{}\\
\Rightarrow{}(ba^{-1})(bd^{-1})(ac^{-1})(cba^{-1}c^{-1})^{10n-2}(c)(ba^{-1})&=1\nonumber{}\\
\Rightarrow{}(cc^{-1})(ba^{-1})(dd^{-1})(bd^{-1})(ad^{-1}da^{-1})(bc^{-1}cb^{-1})\nonumber{}\\
(ac^{-1})(cba^{-1}c^{-1})^{10n-1}(cba^{-1})&=1\nonumber{}\\
\Rightarrow{}c(c^{-1}ba^{-1}d)(d^{-1}b)(d^{-1}ad^{-1})(da^{-1}bc^{-1})\nonumber{}\\
(cb^{-1}ac^{-1})(cba^{-1}c^{-1})^{10n-1}(cba^{-1})&=1.\label{proposition:case1.viii:contradiction}
\end{align}
We know that $c^{-1}ba^{-1}d>1$ by Lemma~\ref{lemma:inEq:CbAd}; we know that $d^{-1}ad^{-1}>1$ in \Case{1}{8}{} by Lemma~\ref{lemma:inEq:DaD}; we know that $da^{-1}bc^{-1}>1$ in \Case{1}{8}{} by Lemma~\ref{lemma:inEq:dAbC}; we know that $cb^{-1}ac^{-1}>1$ in \Case{1}{8}{} by Lemma~\ref{lemma:inEq:cBaC}; we know that $cba^{-1}c^{-1}>1$ in \Case{1}{8}{} by Lemma~\ref{lemma:inEq:cbAC}; we know that $cba^{-1}>1$ in \Case{1}{8}{} by Lemma~\ref{lemma:inEq:cbA}; and we know that $c>1$ in \Case{1}{8}{} by assumption. Therefore (\ref{proposition:case1.viii:contradiction}) shows that if $G_n$ is left-orderable, then \Case{1}{8}{} is not possible.
\end{proof}

    %end{case1.viii}
%end{case1}

%begin{case16}

\noindent{}With Proposition~\ref{proposition:case1.viii}, we have eliminated the one remaining sub-case of Case 1. Thus, we have shown that if $G_n$ is left-orderable, then the only option for the signs of the four generators is Case 16. That is, if $G_n$ is left-orderable then $a<1$, $b<1$, $c<1$, and $d<1$.

\section{Case 16}
\label{section:case16}

\noindent{}We will now show that if $G_n$ is left-orderable, then Case 16 (see Table~\ref{table:case16}) is not possible.

\begin{table}[ht]
\begin{center}
\begin{tabular}{l | l | l | l | l}
Case\hspace{10 pt} & $a$\hspace{10 pt} & $b$\hspace{10 pt} & $c$\hspace{10 pt} & $d$\hspace{10 pt} \\\hline\hline
16 & $-$ & $-$ & $-$ & $-$
\end{tabular}
\end{center}
\caption{The signs of the 4 generators in Case 16.}
\label{table:case16}
\end{table}

\noindent{}We start by proving the signs of a few key elements.

\subsection{Inequalities for Case 16}

\begin{lemma} In Case 16, $d^{-1}a>1$.
\label{lemma:case16:Da}
\end{lemma}
\begin{proof}
By (\ref{eq1:2}):
\begin{align*}
d^{-1}a^{2}=(b^{-1}a)^{10n}>1,
\end{align*}
and in Case 16, $a^{-1}>1$, so:
\begin{align*}
(d^{-1}a^{2})(a^{-1})&>1\\
\Rightarrow{}d^{-1}a&>1.\qedhere
\end{align*}
\end{proof}

\begin{lemma} In Case 16, $c^{-1}b>1$.
\label{lemma:case16:Cb}
\end{lemma}
\begin{proof}
By (\ref{eq2:2}):
\begin{align}
c^{-1}b^{2}=(b^{-1}a)^{10n}>1,\label{inEq2}
\end{align}
and in Case 16, $b^{-1}>1$, so:
\begin{align*}
(c^{-1}b^{2})(b^{-1})&>1\\
\Rightarrow{}c^{-1}b&>1.\qedhere
\end{align*}
\end{proof}

\begin{lemma} In Case 16, $c^{-1}d>1$.
\label{lemma:case16:Cd}
\end{lemma}
\begin{proof} Suppose (for contradiction) that $c^{-1}d<1$, or equivalently that $d^{-1}c>1$. By the third group relation, we have:
\begin{align}
(d^{-1}c)^{10n+3}c^{-1}bc^{-2}&=1\nonumber{}\\
\Rightarrow{}(d^{-1}c)^{10n+3}&=c^{2}b^{-1}c.\label{lemma:case16:Cd:contradiction}
\end{align}
Since we are assuming $d^{-1}c>1$, (\ref{lemma:case16:Cd:contradiction}) shows that $c^{2}b^{-1}c>1$. By (\ref{inEq2}), $c^{-1}b^{2}>1$, thus:
\begin{align*}
(c^{2}b^{-1}c)(c^{-1}b^{2})>1\\
\Rightarrow{}c^{2}b>1.
\end{align*}
This is a contradiction, since both $b$ and $c$ and negative in Case 16. Therefore, $c^{-1}d\geq{}1$. However, by Proposition~\ref{proposition:id:Dc}, $c\neq{}d$ and thus $c^{-1}d\neq{}1$. Therefore, $c^{-1}d>1$.
\end{proof}

\begin{corollary} In Case 16, $c^{-1}a>1$.
\label{lemma:case16:Ca}
\end{corollary}
\begin{proof}
This follows from Lemma~\ref{lemma:case16:Da} and Lemma~\ref{lemma:case16:Cd} since:
\begin{align*}
c^{-1}a&=(c^{-1}d)(d^{-1}a).\qedhere
\end{align*}
\end{proof}

\begin{lemma} In Case 16, $ba^{-1}>1$.
\label{lemma:case16:bA}
\end{lemma}
\begin{proof}
Starting from the fourth group relation, we have:
\begin{align}
d^{2}a^{-1}d(c^{-1}d)^{10n+3}&=1\nonumber{}\\
\Rightarrow{}d^{2}a^{-1}d^{2}(d^{-1}(cc^{-1})c^{-1}d^{2})^{10n+3}d^{-1}&=1\nonumber{}\\
\Rightarrow{}da^{-1}d^{2}([d^{-1}c][c^{-2}d^{2}])^{10n+3}&=1\nonumber{}\\
\Rightarrow{}da^{-1}d^{2}([d^{-1}c][ba^{-1}])^{10n+3}&=1,\label{lemma:case16:bA:contradiction}
\end{align}
where the last implication follows from Lemma \ref{lemma:eq8}. Now $a^{-1}d<1$ by Lemma~\ref{lemma:case16:Da}, so we know $da^{-1}d^{2}<1$. Therefore, (\ref{lemma:case16:bA:contradiction}) tells us that:
\begin{align*}
([d^{-1}c][ba^{-1}])^{10n+3}>1,
\end{align*}
but $d^{-1}c<1$ by Lemma~\ref{lemma:case16:Cd}, so we must have:
\begin{align*}
ba^{-1}&>1.\qedhere
\end{align*}
\end{proof}



\subsection{Concordance of signs of a few useful elements}

Next we will show that in Case 16, $bc^{-1}$, $dc^{-1}$, $ad^{-1}$, $ac^{-1}$, $ad^{-3}$, and $bc^{-3}$ must all have the same sign. We begin by showing that all of these elements are non-trivial.

\begin{proposition}
In Case 16, $bc^{-1}\neq1$, $dc^{-1}\neq1$, $ad^{-1}\neq1$, $ac^{-1}\neq1$, $ac^{-1}\neq1$, and $bc^{-3}\neq1$.
\end{proposition}
\begin{proof}
Consequence of Lemma~\ref{lemma:case16:Cb}, Lemma~\ref{lemma:case16:Cd},  Lemma~\ref{lemma:case16:Da}, and Lemma~\ref{lemma:case16:Ca}.
\end{proof}

\noindent{}Now that we know that each of these elements can only be positive or negative, we proceed to show that they all have the same sign.


\begin{lemma} In Case 16, $ad^{-1}<1$ implies $dc^{-1}<1$.
\label{lemma:iffs:1}
\end{lemma}
\begin{proof}
Starting from the second group relation, we have:
\begin{align}
b^{-2}c(b^{-1}a)^{10n}&=1\nonumber{}\\
\Rightarrow{}b^{-1}c(b^{-1}a)^{10n}b^{-1}&=1\nonumber{}\\
\Rightarrow{}b^{-1}cb^{-1}(ab^{-1})^{10n}&=1\nonumber{}\\
\Rightarrow{}b^{-1}cb^{-1}a^{-1}(a^{2}b^{-1}a^{-1})^{10n}a&=1\nonumber{}\\
\Rightarrow{}ab^{-1}cb^{-1}a^{-1}((a^{2}b^{-2})(ba^{-1}))^{10n}&=1\nonumber{}\\
\Rightarrow{}a(b^{-1}cb^{-1})a^{-1}([dc^{-1}][ba^{-1}])^{10n}&=1\nonumber{}\\
\Rightarrow{}da^{-2}([dc^{-1}][ba^{-1}])^{10n}&=1.\label{lemma:iffs:1:contradiction}
\end{align}
Where the second to last implication follows from Corollary~\ref{corollary:eq6} and last implication follows from Lemma~\ref{lemma:eq16}. Now if $ad^{-1}<1$ we have:
\begin{align*}
da^{-1}&>1\\
\Rightarrow{}da^{-2}&>1.
\end{align*}
Thus if $ad^{-1}<1$, (\ref{lemma:iffs:1:contradiction}) implies:
\begin{align*}
(dc^{-1})(ba^{-1})<1,
\end{align*}
which shows that $dc^{-1}<1$, since $ba^{-1}>1$ by Lemma~\ref{lemma:case16:bA}.
\end{proof}

\begin{lemma} In Case 16, $dc^{-1}>1$ implies $ac^{-1}>1$.
\label{lemma:iffs:2.1}
\end{lemma}
\begin{proof}
By Lemma~\ref{lemma:iffs:1}:
\begin{align*}
dc^{-1}>1 \Rightarrow{} ad^{-1}>1.
\end{align*}
This completes the proof since:
\begin{align*}
ac^{-1}&=(ad^{-1})(dc^{-1}).\qedhere
\end{align*}
\end{proof}

\begin{lemma} In Case 16, $ad^{-1}<1$ implies $ac^{-1}<1$.
\label{lemma:iffs:2.2}
\end{lemma}
\begin{proof}
By Lemma~\ref{lemma:iffs:1}:
\begin{align*}
ad^{-1}<1 \Rightarrow{} dc^{-1}<1.
\end{align*}
This completes the proof since:
\begin{align*}
ac^{-1}&=(ad^{-1})(dc^{-1}).\qedhere
\end{align*}
\end{proof}

\begin{corollary} In Case 16, $dc^{-1}>1$ if and only if $ad^{-1}>1$ if and only if $ac^{-1}>1$.
\label{lemma:iffs:3}
\end{corollary}
\begin{proof}
Starting from the fourth group relation, we have:
\begin{align}
d^{2}a^{-1}d(c^{-1}d)^{10n+3}&=1\nonumber{}\\
\Rightarrow{}a^{-1}d(c^{-1}d)^{10n+3}d^{2}&=1\nonumber{}\\
\Rightarrow{}a^{-1}(dc^{-1})^{10n+3}d^{3}&=1\nonumber{}\\
\Rightarrow{}d^{2}(da^{-1})(dc^{-1})^{10n+3}&=1.\label{lemma:iffs:3:contradiction}
\end{align}
Now $d<1$ in Case 16, so (\ref{lemma:iffs:3:contradiction}) shows that:
\begin{align*}
ad^{-1}>1\Rightarrow{}da^{-1}<1\Rightarrow{}dc^{-1}>1.
\end{align*}
In conjunction with Lemma~\ref{lemma:iffs:1}, this shows that $dc^{-1}>1$ if and only if $ad^{-1}>1$. Now by Lemma~\ref{lemma:iffs:2.1} we have:
\begin{align*}
ad^{-1}>1\Rightarrow{}dc^{-1}>1\Rightarrow{}ac^{-1}>1,
\end{align*}
and by Lemma~\ref{lemma:iffs:2.2} we have:
\begin{align*}
ad^{-1}<1\Rightarrow{}ac^{-1}<1.
\end{align*}
Thus, $ad^{-1}>1$ if and only if $ac^{-1}>1$.
\end{proof}

\begin{lemma} In Case 16, $dc^{-1}>1$ if and only if $bc^{-1}>1$.
\label{lemma:iffs:4}
\end{lemma}
\begin{proof}[Proof of reverse direction]
Starting from the third group relation, we have:
\begin{align*}
(d^{-1}c)^{10n+3}c^{-1}bc^{-2}&=1\\
\Rightarrow{}c^{-1}(cd^{-1})^{10n+3}bc^{-2}&=1\\
\Rightarrow{}(cd^{-1})^{10n+3}(bc^{-2})&=c<1,
\end{align*}
and so:
\begin{align*}
dc^{-1}<1\Rightarrow{}cd^{-1}>1\Rightarrow{}bc^{-2}&<1\Rightarrow{}bc^{-1}<1. \qedhere
\end{align*}
\end{proof}

\begin{proof}[Proof of forward direction]
First note that:
\begin{align*}
bc^{-1}=(ba^{-1})(ac^{-1}).
\end{align*}
But $ba^{-1}>1$ by Lemma~\ref{lemma:case16:bA} so this shows that:
\begin{align*}
dc^{-1}>1\Rightarrow{}ac^{-1}>1\Rightarrow{}bc^{-1}>1,
\end{align*}
where the first implication follows from Corollary~\ref{lemma:iffs:3}.
\end{proof}

\begin{lemma} In Case 16, $dc^{-1}>1$ if and only if $bc^{-3}>1$ if and only if $ad^{-3}>1$.
\label{lemma:iffs:5}
\end{lemma}
\begin{proof}
By the third group relation, we have:
\begin{align*}
(d^{-1}c)^{10n+3}c^{-1}bc^{-2}&=1\\
\Rightarrow{}(cd^{-1})^{10n+3}bc^{-3}&=1\\
\Rightarrow{}bc^{-3}&=(dc^{-1})^{10n+3}.
\end{align*}
Thus $bc^{-3}>1$ must have the same sign as $dc^{-1}$. By the fourth group relation, we have:
\begin{align*}
(d^{-1}c)^{10n+3}d^{-1}ad^{-2}&=1\\
\Rightarrow{}(cd^{-1})^{10n+3}ad^{-3}&=1\\
\Rightarrow{}ad^{-3}&=(dc^{-1})^{10n+3}.
\end{align*}
Thus $ad^{-3}>1$ must have the same sign as $dc^{-1}$.
\end{proof}

\begin{proposition} In Case 16, the following elements all have the same sign: $dc^{-1}$, $ad^{-1}$, $ac^{-1}$, $bc^{-1}$, $bc^{-3}$, and $ad^{-3}$.
\label{proposition:iffs}
\end{proposition}
\begin{proof}
The proposition is evident by combining Corollary~\ref{lemma:iffs:3}, Lemma~\ref{lemma:iffs:4}, and Lemma~\ref{lemma:iffs:5}.
\end{proof}

\noindent{}In order to show Case 16 ($a,b,c,d<1$) is not possible if $G_n$ is left-orderable, we consider sub-cases (see Table~\ref{table:cases16b}). Because of Proposition~\ref{proposition:iffs} , it is easy to see that there are only two possible sub-cases of Case 16 considering the signs of $dc^{-1}$, $ad^{-1}$, $ac^{-1}$, $bc^{-1}$, $bc^{-3}$, and $ad^{-3}$

\begin{table}[ht]
\begin{center}
\begin{tabular}{l | l | l | l | l | l | l}
Case \hspace{10 pt} & $dc^{-1}$\hspace{10 pt} & $ad^{-1}$\hspace{10 pt} & $ac^{-1}$\hspace{10 pt} & $bc^{-1}$\hspace{10 pt} & $bc^{-3}$\hspace{10 pt} & $ad^{-3}$\hspace{10 pt}  \\\hline\hline
\case{16}{1}{} & $+$ & $+$ & $+$ & $+$ & $+$ & $+$ \\\hline
\case{16}{2}{} & $-$ & $-$ & $-$ & $-$ & $-$ & $-$
\end{tabular}
\end{center}
\caption{The two possible sub-cases of Case 16 considering the signs of $dc^{-1}$, $ad^{-1}$, $ac^{-1}$, $bc^{-1}$, $bc^{-3}$, and $ad^{-3}$}
\label{table:cases16b}
\end{table}

\subsection{\Case{16}{1}{}}

As a reminder, since we are working in a sub-case of Case 16, we know $a<1$, $b<1$, $c<1$, and $d<1$.

\begin{lemma} In \Case{16}{1}{}, $a^{-1}dc^{-1}a>1$.
\label{lemma:case16.A:AdCa}
\end{lemma}
\begin{proof} By the third group relation, we have:
\begin{align}
(d^{-1}c)^{10n+3}c^{-1}bc^{-2}&=1\nonumber{}\\
\Rightarrow{}c^{-1}(cd^{-1})^{10n+3}bc^{-2}&=1\nonumber{}\\
\Rightarrow{}c^{-1}(aa^{-1})(cd^{-1})^{10n+3}(aa^{-1})b(a^{-1}a)c^{-1}(a^{-1}a)c^{-1}&=1\nonumber{}\\
\Rightarrow{}(c^{-1}a)(a^{-1}cd^{-1}a)^{10n+3}(a^{-1})(ba^{-1})(ac^{-1})(a^{-1})(ac^{-1})&=1.\label{16.A:AdCa}\
\end{align}
But $c^{-1}a>1$ by Corollary \ref{lemma:case16:Ca} , $a^{-1}>1$ in Case 16, $ba^{-1}>1$ by Lemma \ref{lemma:case16:bA} and $ac^{-1}>1$ in \Case{16}{1}{} by assumption, so (\ref{16.A:AdCa}) shows that:
\begin{align*}
a^{-1}cd^{-1}a&<1\\
\Rightarrow{}a^{-1}dc^{-1}a&>1.\qedhere
\end{align*}
\end{proof}

\begin{lemma} In \Case{16}{1}{}, $ab^{-2}a>1$.
\label{lemma:case16.A:aBBa}
\end{lemma}
\begin{proof} By Lemma \ref{lemma:eq5}, we have:
\begin{align}
dc^{-1}b^{2}a^{-2}=1\nonumber{}\\
\Rightarrow{}a^{-1}dc^{-1}b^{2}a^{-1}=1\nonumber{}\\
\Rightarrow{}a^{-1}dc^{-1}(aa^{-1})b^{2}a^{-1}=1\nonumber{}\\
\Rightarrow{}(a^{-1}dc^{-1}a)(a^{-1}b^{2}a^{-1})=1.\label{16.A:aBBa}
\end{align}
But $a^{-1}dc^{-1}a>1$ by Lemma \ref{lemma:case16.A:AdCa}, so (\ref{16.A:aBBa}) shows that:
\begin{align*}
a^{-1}b^{2}a^{-1}&<1\\
\Rightarrow{}ab^{-2}a&>1.\qedhere
\end{align*}
\end{proof}

\begin{lemma} In \Case{16}{1}{}, $d^{-2}cb>1$.
\label{lemma:case16.A:DDcb}
\end{lemma}
\begin{proof} By Lemma \ref{lemma:eq8} and Lemma \ref{lemma:eq16}:
\begin{align}
d^{-2}cb=&d^{-2}c(cb^{-1}bc^{-1})b\nonumber{}\\
\Rightarrow{}d^{-2}cb=&(d^{-2}c^{2})b^{-1}(bc^{-1}b)\nonumber{}\\
\Rightarrow{}d^{-2}cb=&(ab^{-1})b^{-1}(ad^{-1}a)\nonumber{}\\
\Rightarrow{}d^{-2}cb=&(ab^{-2}a)(d^{-1}a).
\label{16.A:DDcb}
\end{align}
But $ab^{-2}a>1$ by Lemma \ref{lemma:case16.A:aBBa}, and $d^{-1}a>1$ by Lemma \ref{lemma:case16:Da}, so (\ref{16.A:DDcb}) shows that:
\begin{align*}
(ab^{-2}a)(d^{-1}a)&>1\\
\Rightarrow{}d^{-2}cb&>1.\qedhere
\end{align*}
\end{proof}

\begin{lemma} In \Case{16}{1}{}, $a^{-1}d^{-2}c^{2}>1$.
\label{lemma:case16.A:ADDcc}
\end{lemma}
\begin{proof} By Lemma \ref{lemma:eq8}:
\begin{align}
b^{-1}c^{-2}d^{2}a=1\nonumber{}\\
\Rightarrow{}b^{-1}(aa^{-1})c^{-2}d^{2}a=1\nonumber{}\\
\Rightarrow{}(b^{-1}a)(a^{-1})(c^{-2}d^{2}a)=1.
\label{16.A:ADDcc}
\end{align}
But $b^{-1}a>1$ in general, and $a^{-1}>1$ in case 16, so (\ref{16.A:ADDcc}) shows that:
\begin{align*}
c^{-2}d^{2}a&<1\\
\Rightarrow{}a^{-1}d^{-2}c^{2}&>1.\qedhere
\end{align*}
\end{proof}

\begin{corollary} In \Case{16}{1}{} $a^{-1}d^{-2}c>1$.
\label{corollary:case16.A:ADDc}
\end{corollary}
\begin{proof} We know $c^{-1}>1$ in Case 16, and by Lemma~\ref{lemma:case16.A:ADDcc}, we know $a^{-1}d^{-2}c^{2}>1$ in \Case{16}{1}{}. Therefore,
\begin{align*}
 (a^{-1}d^{-2}c^{2})(c^{-1})&>1\\
 \Rightarrow{}a^{-1}d^{-2}c&>1.\qedhere
 \end{align*}
 \end{proof}
 
\begin{lemma} In \Case{16}{1}{}, $a^{-1}c^{-1}da > 1$.
\label{lemma:case16:A:ACda}
\end{lemma}
\begin{proof} By the third group relation, we have:
\begin{align}
(d^{-1}c)^{10n+3}c^{-1}bc^{-2} =1\nonumber{}\\
\Rightarrow{}(d^{-1}c)^{10n+2}d^{-1}bc^{-2} =1\nonumber{}\\
\Rightarrow{}d^{-1}bc^{-2}(d^{-1}c)^{10n+2}=1\nonumber{}\\
\Rightarrow{}(a^{-1}a)d^{-1}b(a^{-1}ad^{-1}d)c^{-2}(aa^{-1})(d^{-1}c)^{10n+2}(aa^{-1})=1\nonumber{}\\
\Rightarrow{}(a^{-1})(ad^{-1})(ba^{-1})(ad^{-1})(dc^{-1})(c^{-1}a)(a^{-1}d^{-1}ca)^{10n+2}(a^{-1})=1.
\label{16.A:ACda}
\end{align} 
But $a^{-1}>1$ in Case 16, $ad^{-1}>1$ in \Case{16}{1}{}, $ba^{-1}>1$ by Lemma \ref{lemma:case16:bA}, $dc^{-1}>1$ in \Case{16}{1}{}, and $c^{-1}a>1$ by Lemma \ref{lemma:case16:Ca}, so (\ref{16.A:ACda}) shows that:
\begin{align*}
a^{-1}d^{-1}ca&<1\\
\Rightarrow{}a^{-1}c^{-1}da&>1.\qedhere
\end{align*}
\end{proof}

\begin{lemma} In \Case{16}{1}{}, $ac^{-1}da^{-1}>1$.
\label{lemma:case16.A:aCdA}
\end{lemma}
\begin{proof} By the third group relation, we have:
\begin{align}
(d^{-1}c)^{10n+3}c^{-1}bc^{-2}=1\nonumber{}\\
\Rightarrow{}(a^{-1}a)(d^{-1}c)^{10n+2}(a^{-1}a)(d^{-1}c)c^{-1}b(a^{-1}a)c^{-1}(a^{-1}a)c^{-1}=1\nonumber{}\\
\Rightarrow{}(a^{-1})(ad^{-1}ca^{-1})^{10n+2}(ad^{-1})(cc^{-1})(ba^{-1})(ac^{-1})(a^{-1})(ac^{-1})=1\nonumber{}\\
\Rightarrow{}(a^{-1})(ad^{-1}ca^{-1})^{10n+2}(ad^{-1})(ba^{-1})(ac^{-1})(a^{-1})(ac^{-1})=1.
\label{16.A:aCdA}
\end{align}
But $ba^{-1}>1$ by Lemma \ref{lemma:case16:bA}, $a^{-1}>1$ in Case 16, $ad^{-1}>1$ in \Case{16}{1}{}, , and $ac^{-1}>1$ in \Case{16}{1}{}, so (\ref{16.A:aCdA}) shows that:
\begin{align*}
ad^{-1}ca^{-1}&<1\\
\Rightarrow{}ac^{-1}da^{-1}&>1.\qedhere
\end{align*}
\end{proof}

\begin{lemma} In \Case{16}{1}{}, $b^{-1}cd^{-1}a>1$.
\label{lemma:case16.A:BcDa}
\end{lemma}
\begin{proof} By Lemma \ref{lemma:eq5}, we have:
\begin{align}
dc^{-1}b^{2}a^{-2}=1\nonumber{}\\
\Rightarrow{}a^{-1}dc^{-1}b^{2}a^{-1}=1\nonumber{}\\
\Rightarrow{}(a^{-1}dc^{-1}b)(ba^{-1})=1.
\label{16.A:BcDa}
\end{align}
But $ba^{-1}>1$ by Lemma \ref{lemma:case16:bA}, so (\ref{16.A:BcDa}) shows that:
\begin{align*}
a^{-1}dc^{-1}b&<1\\
\Rightarrow{}b^{-1}cd^{-1}a&>1.\qedhere
\end{align*}
\end{proof}

\begin{lemma} In \Case{16}{1}{}, $ad^{-2}c>1$.
\label{lemma:16:A:aDDc}
\end{lemma}
\begin{proof} By Lemma~\ref{lemma:eq7}, we have:
\begin{align}
d^{2}a^{-1}d&=c^{2}b^{-1}c\nonumber{}\\
\Rightarrow{}bc^{-2}d^{2}a^{-1}dc^{-1}&=1\nonumber{}\\
\Rightarrow{}b(a^{-1}a)(d^{-1}d)c^{-2}d^{2}a^{-1}dc^{-1}&=1\nonumber{}\\
\Rightarrow{}(ba^{-1})(ad^{-1})(dc^{-1})(c^{-1}d^{2}a^{-1})(dc^{-1})&=1.\label{lemma:16:A:aDDc:contradiction}
\end{align}
But $ba^{-1}>1$ in Case 16 by Lemma~\ref{lemma:case16:bA}, $ad^{-1}>1$ in \Case{16}{1}{} by assumption, and $dc^{-1}>1$ in \Case{16}{1}{} by assumption. Therefore, (\ref{lemma:16:A:aDDc:contradiction}) shows that:
\begin{align*}
c^{-1}d^{2}a^{-1}&<1\\
\Rightarrow{}ad^{-2}c&>1.\qedhere
\end{align*}
\end{proof}

\begin{lemma} In \Case{16}{1}{}, $a^{-1}d^{-1}cd^{-1}c>1$.
\label{lemma:16:A:ADcDc}
\end{lemma}
\begin{proof} By Lemma~\ref{lemma:eq7}, we have:
\begin{align}
d^{2}a^{-1}d&=c^{2}b^{-1}c\nonumber{}\\
\Rightarrow{}bc^{-2}d^{2}a^{-1}dc^{-1}&=1\nonumber{}\\
\Rightarrow{}b(a^{-1}a)c^{-1}(da^{-1}ad^{-1})(d^{-1}cc^{-1}d)c^{-1}d(aa^{-1})da^{-1}dc^{-1}&=1\nonumber{}\\
\Rightarrow{}(ba^{-1})(ac^{-1}da^{-1})(ad^{-2}c)(c^{-1}dc^{-1}da)(a^{-1}da^{-1})(dc^{-1})&=1.\label{lemma:16:A:ADcDc:contradiction}
\end{align}
But $ba^{-1}>1$ in Case 16 by Lemma~\ref{lemma:case16:bA}, $ac^{-1}da^{-1}>1$ by Lemma~\ref{lemma:case16.A:aCdA}, $ad^{-2}c>1$ in \Case{16}{1}{} by Lemma~\ref{lemma:16:A:aDDc}, $a^{-1}da^{-1}>1$ in Case 16 since $a^{-1}da^{-1}=(ba^{-1})^{10n}$ (see (\ref{eq1:3})) which is positive by Lemma~\ref{lemma:case16:bA}, and $dc^{-1}>1$ in \Case{16}{1}{} by assumption. Therefore, (\ref{lemma:16:A:ADcDc:contradiction}) shows that:
\begin{align*}
c^{-1}dc^{-1}da&<1\\
\Rightarrow{}a^{-1}d^{-1}cd^{-1}c&>1.\qedhere
\end{align*}
\end{proof}

\begin{lemma} In \Case{16}{1}{}, $a^{-1}c^{-2}dca>1$.
\label{lemma:case16:A:ACCdca}
\end{lemma}
\begin{proof} Suppose that $G_n$ is left-orderable, and suppose (for contradiction), that $a<1$, $b<1$, $c<1$, and $d<1$. By the third group relation, we have: 
\begin{align}
(d^{-1}c)^{10n+3}c^{-1}bc^{-2}&=1\nonumber{}\\
\Rightarrow{}(d^{-1}c)^{10n+2}d^{-1}bc^{-2}&=1\nonumber{}\\
\Rightarrow{}d^{-1}bc^{-2}(d^{-1}c)^{10n+2}&=1\nonumber{}\\
\Rightarrow{}(d^{-1}bc^{-2})(c^{-1}aa^{-1}c^{-1})(d^{-1}c)^{10n-2}(caa^{-1}c^{-1})&\nonumber{}\\
(daa^{-1}d^{-1})(d^{-1}c)(cc^{-1}dd^{-1})(d^{-1}c)(bb^{-1}aa^{-1})(d^{-1}cd^{-1}c)&=1\nonumber{}\\
\Rightarrow{}(d^{-1})(bc^{-2})(c^{-1}a)(a^{-1}c^{-1}d^{-1}cca)^{10n-2}(a^{-1}c^{-1}da)&\nonumber{}\\
(a^{-1}d^{-1}d^{-1}cc)(c^{-1}d)(d^{-1}d^{-1}cb)(b^{-1}a)(a^{-1}d^{-1}cd^{-1}c)&=1\nonumber{}\\
\Rightarrow{}(d^{-1})(bc^{-2})(c^{-1}a)(a^{-1}c^{-1}d^{-1}c^{2}a)^{10n-2}(a^{-1}c^{-1}da)&\nonumber{}\\
(a^{-1}d^{-2}c^{2})(c^{-1}d)(d^{-2}cb)(b^{-1}a)(a^{-1}d^{-1}cd^{-1}c)&=1.
\label{16.A:ACCdca}
\end{align}
But $d^{-1}>1$ by Case 16, $bc^{-2}>1$ by \Case{16}{1}{}, $c^{-1}a>1$ by Lemma \ref{lemma:case16:Ca}, $a^{-1}c^{-1}da>1$ by Lemma \ref{lemma:case16:A:ACda}, $a^{-1}d^{-2}c^{2}>1$ by Lemma \ref{lemma:case16.A:ADDcc}, $c^{-1}d>1$ by Lemma \ref{lemma:case16:Cd}, $d^{-2}cb>1$ by Lemma \ref{lemma:case16.A:DDcb}, $b^{-1}a>1$ by general assumption, and $a^{-1}d^{-1}cd^{-1}c>1$ by Lemma \ref{lemma:16:A:ADcDc}, so (\ref{16.A:ACCdca}) shows that:
\begin{align*}
a^{-1}c^{-1}d^{-1}c^{2}a&<1\\
\Rightarrow{}a^{-1}c^{-2}dca&>1.\qedhere
\end{align*}
\end{proof}

\begin{proposition}  If $G_n$ is left-orderable, then \Case{16}{1}{} ($a<1$, $b<1$, $c<1$, and $d<1$) is impossible.
\label{proposition:case16.A}
\end{proposition}
\begin{proof} By Lemma \ref{lemma:eq8}, we have:
\begin{align}
b^{-1}c^{-2}d^{2}a&=1\nonumber{}\\
\Rightarrow{}b^{-1}(aa^{-1})c^{-2}d(caa^{-1}c^{-1})da&=1\nonumber{}\\
\Rightarrow{}(b^{-1}a)(a^{-1}c^{-2}dca)(a^{-1}c^{-1}da)&=1.
\label{proposition:case16.A:contradiction}
\end{align}
But $b^{-1}a>1$ by general assumption, $a^{-1}c^{-2}dca>1$ by Lemma \ref{lemma:case16:A:ACCdca}, and $a^{-1}c^{-1}da>1$ by Lemma \ref{lemma:case16:A:ACda}. Therefore, (\ref{proposition:case16.A:contradiction}) states that a product of positives is the identity, a contradiction.
\end{proof}

\subsection{\Case{16}{2}{}}

\noindent{} As a reminder, since we are working in a sub-case of Case 16, we know $a<1$, $b<1$, $c<1$, and $d<1$.

\begin{lemma} In \Case{16}{2}{}, $a^{-1}bc^{-1}b>1$.
\label{lemma:case16:AbCb}
\end{lemma}
\begin{proof}By Lemma~\ref{lemma:eq16}, we have:
\begin{align*}
bc^{-1}b=ad^{-1}a.
\end{align*}
Thus:
\begin{align*}
a^{-1}(bc^{-1}b)=a^{-1}(ad^{-1}a)=(d^{-1}a).
\end{align*}
But $d^{-1}a>1$ in Case 16 by Lemma~\ref{lemma:case16:Da}; therefore, we know:
\begin{align*}
a^{-1}bc^{-1}b=d^{-1}a&>1.\qedhere
\end{align*}
\end{proof}

\begin{corollary} In \Case{16}{2}{}, $a^{-1}bc^{-1}a>1$ and $a^{-1}bc^{-1}>1$.
\label{corollary:case16:AbC}
\label{corollary:case16:AbCa}
\end{corollary}
\begin{proof}By Lemma \ref{lemma:case16:AbCb}, we have
\begin{align*}
a^{-1}bc^{-1}b>1.
\end{align*}
Since $b^{-1}a>1$, we have:
\begin{align*}
b^{-1}a&>1\\
\Rightarrow{}a^{-1}bc^{-1}b(b^{-1}a)>a^{-1}bc^{-1}b&>1\\
\Rightarrow{}a^{-1}bc^{-1}a>a^{-1}bc^{-1}b&>1.
\end{align*}
By $b^{-1}>1$, we have:
\begin{align*}
b^{-1}&>1\\
\Rightarrow{}a^{-1}bc^{-1}b(b^{-1})>a^{-1}bc^{-1}b&>1\\
\Rightarrow{}a^{-1}bc^{-1}>a^{-1}bc^{-1}b&>1.\qedhere
\end{align*}
\end{proof}

\begin{lemma} In \Case{16}{2}{}, $c^{-1}d^{-1}c > 1 $.
\label{lemma:case16:CDc}
\end{lemma}
\begin{proof} By Lemma~\ref{lemma:eq8}, we have:
\begin{align}
b^{-1}c^{-2}d^{2}a &= 1\nonumber{}\\
\Rightarrow{}b^{-1}(b^{-1}b)c^{-2}d(cc^{-1})d(cc^{-1})a &= 1\nonumber{}\\
\Rightarrow{}b^{-1}b^{-1}(aa^{-1})bc^{-2}d(cc^{-1})d(cc^{-1})a &= 1\nonumber{}\\
\Rightarrow{}(b^{-1})(b^{-1}a)(a^{-1}bc^{-1})(c^{-1}dc)(c^{-1}dc)(c^{-1}a) &=1.\label{lemma:case16:CDc:contradiction}
\end{align}
Now $b^{-1}>1$ and $b^{-1}a>1$ by genera assumption, $a^{-1}bc^{-1}>1$ in \Case{16}{2}{} by Corollary~\ref{corollary:case16:AbC}, and $c^{-1}a>1$ in Case 16 by Corollary~\ref{lemma:case16:Ca}. Therefore, (\ref{lemma:case16:CDc:contradiction}) shows that
\begin{align*}
c^{-1}dc&<1\\
\Rightarrow{}c^{-1}d^{-1}c&>1.\qedhere
\end{align*}
\end{proof}

\begin{lemma} In \Case{16}{2}{}, $b^{-1}d^{-1}b>1$.
\label{lemma:case16:BDb}
\end{lemma}
\begin{proof} By Lemma~\ref{lemma:eq8}, we have:
\begin{align}
d^{2}a&=c^{2}b\nonumber{}\\
\Rightarrow{}b^{-1}c^{-2}d^{2}a&=1\nonumber{}\\
\Rightarrow{}b^{-1}c^{-2}(bb^{-1})d(bb^{-1})d(bb^{-1})a&=1\nonumber{}\\
\Rightarrow{}(b^{-1})(c^{-1})(c^{-1}b)(b^{-1}db)(b^{-1}db)(b^{-1}a)&=1.\label{lemma:case16:BDb:contradiction}
\end{align}
Now $b^{-1}$, $c^{-1}$, and $b^{-1}a$ are positive by assumption in Case 16, and $c^{-1}b>1$ by Lemma~\ref{lemma:case16:Cb}. Therefore, (\ref{lemma:case16:BDb:contradiction}) shows that
\begin{align*}
b^{-1}db&<1\\
\Rightarrow{}b^{-1}d^{-1}b&>1.\qedhere
\end{align*}
\end{proof}

\begin{lemma} In \Case{16}{2}{}, $a^{-1}dc^{-1}a>1$.
\label{lemma:case16:AdCa}
\end{lemma}

\begin{proof} Starting from the third group relation, we have:
\begin{align}
(d^{-1}c)^{10n+3}c^{-1}bc^{-2}&=1\nonumber{}\\
\Rightarrow{}c^{-1}(cd^{-1})^{10n+3}bc^{-2}&=1\nonumber{}\\
\Rightarrow{}(c^{-1}a)(a^{-1}cd^{-1}a)^{10n+3}(a^{-1}bc^{-1})(c^{-1})&=1.\label{lemma:case16:AdCa:contradiction}
\end{align}
Now $c^{-1}>1$ in Case 16 by assumption, $c^{-1}a>1$ in Case 16 by Corollary~\ref{lemma:case16:Ca}, and $a^{-1}bc^{-1}>1$ in \Case{16}{2}{} by Corollary~\ref{corollary:case16:AbC}. Therefore, (\ref{lemma:case16:AdCa:contradiction}) shows that:
\begin{align*}
a^{-1}cd^{-1}a&<1\\
\Rightarrow{}a^{-1}dc^{-1}a&>1.\qedhere
\end{align*}
\end{proof}

\begin{lemma} In \Case{16}{2}{}, $d^{-2}cb>1$.
\label{lemma:case16:DDcb}
\end{lemma}

\begin{proof} By Lemma~\ref{lemma:eq8}, we have:
\begin{align}
c^{2}b&=d^{2}a\nonumber{}\\
\Rightarrow{}d^{-2}c^{2}&=ab^{-1}.\label{eq8:4}
\end{align}
By Corollary~\ref{corollary:eq6}, we have:
\begin{align}
a^{2}b^{-2}&=dc^{-1}\nonumber{}\\
\Rightarrow{}a^{2}b^{-2}cd^{-1}&=1.\label{eq6:4}
\end{align}
Combining (\ref{eq8:4}) and (\ref{eq6:4}), we find:
\begin{align}
ad^{-2}c^{2}b^{-1}cd^{-1}&=1\label{eq18}\\
\Rightarrow{}dc^{-1}bc^{-2}d^{2}a^{-1}&=1\nonumber{}\\
\Rightarrow{}a^{-1}dc^{-1}bc^{-2}d^{2}&=1\nonumber{}\\
\Rightarrow{}a^{-1}dc^{-1}(aa^{-1})bc^{-1}(bb^{-1})c^{-1}d^{2}&=1\nonumber{}\\
\Rightarrow{}(a^{-1}dc^{-1}a)(a^{-1}bc^{-1}b)(b^{-1}c^{-1}d^{2})&=1.\label{lemma:case16:DDcb:contradiction}
\end{align}
Now $a^{-1}dc^{-1}a>1$ in \Case{16}{2}{} by Lemma~\ref{lemma:case16:AdCa} and $a^{-1}bc^{-1}b>1$ in \Case{16}{2}{} by Lemma~\ref{lemma:case16:AbCb}. Therefore, (\ref{lemma:case16:DDcb:contradiction}) shows that
\begin{align*}
b^{-1}c^{-1}d^{2}&<1\\
\Rightarrow{}d^{-2}cb&>1.\qedhere
\end{align*}
\end{proof}

\begin{lemma} In \Case{16}{2}{}, $dc^{2}a^{-1}>1$.
\label{lemma:case16:dccA}
\end{lemma}
\begin{proof} By the third group relation, we have:
\begin{align*}
(d^{-1}c)^{10n+3}c^{-1}bc^{-2}=1\\
\Rightarrow{}d^{-1}c(d^{-1}c)^{10n+2}c^{-1}b(a^{-1}a)c^{-2}=1\\
\Rightarrow{}(cd^{-1})^{10n+2}(cc^{-1})(ba^{-1})(ac^{-2}d^{-1})=1\\
\Rightarrow{}(cd^{-1})^{10n+2}(ba^{-1})(ac^{-2}d^{-1})=1,
\end{align*}
where the last equality implies $ac^{-2}d^{-1}<1$, since $cd^{-1}>1$ in \Case{16}{2}{} by assumption, and $ba^{-1}>1$ by Lemma~\ref{lemma:case16:bA}. Therefore, we know $dc^{2}a^{-1}>1$.
\end{proof}

\begin{lemma} In \Case{16}{2}{}, $dab^{-1}a^{-1}>1$.
\label{lemma:case16:daBA}
\end{lemma}
\begin{proof} By Lemma~\ref{lemma:eq8}, we have:
\begin{align*}
c^{2}b&=d^{2}a\\
c^{2}ba^{-1}d^{-2}&=1\\
\Rightarrow{}d^{-1}c^{2}ba^{-1}d^{-1}&=1\\
\Rightarrow{}d^{-1}(d^{-1}d)c^{2}(a^{-1}a)ba^{-1}d^{-1}&=1\\
\Rightarrow{}(d^{-2})(dc^{2}a^{-1})(aba^{-1}d^{-1})&=1,
\end{align*}
where the last equality implies $aba^{-1}d^{-1}<1$, since $d^{-1}>1$ in Case 16 by assumption, and $dc^{2}a^{-1}>1$ in \Case{16}{2}{} by Lemma \ref{lemma:case16:dccA}. Therefore, we know $dab^{-1}a^{-1}>1$.
\end{proof}

\begin{lemma} In \Case{16}{2}{}, $b^{-1}cd^{-1}a>1$.
\label{lemma:case16:BcDa}
\end{lemma}
\begin{proof} By Lemma~\ref{lemma:eq5}, we have:
\begin{align*}
d^{-1}a^{2}&=c^{-1}b^{2}\\
\Rightarrow{}dc^{-1}b^{2}a^{-2}&=1\\
\Rightarrow{}a^{-1}dc^{-1}b^{2}a^{-1}&=1\\
\Rightarrow{}(a^{-1}dc^{-1}b)(ba^{-1})&=1,
\end{align*}
where the last equality implies $a^{-1}dc^{-1}b<1$, since $ba^{-1}>1$ in Case 16 by Lemma~\ref{lemma:case16:bA}. Therefore, we know $b^{-1}cd^{-1}a>1$.
\end{proof}

\begin{lemma} In \Case{16}{2}{}, $ab^{-1}d^{-1}a>1$.
\label{lemma:case16:aBDa}
\end{lemma}
\begin{proof} By Lemma~\ref{lemma:eq5}, we have:
\begin{align*}
d^{-1}a^{2}&=c^{-1}b^{2}\\
\Rightarrow{}dc^{-1}b^{2}a^{-2}&=1\\
\Rightarrow{}a^{-1}dc^{-1}(aa^{-1})b(c^{-1}bb^{-1}c)(d^{-1}aa^{-1}d)ba^{-1}&=1\\
\Rightarrow{}(a^{-1}dc^{-1}a)(a^{-1}bc^{-1}b)(b^{-1}cd^{-1}a)(a^{-1}dba^{-1})&=1,
\end{align*}
where the last equality implies $a^{-1}dba^{-1}<1$, since $a^{-1}dc^{-1}a>1$ in \Case{16}{2}{} by Lemma \ref{lemma:case16:AdCa}, $a^{-1}bc^{-1}b>1$ in \Case{16}{2}{} by Lemma \ref{lemma:case16:AbCb}, and $b^{-1}cd^{-1}a>1$ in \Case{16}{2}{} by Lemma \ref{lemma:case16:BcDa}. Therefore, we know $ab^{-1}d^{-1}a>1$.
\end{proof}

\begin{lemma} In \Case{16}{2}{}, $cda^{-1}>1$.
\label{lemma:case16:cdA}
\end{lemma}
\begin{proof} In \Case{16}{2}{}, $1>dc^{-1}$ by assumption, and so we have:
\begin{align*}
1>dc^{-1}&=(d^{-1}d)d(da^{-1}ad^{-1})c^{-1}\\
\Rightarrow{}1>dc^{-1}&=(d^{-1})(d^{3}a^{-1})(ad^{-1}c^{-1}),
\end{align*}
where the last equality implies $ad^{-1}c^{-1}<1$, since $d^{-1}>1$ and $d^{3}a^{-1}>1$ in \Case{16}{2}{} by assumption.
\end{proof}

\begin{lemma} In \Case{16}{2}{}, $a^{-1}dc^{-1}>1$.
\label{lemma:case16:AdC}
\end{lemma}
\begin{proof}
Starting from the third group relation, we have:
\begin{align}
(d^{-1}c)^{10n+3}c^{-1}bc^{-2}&=1\nonumber{}\\
\Rightarrow{}c^{-1}(cd^{-1})^{10n+3}bc^{-2}&=1\nonumber{}\\
\Rightarrow{}c^{-1}(cd^{-1})^{10n+2}(cd^{-1})(aa^{-1})(bc^{-1})c^{-1}&=1\nonumber{}\\
\Rightarrow{}c^{-1}(cd^{-1})^{10n+2}(cd^{-1}a)(a^{-1}bc^{-1})c^{-1}&=1.\label{lemma:case16:AdC:contradiction}
\end{align}
Now $c^{-1}>1$ in Case 16, $cd^{-1}>1$ in \Case{16}{2}{} by assumption, and $a^{-1}bc^{-1}>1$ in \Case{16}{2}{} by Corollary~\ref{corollary:case16:AbC}. Therefore, (\ref{lemma:case16:AdC:contradiction}) shows that:
\begin{align*}
cd^{-1}a&<1\\
\Rightarrow{}a^{-1}dc^{-1}&>1.\qedhere
\end{align*}
\end{proof}


\begin{lemma} In \Case{16}{2}{}, $ca^{-1}bc^{-1}>1$.
\label{lemma:case16:cAbC}
\end{lemma}
\begin{proof} By the first group relation, we have:
\begin{align*}
(b^{-1}a)^{10n}a^{-2}d&=1\\
\Rightarrow{}a^{-2}d(b^{-1}a)^{10n}&=1\\
\Rightarrow{}a^{-2}d(c^{-1}c)(b^{-1}a)^{10n-1}(c^{-1}c)(b^{-1}a)&=1\\
\Rightarrow{}(a^{-1}dc^{-1})(cb^{-1}ac^{-1})^{10n-1}(cb^{-1})&=1,
\end{align*}
where the last equality implies $cb^{-1}ac^{-1}<1$, since $a^{-1}dc^{-1}>1$ in \Case{16}{2}{} by Lemma \ref{lemma:case16:AdC}, and $cb^{-1}>1$ in \Case{16}{2}{} by assumption. Therefore, we know $ca^{-1}bc^{-1}>1$.
\end{proof}

\begin{lemma} In \Case{16}{2}{}, $b^{-1}c^{-1}db>1$.
\label{lemma:case16:BCdb}
\end{lemma}
\begin{proof} Starting from the third group relation, we have:
\begin{align}
(d^{-1}c)^{10n+3}c^{-1}bc^{-2}&=1\nonumber{}\\
\Rightarrow{}c^{-2}(d^{-1}c)^{10n+3}c^{-1}b&=1\nonumber{}\\
\Rightarrow{}c^{-2}(d^{-1}c)^{10n+2}d^{-1}b&=1\nonumber{}\\
\Rightarrow{}(c^{-1})(c^{-1}b)(b^{-1}d^{-1}cb)^{10n+2}(b^{-1}d^{-1}b)&=1.\label{lemma:case16:BCdb:contradiction}
\end{align}
Now $c^{-1}>1$ in Case 16 by assumption, $c^{-1}b>1$ in Case 16 by Lemma~\ref{lemma:case16:Cb}, and $b^{-1}d^{-1}b>1$ in \Case{16}{2}{} by Lemma~\ref{lemma:case16:BDb}. Therefore, (\ref{lemma:case16:BCdb:contradiction}) shows that:
\begin{align*}
b^{-1}d^{-1}cb&<1\\
\Rightarrow{}b^{-1}c^{-1}db&>1.\qedhere
\end{align*}
\end{proof}

\begin{lemma} In \Case{16}{2}{}, $c^{-2}dc>1$.
\label{lemma:case16:CCdc}
\end{lemma}
\begin{proof}
Starting from the third group relation, we have:
\begin{align}
(d^{-1}c)^{10n+3}c^{-1}bc^{-2}&=1\nonumber{}\\
\Rightarrow{}c^{-1}bc^{-2}(d^{-1}c)^{10n+3}&=1\nonumber{}\\
\Rightarrow{}c^{-1}bc^{-2}(d^{-1}c)^{10n}(d^{-1}c)^{3}&=1\nonumber{}\\
\Rightarrow{}(d^{-1}c)^{2}c^{-1}bc^{-2}(d^{-1}c)^{10n}(d^{-1}c)&=1\nonumber{}\\
\Rightarrow{}d^{-1}cd^{-1}bc^{-2}(d^{-1}c)^{10n}(d^{-1}c)&=1\nonumber{}\\
\Rightarrow{}d^{-1}cd^{-1}bc^{-1}(c^{-1}d^{-1}c^{2})^{10n}(c^{-1}d^{-1}c)&=1\nonumber{}\\
\Rightarrow{}d^{-1}(d^{-1}cbb^{-1}c^{-1}d)(bb^{-1})cd^{-1}(aa^{-1})bc^{-1}(c^{-1}d^{-1}c^{2})^{10n}(c^{-1}d^{-1}c)&=1\nonumber{}\\
\Rightarrow{}(d^{-2}cb)(b^{-1}c^{-1}db)(b^{-1}cd^{-1}a)(a^{-1}bc^{-1})(c^{-1}d^{-1}c^{2})^{10n}(c^{-1}d^{-1}c)&=1.\label{lemma:case16:CCdc:contradiction}
\end{align}
Now $d^{-2}cb>1$ in \Case{16}{2}{} by Lemma~\ref{lemma:case16:DDcb}, $b^{-1}c^{-1}db>1$ in \Case{16}{2}{} by Lemma~\ref{lemma:case16:BCdb}, $b^{-1}cd^{-1}a>1$ in \Case{16}{2}{} by Lemma~\ref{lemma:case16:BcDa}, $a^{-1}bc^{-1}>1$ in \Case{16}{2}{} by Corollary~\ref{corollary:case16:AbC}, and $c^{-1}d^{-1}c>1$ in \Case{16}{2}{} by Lemma~\ref{lemma:case16:CDc}. Therefore, (\ref{lemma:case16:CCdc:contradiction}) shows that
\begin{align*}
c^{-1}d^{-1}c^{2}&<1\\
\Rightarrow{}c^{-2}dc&>1.\qedhere
\end{align*}
\end{proof}

\begin{lemma} In \Case{16}{2}{}, $cab^{-1}c^{-1}>1$.
\label{lemma:case16:caBC}
\end{lemma}
\begin{proof}
By Lemma~\ref{lemma:eq8} we have:
\begin{align}
d^{2}a&=c^{2}b\nonumber{}\\
\Rightarrow{}d^{2}&=c^{2}ba^{-1}\nonumber{}\\
\Rightarrow{}d^{3}&=dc^{2}ba^{-1}.\label{eq8:5}
\end{align}
By Lemma~\ref{lemma:eq9} we have:
\begin{align}
c^{-3}d^{3}&=b^{-1}a\nonumber{}\\
\Rightarrow{}d^{3}&=c^{3}b^{-1}a.\label{eq9:2}
\end{align}
Combining (\ref{eq8:5}) and (\ref{eq9:2}), we find:
\begin{align}
c^{3}b^{-1}a&=dc^{2}ba^{-1}\nonumber{}\\
\Rightarrow{}c^{-1}d^{-1}c^{3}b^{-1}a^{2}b^{-1}c^{-1}&=1\nonumber{}\\
\Rightarrow{}cba^{-2}bc^{-3}dc&=1\label{eq12}\\
\Rightarrow{}cba^{-1}(c^{-1}c)a^{-1}bc^{-1}c^{-2}dc&=1\nonumber{}\\
\Rightarrow{}(cba^{-1}c^{-1})(ca^{-1}bc^{-1})(c^{-2}dc)&=1.\label{lemma:case16:caBC:contradiction}
\end{align}
Now $ca^{-1}bc^{-1}>1$ in \Case{16}{2}{} by Lemma~\ref{lemma:case16:cAbC} and $c^{-2}dc>1$ in \Case{16}{2}{} by Lemma~\ref{lemma:case16:CCdc}. Therefore, (\ref{lemma:case16:caBC:contradiction}) shows that:
\begin{align*}
cba^{-1}c^{-1}&<1\\
\Rightarrow{}cab^{-1}c^{-1}&>1.\qedhere
\end{align*}
\end{proof}

\begin{proposition} If $G_n$ is left-orderable, then \Case{16}{2}{} is impossible.
\label{proposition:case16.B}
\end{proposition}
\begin{proof} Suppose $G_n$ is left-orderable, and suppose (for contradiction) that the signs of elements are as in \Case{16}{2}{}. Starting from the first group relation, we have:
\begin{align*}
(a^{-1}b)^{10n}d^{-1}a^{2}&=1\\
\Rightarrow{}a^{-2}d(b^{-1}a)^{10n}&=1\\
\Rightarrow{}a^{-2}da^{-1}(ab^{-1})^{10n}a&=1\\
\Rightarrow{}da^{-1}(ab^{-1})^{10n}a^{-1}&=1\\
\Rightarrow{}db^{-1}(ab^{-1})^{10n-1}a^{-1}&=1\\
\Rightarrow{}db^{-1}(ab^{-1})^{10n-2}ab^{-1}a^{-1}&=1\\
\Rightarrow{}db^{-1}c^{-1}(cab^{-1}c^{-1})^{10n-2}cab^{-1}a^{-1}&=1\\
\Rightarrow{}d(a^{-1}a)b^{-1}(d^{-1}aa^{-1}d)c^{-1}(cab^{-1}c^{-1})^{10n-2}c(d^{-1}d)ab^{-1}a^{-1}&=1\\
\Rightarrow{}(da^{-1})(ab^{-1}d^{-1}a)(a^{-1}dc^{-1})(cab^{-1}c^{-1})^{10n-2}(cd^{-1})(dab^{-1}a^{-1})&=1.
\end{align*}
This is a contradiction, since $da^{-1}>1$ in \Case{16}{2}{} by assumption, $ab^{-1}d^{-1}a>1$ in \Case{16}{2}{} by Lemma~\ref{lemma:case16:aBDa}, $a^{-1}dc^{-1}>1$ in \Case{16}{2}{} by Lemma~\ref{lemma:case16:AdC}, $cab^{-1}c^{-1}>1$ in \Case{16}{2}{} by Lemma~\ref{lemma:case16:caBC}, $cd^{-1}>1$ in \Case{16}{2}{} by assumption, and $dab^{-1}a^{-1}>1$ in \Case{16}{2}{} by Lemma~\ref{lemma:case16:daBA}.
\end{proof}

\noindent{}With Proposition~\ref{proposition:case16.B}, we have eliminated the final sub-case of Case 16. We have therefore shown that $G_n$ is not left-orderable.


