\section{Cases 2, 5, 6, 7, 9, 10, 11, 12, 13, 14, and 15}
\label{section:manyCases}

\begin{proposition} If $G_n$ is left-orderable, then Cases 9, 10, 11, and 12 are impossible.\label{proposition:case9,10,11,12}
\end{proposition}
\begin{proof} Suppose $G_n$ is left-orderable. In Cases 9, 10, 11, and 12, $a<1$ and $b>1$ and so $a<b$, but we have taken $b^{-1}a$ to be positive, telling us that $a>b$. Therefore, Cases 9, 10, 11, and 12 are not possible.
\end{proof}

\begin{proposition} If $G_n$ is left-orderable, then Cases 2 and 15 are impossible.\label{proposition:case2,15}
\end{proposition}
\begin{proof} Suppose $G_n$ is left-orderable. The fourth group relation for $G_n$ tells us that:
\begin{align}
d^{2}a^{-1}d(c^{-1}d)^{10n+3}&=1.\label{eq4}
\end{align}
Suppose (for contradiction) that the signs of the generators are as in Case 2. Then, $d<1$, while $a>1$ implies $a^{-1}<1$ and $c>1$ implies $c^{-1}<1$. These statements show that:
\begin{align*}
d^{2}a^{-1}d(c^{-1}d)^{10n+3}<1.
\end{align*}
This contradicts (\ref{eq4}); therefore, Case 2 is impossible.\\
\noindent{}Suppose now (for contradiction) that the signs of the generators are as in Case 15. Then, $d>1$, while $a<1$ implies $a^{-1}>1$, and $c<1$ implies $c^{-1}>1$. These statements show that:
\begin{align*}
d^{2}a^{-1}d(c^{-1}d)^{10n+3}>1.
\end{align*}
This contradicts (\ref{eq4}); therefore, Case 15 is impossible.
\end{proof}

\begin{proposition} If $G_n$ is left-orderable, then Case 7 is impossible.
\label{proposition:case7}
\end{proposition}
\begin{proof} Suppose that $G_n$ is left-orderable. By Lemma~\ref{lemma:eq8} we have:
\begin{align*}
d^{2}a&=c^{2}b.
\end{align*}
This shows that if $a$ and $d$ are both positive then $b$ and $c$ cannot both be negative, eliminating Case 7 as a possibility.
\end{proof}

\begin{proposition} If $G_n$ is left-orderable, then Cases 5, 6, 13, and 14 are impossible.\label{proposition:case5,6,13,14}
\end{proposition}
\begin{proof} Suppose that $G_n$ is left-orderable. By Lemma~\ref{genImp1}, Cases 5, 6, 13, and 14 are not possible, since in these cases $b<1$ but $c>1$.
\end{proof}
\vspace{20 pt}
\noindent{}To summarize, the remaining cases are 1, 3, 4, 8, and 16. They are shown in Table~\ref{table:casesPart1}.

\begin{table}[ht]
\begin{center}
\begin{tabular}{l | l | l | l | l}
Case\hspace{10 pt} & $a$\hspace{10 pt} & $b$\hspace{10 pt} & $c$\hspace{10 pt} & $d$\hspace{10 pt} \\\hline\hline
1 & $+$ & $+$ & $+$ & $+$ \\\hline
3 & $+$ & $+$ & $-$ & $+$ \\\hline
4 & $+$ & $+$ & $-$ & $-$ \\\hline
8 & $+$ & $-$ & $-$ & $-$ \\\hline
16 & $-$ & $-$ & $-$ & $-$
\end{tabular}
\end{center}
\caption{The five cases that remain after considering Propositions~\ref{proposition:case9,10,11,12},~\ref{proposition:case2,15},~\ref{proposition:case7}, and~\ref{proposition:case5,6,13,14}.}
\label{table:casesPart1}
\end{table}

%end{part1}
%begin{case3}

\section{Case 3}
\label{section:case3}

\noindent{}We will now show that if $G_n$ is left-orderable then the four generators cannot have the signs shown in Case 3 (see Table~\ref{table:case3}). To accomplish this we will assume that $G_n$ is left-orderable and that the signs of the generators are as in Case 3 and reach a contradiction. 

\begin{table}[ht]
\begin{center}
\begin{tabular}{l | l | l | l | l}
Case\hspace{10 pt} & $a$\hspace{10 pt} & $b$\hspace{10 pt} & $c$\hspace{10 pt} & $d$\hspace{10 pt} \\\hline\hline
3 & $+$ & $+$ & $-$ & $+$ 
\end{tabular}
\end{center}
\caption{The signs of the four generators in Case 3}
\label{table:case3}
\end{table}

\begin{lemma} In Case 3, $ba^{-1}>1$.
\label{lemma:inEq3.1}
\end{lemma}
\begin{proof}By Lemma~\ref{lemma:eq8}, we have:
\begin{align}
c^{2}b&=d^{2}a\nonumber{}\\
\Rightarrow{}(c^{-2}d^{2})(ab^{-1})&=1.\label{eq8:2}
\end{align}
In Case 3, we assume that $c<1$ and $d>1$, thus $c^{-2}d^{2}>1$, and thus by (\ref{eq8:2}) we see that:
\begin{align*}
ab^{-1}&<1\\
\Rightarrow{}ba^{-1}&>1.\qedhere
\end{align*}
\end{proof}

\begin{proposition}
If $G_n$ is left-orderable, then Case 3 ($a>1$, $b>1$, $c<1$, and $d>1$) is impossible.
\end{proposition}
\begin{proof}Suppose that $G_n$ is left-orderable, and suppose (for contradiction) that $a>1$, $b>1$, $c<1$, and $d>1$. By the second group relation, we have:
\begin{align}
b^{-2}c(b^{-1}a)^{10n}&=1\nonumber{}\\
\Rightarrow{}b(a^{-1}b)^{10n}c^{-1}b&=1\nonumber{}\\
\Rightarrow{}(ba^{-1})^{10n}(bc^{-1}b)&=1.\label{proposition:case3:contradiction}
\end{align}
Now by Lemma~\ref{lemma:inEq3.1}, $(ba^{-1})^{10n}>1$. Further, we are assuming that $b>1$ and $c<1$ so $bc^{-1}b>1$. Therefore, (\ref{proposition:case3:contradiction}) states that a product of positive elements equals the identity, a contradiction.
\end{proof}

%end{case3}
%begin{case4}

\section{Case 4}
\label{section:case4}

\noindent{}We will now show that if $G_n$ is left-orderable then the four generators cannot have the signs shown in Table~\ref{table:case4}. To accomplish this we will assume that $G_n$ is left-orderable and that the signs of the generators are as in Case 4 and reach a contradiction.

\begin{table}[ht]
\begin{center}
\begin{tabular}{l | l | l | l | l}
Case\hspace{10 pt} & $a$\hspace{10 pt} & $b$\hspace{10 pt} & $c$\hspace{10 pt} & $d$\hspace{10 pt} \\\hline\hline
4 & $+$ & $+$ & $-$ & $-$ 
\end{tabular}
\end{center}
\caption{The signs of the four generators in Case 4.}
\label{table:case4}
\end{table}

\begin{lemma} In Case 4, $c^{-1}d>1$.
\label{inEq4.1}
\end{lemma}
\begin{proof} By the third group relation, we have:
\begin{align}
(d^{-1}c)^{10n+3}(c^{-1}bc^{-2})&=1.\label{lemma:inEq4.1:contradiction}
\end{align}
In Case 4, $b>1$, and $c^{-1}>1$, thus (\ref{lemma:inEq4.1:contradiction}) shows that $d^{-1}c<1$ or equivalently $c^{-1}d>1$.
\end{proof}

\begin{lemma} In Case 4, $ab^{-1}>1$.
\label{inEq4.2}
\end{lemma}
\begin{proof} By the first group relation, we have:
\begin{align}
(a^{-1}b)^{10n}d^{-1}a^2&=1\nonumber{}\\
\Rightarrow a^{-1}(ba^{-1})^{10n}ad^{-1}a^{2}&=1\nonumber{}\\
\Rightarrow(ba^{-1})^{10n}(a)(d^{-1})(a)&=1.\label{eq1:4}
\end{align}
In Case 4, $a>1$ and $d^{-1}>1$, thus (\ref{eq1:4}) shows that $ba^{-1}<1$, or equivalently $ab^{-1}>1$.
\end{proof}

\begin{lemma} In Case 4, $c^{-2}d^{2}>1$.\label{inEq4.3}
\end{lemma}
\begin{proof} By the third group relation, we have:
\begin{align}
(d^{-1}c)^{10n+3}c^{-1}bc^{-2}&=1\nonumber{}\\
\Rightarrow(bc^{-2})(d^{-1}c)^{10n+3}(c^{-1})&=1\nonumber{}\\
\Rightarrow(bc^{-2})(cc^{-1})((dd^{-1})d^{-1}c)^{10n+3}(cc^{-1})(c^{-1})&=1\nonumber{}\\
\Rightarrow(bc^{-2})c(c^{-1}dd^{-2}cc)^{10n+3}c^{-1}(c^{-1})&=1\nonumber{}\\
\Rightarrow(b)(c^{-1})([c^{-1}d][d^{-2}c^2])^{10n+3}c^{-2}&=1.\label{lemma:inEq4.3:contradiction}
\end{align}
In Case 4, $b>1$, $c^{-1}>1$, and $c^{-1}d>1$ by Lemma~\ref{inEq4.1}. Therefore, (\ref{lemma:inEq4.3:contradiction}) shows that $d^{-2}c^2<1$, or equivalently, $c^{-2}d^2>1$.
\end{proof}

\begin{proposition} If $G_n$ is left-orderable, then Case 4 ($a>1$, $b>1$, $c<1$, and $d<1$) is impossible.
\end{proposition}
\begin{proof} Suppose $G_n$ is left-orderable. By Lemma~\ref{lemma:eq8}, we have:
\begin{align}
d^2a&=c^2b\nonumber{}\\
\Rightarrow b^{-1}c^{-2}d^2a&=1\nonumber{}\\
\Rightarrow (c^{-2}d^2)(ab^{-1})&=1.\label{proposition:case4:contradiction}
\end{align}
But $ab^{-1}>1$ by Lemma~\ref{inEq4.2} and $c^{-2}d^2>1$ by Lemma~\ref{inEq4.3}, thus (\ref{proposition:case4:contradiction}) is a contradiction.
\end{proof}

%end{case4}
%begin{case8}

\section{Case 8}
\label{section:case8}

\noindent{}We will now show that if $G_n$ is left-orderable then the four generators cannot have the signs shown in Table~\ref{table:case8}. To accomplish this we will assume that $G_n$ is left-orderable and that the signs of the generators are as in Case 8 and reach a contradiction. 

\begin{table}[ht]
\begin{center}
\begin{tabular}{l | l | l | l | l}
Case\hspace{10 pt} & $a$\hspace{10 pt} & $b$\hspace{10 pt} & $c$\hspace{10 pt} & $d$\hspace{10 pt} \\\hline\hline
8 & $+$ & $-$ & $-$ & $-$ \\
\end{tabular}
\end{center}
\caption{The signs of the four generators in Case 8.}
\label{table:case8}
\end{table}

\begin{lemma} In Case 8, $bc^{-1}>1$.
\label{lemma:inEq8.2}
\end{lemma}
\begin{proof} In Case 8, we have:
\begin{align}
ab^{-1}>1,\label{inEq8.1}
\end{align} 
since $a>1$ and $b^{-1}>1$. The second group relation of $G_n$ tells us that:
\begin{align}
b^{-2}c(b^{-1}a)^{10n}&=1\nonumber{}\\
\Rightarrow{}b^{-1}c(b^{-1}a)^{10n}b^{-1}&=1\nonumber{}\\
\Rightarrow{}(b^{-1})(cb^{-1})(ab^{-1})^{10n}&=1.\label{lemma:inEq8.2:contradiction}
\end{align}
In Case 8, $b^{-1}>1$ and $ab^{-1}>1$ by (\ref{inEq8.1}), therefore (\ref{lemma:inEq8.2:contradiction}) tells us that:
\begin{align*}
cb^{-1}&<1\\
\Rightarrow{}bc^{-1}&>1.\qedhere
\end{align*}
\end{proof}

\begin{lemma} In Case 8, $c^{-1}b>1$.
\label{lemma:inEq8.3}
\end{lemma}
\begin{proof} The second group relation of $G_n$ tells us that:
\begin{align}
b^{-2}c(b^{-1}a)^{10n}&=1\nonumber{}\\
\Rightarrow{}(b^{-1}c)(b^{-1}a)^{10n}b^{-1}&=1.\label{lemma:inEq8.3:contradiction}
\end{align}
In Case 8, $b<1\Rightarrow{}b^{-1}>1$ and we have assumed in general that $b^{-1}a>1$, thus (\ref{lemma:inEq8.3:contradiction}) tells us that:
\begin{align*}
b^{-1}c&<1\\
\Rightarrow{}c^{-1}b&>1.\qedhere
\end{align*}
\end{proof}

\begin{lemma} In Case 8, $c^{-1}d>1$.
\label{lemma:inEq8.4}
\end{lemma}
\begin{proof} In Case 8, $c<1$ so $c^{-2}>1$, and Lemma~\ref{lemma:inEq8.3} states $c^{-1}b>1$. Therefore, we have:
\begin{align}
c^{-1}bc^{-2}>1.\label{inEq8.5}
\end{align}
The third group relation tells us that:
\begin{align}
(d^{-1}c)^{10n+3}c^{-1}bc^{-2}&=1\nonumber{}\\
\Rightarrow{}(d^{-1}c)^{10n+3}(c^{-1}bc^{-2})&=1.\label{lemma:inEq8.4:contradiction}
\end{align}
Together, (\ref{inEq8.5}) and (\ref{lemma:inEq8.4:contradiction}) show that:
\begin{align*}
d^{-1}c&<1\\
\Rightarrow{}c^{-1}d&>1.\qedhere
\end{align*}
\end{proof}

\begin{lemma} In Case 8, $d^{-2}c^{2}>1$.
\label{lemma:inEq8.6} 
\end{lemma}
\begin{proof} By Lemma~\ref{lemma:eq8}, we have:
\begin{align}
d^{2}a&=c^{2}b\nonumber{}\\
\Rightarrow{}(ab^{-1})(c^{-2}d^{2})&=1.\label{lemma:inEq8.6:contradiction}
\end{align}
In Case 8, $b<1$ and $a>1$ so $ab^{-1}>1$. Therefore, (\ref{lemma:inEq8.6:contradiction}) shows that:
\begin{align*}
c^{-2}d^{2}&<1\\
\Rightarrow{}d^{-2}c^{2}&>1.\qedhere
\end{align*}
\end{proof}

\begin{proposition} If $G_n$ is left-orderable, then Case 8 ($a>1$, $b<1$, $c<1$, and $d<1$) is impossible.
\label{proposition:case8}
\end{proposition}
\begin{proof} Suppose that $G_n$ is left-orderable, and suppose (for contradiction), that $a>1$, $b<1$, $c<1$, and $d<1$.
By the third group relation, we have: 
\begin{align}
(d^{-1}c)^{10n+3}c^{-1}bc^{-2}&=1\nonumber{}\\
\Rightarrow{}(bc^{-2})((d^{-1}c)^{10n+3})(c^{-1})&=1\nonumber{}\\
\Rightarrow{}(bc^{-2})(cc^{-1})(((dd^{-1})d^{-1}c)^{10n+3})(cc^{-1})(c^{-1})&=1\nonumber{}\\
\Rightarrow{}(bc^{-2}c)(c^{-1}dd^{-2}cc)^{10n+3}(c^{-1})(c^{-1})&=1\nonumber{}\\
\Rightarrow{}(bc^{-1})([c^{-1}d][d^{-2}c^{2}])^{10n+3}(c^{-2})&=1.\label{proposition:case8:contradiction}
\end{align}
By Lemma~\ref{lemma:inEq8.2}, $bc^{-1}>1$, by Lemma~\ref{lemma:inEq8.4}, $c^{-1}d>1$, and by Lemma~\ref{lemma:inEq8.6}, $d^{-2}c^{2}>1$. Furthermore, in Case 8 $c<1$ so $c^{-2}>1$. Therefore, (\ref{proposition:case8:contradiction}) states that a product of positives is the identity, a contradiction.
\end{proof}