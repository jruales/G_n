\section{Case 16}
\label{section:case16}

\noindent{}We will now show that if $G_n$ is left-orderable, then Case 16 (see Table~\ref{table:case16}) is not possible.

\begin{table}[ht]
\begin{center}
\begin{tabular}{l | l | l | l | l}
Case\hspace{10 pt} & $a$\hspace{10 pt} & $b$\hspace{10 pt} & $c$\hspace{10 pt} & $d$\hspace{10 pt} \\\hline\hline
16 & $-$ & $-$ & $-$ & $-$
\end{tabular}
\end{center}
\caption{The signs of the 4 generators in Case 16.}
\label{table:case16}
\end{table}

\noindent{}We start by proving the signs of a few key elements.

\subsection{Inequalities for Case 16}

\begin{lemma} In Case 16, $d^{-1}a>1$.
\label{lemma:case16:Da}
\end{lemma}
\begin{proof}
By (\ref{eq1:2}):
\begin{align*}
d^{-1}a^{2}=(b^{-1}a)^{10n} \\
\Rightarrow{}d^{-1}a=(b^{-1}a)^{10n}a^{-1}.
\end{align*}
Thus, $d^{-1}a$ is a product of positive elements.\qedhere
\end{proof}

\begin{lemma} In Case 16, $c^{-1}b>1$.
\label{lemma:case16:Cb}
\end{lemma}
\begin{proof}
By (\ref{eq2:2}):
\begin{align*}
c^{-1}b^{2}=(b^{-1}a)^{10n}\\
\Rightarrow{}c^{-1}b=(b^{-1}a)^{10n}b^{-1}.
\end{align*}
Thus, $c^{-1}b$ is a product of positive elements.
\end{proof}

\begin{lemma} In Case 16, $c^{-1}d>1$.
\label{lemma:case16:Cd}
\end{lemma}
\begin{proof} Suppose (for contradiction) that $c^{-1}d<1$, or equivalently that $d^{-1}c>1$. By the third group relation, we have:
\begin{align}
(d^{-1}c)^{10n+3}c^{-1}bc^{-2}&=1\nonumber{}\\
\Rightarrow{}(d^{-1}c)^{10n+3}&=c^{2}b^{-1}c.\label{lemma:case16:Cd:contradiction}
\end{align}
Since we are assuming $d^{-1}c>1$, (\ref{lemma:case16:Cd:contradiction}) shows that $c^{2}b^{-1}c>1$. By (\ref{eq2:2}), $c^{-1}b^{2}=(b^{-1}a)^{10n}>1$, thus:
\begin{align*}
(c^{2}b^{-1}c)(c^{-1}b^{2})>1\\
\Rightarrow{}c^{2}b>1.
\end{align*}
This is a contradiction, since both $b$ and $c$ are negative in Case 16. Therefore, $c^{-1}d\geq{}1$. However, by Proposition~\ref{proposition:non-trivial}, $c\neq{}d$ and thus $c^{-1}d\neq{}1$. Therefore, $c^{-1}d>1$.
\end{proof}

\begin{corollary} In Case 16, $c^{-1}a>1$.
\label{lemma:case16:Ca}
\end{corollary}
\begin{proof}
This follows from Lemma~\ref{lemma:case16:Da} and Lemma~\ref{lemma:case16:Cd} since:
\begin{align*}
c^{-1}a&=(c^{-1}d)(d^{-1}a).\qedhere
\end{align*}
\end{proof}

\begin{lemma} In Case 16, $ba^{-1}>1$.
\label{lemma:case16:bA}
\end{lemma}
\begin{proof}
Starting from the fourth group relation, we have:
\begin{align}
d^{2}a^{-1}d(c^{-1}d)^{10n+3}&=1\nonumber{}\\
\Rightarrow{}d^{2}a^{-1}d^{2}(d^{-1}(cc^{-1})c^{-1}d^{2})^{10n+3}d^{-1}&=1\nonumber{}\\
\Rightarrow{}da^{-1}d^{2}([d^{-1}c][c^{-2}d^{2}])^{10n+3}&=1\nonumber{}\\
\Rightarrow{}da^{-1}d^{2}([d^{-1}c][ba^{-1}])^{10n+3}&=1,\label{lemma:case16:bA:contradiction}
\end{align}
where the last implication follows from Lemma \ref{lemma:eq8}. Now $a^{-1}d<1$ by Lemma~\ref{lemma:case16:Da}, so we know $da^{-1}d^{2}<1$. Therefore, (\ref{lemma:case16:bA:contradiction}) tells us that:
\begin{align*}
([d^{-1}c][ba^{-1}])^{10n+3}>1,
\end{align*}
but $d^{-1}c<1$ by Lemma~\ref{lemma:case16:Cd}, so we must have:
\begin{align*}
ba^{-1}&>1.\qedhere
\end{align*}
\end{proof}



\subsection{Concordance of signs of a few useful elements}

Next we will show that in Case 16, $bc^{-1}$, $dc^{-1}$, $ad^{-1}$, $ac^{-1}$, $bc^{-3}$, and $ad^{-3}$ must all have the same sign. We begin by showing that all of these elements are non-trivial.

\begin{proposition}
In Case 16, $bc^{-1}\neq1$, $dc^{-1}\neq1$, $ad^{-1}\neq1$, $ac^{-1}\neq1$, $bc^{-3}\neq1$, and $ad^{-3}\neq1$.
\end{proposition}
\begin{proof}
Consequence of Proposition~\ref{proposition:non-trivial} and Corollary~\ref{corollary:non-trivial2}.
\end{proof}

\noindent{}Now that we know that each of these elements can only be positive or negative, we proceed to show that they all have the same sign.


\begin{lemma} In Case 16, $ad^{-1}<1$ implies $dc^{-1}<1$.
\label{lemma:iffs:1}
\end{lemma}
\begin{proof}
Starting from the second group relation, we have:
\begin{align}
b^{-2}c(b^{-1}a)^{10n}&=1\nonumber{}\\
\Rightarrow{}b^{-1}c(b^{-1}a)^{10n}b^{-1}&=1\nonumber{}\\
\Rightarrow{}b^{-1}cb^{-1}(ab^{-1})^{10n}&=1\nonumber{}\\
\Rightarrow{}b^{-1}cb^{-1}a^{-1}(a^{2}b^{-1}a^{-1})^{10n}a&=1\nonumber{}\\
\Rightarrow{}ab^{-1}cb^{-1}a^{-1}((a^{2}b^{-2})(ba^{-1}))^{10n}&=1\nonumber{}\\
\Rightarrow{}a(b^{-1}cb^{-1})a^{-1}([dc^{-1}][ba^{-1}])^{10n}&=1\nonumber{}\\
\Rightarrow{}da^{-2}([dc^{-1}][ba^{-1}])^{10n}&=1.\label{lemma:iffs:1:contradiction}
\end{align}
Where the second to last implication follows from Corollary~\ref{corollary:eq6} and last implication follows from Lemma~\ref{lemma:eq16}. Now if $ad^{-1}<1$ we have:
\begin{align*}
da^{-1}&>1\\
\Rightarrow{}da^{-2}&>1.
\end{align*}
Thus if $ad^{-1}<1$, (\ref{lemma:iffs:1:contradiction}) implies:
\begin{align*}
(dc^{-1})(ba^{-1})<1,
\end{align*}
which shows that $dc^{-1}<1$, since $ba^{-1}>1$ by Lemma~\ref{lemma:case16:bA}.
\end{proof}

\begin{lemma} In Case 16, $dc^{-1}>1$ implies $ac^{-1}>1$.
\label{lemma:iffs:2.1}
\end{lemma}
\begin{proof}
By Lemma~\ref{lemma:iffs:1}:
\begin{align*}
dc^{-1}>1 \Rightarrow{} ad^{-1}>1.
\end{align*}
This completes the proof since:
\begin{align*}
ac^{-1}&=(ad^{-1})(dc^{-1}).\qedhere
\end{align*}
\end{proof}

\begin{lemma} In Case 16, $ad^{-1}<1$ implies $ac^{-1}<1$.
\label{lemma:iffs:2.2}
\end{lemma}
\begin{proof}
By Lemma~\ref{lemma:iffs:1}:
\begin{align*}
ad^{-1}<1 \Rightarrow{} dc^{-1}<1.
\end{align*}
This completes the proof since:
\begin{align*}
ac^{-1}&=(ad^{-1})(dc^{-1}).\qedhere
\end{align*}
\end{proof}

\begin{corollary} In Case 16, $dc^{-1}>1$ if and only if $ad^{-1}>1$ if and only if $ac^{-1}>1$.
\label{lemma:iffs:3}
\end{corollary}
\begin{proof}
Starting from the fourth group relation, we have:
\begin{align}
d^{2}a^{-1}d(c^{-1}d)^{10n+3}&=1\nonumber{}\\
\Rightarrow{}a^{-1}d(c^{-1}d)^{10n+3}d^{2}&=1\nonumber{}\\
\Rightarrow{}a^{-1}(dc^{-1})^{10n+3}d^{3}&=1\nonumber{}\\
\Rightarrow{}d^{2}(da^{-1})(dc^{-1})^{10n+3}&=1.\label{lemma:iffs:3:contradiction}
\end{align}
Now $d<1$ in Case 16, so (\ref{lemma:iffs:3:contradiction}) shows that:
\begin{align*}
ad^{-1}>1\Rightarrow{}da^{-1}<1\Rightarrow{}dc^{-1}>1.
\end{align*}
In conjunction with Lemma~\ref{lemma:iffs:1}, this shows that $dc^{-1}>1$ if and only if $ad^{-1}>1$. Now by Lemma~\ref{lemma:iffs:2.1} we have:
\begin{align*}
ad^{-1}>1\Rightarrow{}dc^{-1}>1\Rightarrow{}ac^{-1}>1,
\end{align*}
and by Lemma~\ref{lemma:iffs:2.2} we have:
\begin{align*}
ad^{-1}<1\Rightarrow{}ac^{-1}<1.
\end{align*}
Thus, $ad^{-1}>1$ if and only if $ac^{-1}>1$.
\end{proof}

\begin{lemma} In Case 16, $dc^{-1}>1$ if and only if $bc^{-1}>1$.
\label{lemma:iffs:4}
\end{lemma}
\begin{proof}[Proof of reverse direction]
Starting from the third group relation, we have:
\begin{align*}
(d^{-1}c)^{10n+3}c^{-1}bc^{-2}&=1\\
\Rightarrow{}c^{-1}(cd^{-1})^{10n+3}bc^{-2}&=1\\
\Rightarrow{}(cd^{-1})^{10n+3}(bc^{-2})&=c<1,
\end{align*}
and so:
\begin{align*}
dc^{-1}<1\Rightarrow{}cd^{-1}>1\Rightarrow{}bc^{-2}&<1\Rightarrow{}bc^{-1}<1. \qedhere
\end{align*}
\end{proof}

\begin{proof}[Proof of forward direction]
First note that:
\begin{align*}
bc^{-1}=(ba^{-1})(ac^{-1}).
\end{align*}
But $ba^{-1}>1$ by Lemma~\ref{lemma:case16:bA} so this shows that:
\begin{align*}
dc^{-1}>1\Rightarrow{}ac^{-1}>1\Rightarrow{}bc^{-1}>1,
\end{align*}
where the first implication follows from Corollary~\ref{lemma:iffs:3}.
\end{proof}

\begin{lemma} In Case 16, $dc^{-1}>1$ if and only if $bc^{-3}>1$ if and only if $ad^{-3}>1$.
\label{lemma:iffs:5}
\end{lemma}
\begin{proof}
By the third group relation, we have:
\begin{align*}
(d^{-1}c)^{10n+3}c^{-1}bc^{-2}&=1\\
\Rightarrow{}(cd^{-1})^{10n+3}bc^{-3}&=1\\
\Rightarrow{}bc^{-3}&=(dc^{-1})^{10n+3}.
\end{align*}
Thus $bc^{-3}>1$ must have the same sign as $dc^{-1}$. By the fourth group relation, we have:
\begin{align*}
(d^{-1}c)^{10n+3}d^{-1}ad^{-2}&=1\\
\Rightarrow{}(cd^{-1})^{10n+3}ad^{-3}&=1\\
\Rightarrow{}ad^{-3}&=(dc^{-1})^{10n+3}.
\end{align*}
Thus $ad^{-3}>1$ must have the same sign as $dc^{-1}$.
\end{proof}

\begin{proposition} In Case 16, the following elements all have the same sign: $dc^{-1}$, $ad^{-1}$, $ac^{-1}$, $bc^{-1}$, $bc^{-3}$, and $ad^{-3}$.
\label{proposition:iffs}
\end{proposition}
\begin{proof}
The proposition is evident by combining Corollary~\ref{lemma:iffs:3}, Lemma~\ref{lemma:iffs:4}, and Lemma~\ref{lemma:iffs:5}.
\end{proof}

\noindent{}In order to show Case 16 ($a,b,c,d<1$) is not possible if $G_n$ is left-orderable, we consider sub-cases (see Table~\ref{table:cases16b}). Because of Proposition~\ref{proposition:iffs} , it is easy to see that there are only two possible sub-cases of Case 16 considering the signs of $dc^{-1}$, $ad^{-1}$, $ac^{-1}$, $bc^{-1}$, $bc^{-3}$, and $ad^{-3}$

\begin{table}[ht]
\begin{center}
\begin{tabular}{l | l | l | l | l | l | l}
Case \hspace{10 pt} & $dc^{-1}$\hspace{10 pt} & $ad^{-1}$\hspace{10 pt} & $ac^{-1}$\hspace{10 pt} & $bc^{-1}$\hspace{10 pt} & $bc^{-3}$\hspace{10 pt} & $ad^{-3}$\hspace{10 pt}  \\\hline\hline
\case{16}{1}{} & $+$ & $+$ & $+$ & $+$ & $+$ & $+$ \\\hline
\case{16}{2}{} & $-$ & $-$ & $-$ & $-$ & $-$ & $-$
\end{tabular}
\end{center}
\caption{The two possible sub-cases of Case 16 considering the signs of $dc^{-1}$, $ad^{-1}$, $ac^{-1}$, $bc^{-1}$, $bc^{-3}$, and $ad^{-3}$}
\label{table:cases16b}
\end{table}

\subsection{\Case{16}{1}{}}

As a reminder, since we are working in a sub-case of Case 16, we know $a<1$, $b<1$, $c<1$, and $d<1$.

\begin{lemma} In \Case{16}{1}{}, $a^{-1}dc^{-1}a>1$.
\label{lemma:case16.A:AdCa}
\end{lemma}
\begin{proof} By the third group relation, we have:
\begin{align}
(d^{-1}c)^{10n+3}c^{-1}bc^{-2}&=1\nonumber{}\\
\Rightarrow{}c^{-1}(cd^{-1})^{10n+3}bc^{-2}&=1\nonumber{}\\
\Rightarrow{}c^{-1}(aa^{-1})(cd^{-1})^{10n+3}(aa^{-1})b(a^{-1}a)c^{-1}(a^{-1}a)c^{-1}&=1\nonumber{}\\
\Rightarrow{}(c^{-1}a)(a^{-1}cd^{-1}a)^{10n+3}(a^{-1})(ba^{-1})(ac^{-1})(a^{-1})(ac^{-1})&=1.\label{16.A:AdCa}\
\end{align}
But $c^{-1}a>1$ by Corollary \ref{lemma:case16:Ca} , $a^{-1}>1$ in Case 16, $ba^{-1}>1$ by Lemma \ref{lemma:case16:bA} and $ac^{-1}>1$ in \Case{16}{1}{} by assumption, so (\ref{16.A:AdCa}) shows that:
\begin{align*}
a^{-1}cd^{-1}a&<1\\
\Rightarrow{}a^{-1}dc^{-1}a&>1.\qedhere
\end{align*}
\end{proof}

\begin{lemma} In \Case{16}{1}{}, $ab^{-2}a>1$.
\label{lemma:case16.A:aBBa}
\end{lemma}
\begin{proof} By Lemma \ref{lemma:eq5}, we have:
\begin{align}
dc^{-1}b^{2}a^{-2}=1\nonumber{}\\
\Rightarrow{}a^{-1}dc^{-1}b^{2}a^{-1}=1\nonumber{}\\
\Rightarrow{}a^{-1}dc^{-1}(aa^{-1})b^{2}a^{-1}=1\nonumber{}\\
\Rightarrow{}(a^{-1}dc^{-1}a)(a^{-1}b^{2}a^{-1})=1.\label{16.A:aBBa}
\end{align}
But $a^{-1}dc^{-1}a>1$ by Lemma \ref{lemma:case16.A:AdCa}, so (\ref{16.A:aBBa}) shows that:
\begin{align*}
a^{-1}b^{2}a^{-1}&<1\\
\Rightarrow{}ab^{-2}a&>1.\qedhere
\end{align*}
\end{proof}

\begin{lemma} In \Case{16}{1}{}, $d^{-2}cb>1$.
\label{lemma:case16.A:DDcb}
\end{lemma}
\begin{proof} By Lemma \ref{lemma:eq8} and Lemma \ref{lemma:eq16}:
\begin{align}
d^{-2}cb=&d^{-2}c(cb^{-1}bc^{-1})b\nonumber{}\\
\Rightarrow{}d^{-2}cb=&(d^{-2}c^{2})b^{-1}(bc^{-1}b)\nonumber{}\\
\Rightarrow{}d^{-2}cb=&(ab^{-1})b^{-1}(ad^{-1}a)\nonumber{}\\
\Rightarrow{}d^{-2}cb=&(ab^{-2}a)(d^{-1}a).
\label{16.A:DDcb}
\end{align}
But $ab^{-2}a>1$ by Lemma \ref{lemma:case16.A:aBBa}, and $d^{-1}a>1$ by Lemma \ref{lemma:case16:Da}, so (\ref{16.A:DDcb}) shows that:
\begin{align*}
(ab^{-2}a)(d^{-1}a)&>1\\
\Rightarrow{}d^{-2}cb&>1.\qedhere
\end{align*}
\end{proof}

\begin{lemma} In \Case{16}{1}{}, $a^{-1}d^{-2}c^{2}>1$.
\label{lemma:case16.A:ADDcc}
\end{lemma}
\begin{proof} By Lemma \ref{lemma:eq8}:
\begin{align}
b^{-1}c^{-2}d^{2}a=1\nonumber{}\\
\Rightarrow{}b^{-1}(aa^{-1})c^{-2}d^{2}a=1\nonumber{}\\
\Rightarrow{}(b^{-1}a)(a^{-1})(c^{-2}d^{2}a)=1.
\label{16.A:ADDcc}
\end{align}
But $b^{-1}a>1$ in general, and $a^{-1}>1$ in case 16, so (\ref{16.A:ADDcc}) shows that:
\begin{align*}
c^{-2}d^{2}a&<1\\
\Rightarrow{}a^{-1}d^{-2}c^{2}&>1.\qedhere
\end{align*}
\end{proof}

\begin{corollary} In \Case{16}{1}{} $a^{-1}d^{-2}c>1$.
\label{corollary:case16.A:ADDc}
\end{corollary}
\begin{proof} We know $c^{-1}>1$ in Case 16, and by Lemma~\ref{lemma:case16.A:ADDcc}, we know $a^{-1}d^{-2}c^{2}>1$ in \Case{16}{1}{}. Therefore,
\begin{align*}
 (a^{-1}d^{-2}c^{2})(c^{-1})&>1\\
 \Rightarrow{}a^{-1}d^{-2}c&>1.\qedhere
 \end{align*}
 \end{proof}
 
\begin{lemma} In \Case{16}{1}{}, $a^{-1}c^{-1}da > 1$.
\label{lemma:case16:A:ACda}
\end{lemma}
\begin{proof} By the third group relation, we have:
\begin{align}
(d^{-1}c)^{10n+3}c^{-1}bc^{-2} =1\nonumber{}\\
\Rightarrow{}(d^{-1}c)^{10n+2}d^{-1}bc^{-2} =1\nonumber{}\\
\Rightarrow{}d^{-1}bc^{-2}(d^{-1}c)^{10n+2}=1\nonumber{}\\
\Rightarrow{}(a^{-1}a)d^{-1}b(a^{-1}ad^{-1}d)c^{-2}(aa^{-1})(d^{-1}c)^{10n+2}(aa^{-1})=1\nonumber{}\\
\Rightarrow{}(a^{-1})(ad^{-1})(ba^{-1})(ad^{-1})(dc^{-1})(c^{-1}a)(a^{-1}d^{-1}ca)^{10n+2}(a^{-1})=1.
\label{16.A:ACda}
\end{align} 
But $a^{-1}>1$ in Case 16, $ad^{-1}>1$ in \Case{16}{1}{}, $ba^{-1}>1$ by Lemma \ref{lemma:case16:bA}, $dc^{-1}>1$ in \Case{16}{1}{}, and $c^{-1}a>1$ by Lemma \ref{lemma:case16:Ca}, so (\ref{16.A:ACda}) shows that:
\begin{align*}
a^{-1}d^{-1}ca&<1\\
\Rightarrow{}a^{-1}c^{-1}da&>1.\qedhere
\end{align*}
\end{proof}

\begin{lemma} In \Case{16}{1}{}, $ac^{-1}da^{-1}>1$.
\label{lemma:case16.A:aCdA}
\end{lemma}
\begin{proof} By the third group relation, we have:
\begin{align}
(d^{-1}c)^{10n+3}c^{-1}bc^{-2}=1\nonumber{}\\
\Rightarrow{}(a^{-1}a)(d^{-1}c)^{10n+2}(a^{-1}a)(d^{-1}c)c^{-1}b(a^{-1}a)c^{-1}(a^{-1}a)c^{-1}=1\nonumber{}\\
\Rightarrow{}(a^{-1})(ad^{-1}ca^{-1})^{10n+2}(ad^{-1})(cc^{-1})(ba^{-1})(ac^{-1})(a^{-1})(ac^{-1})=1\nonumber{}\\
\Rightarrow{}(a^{-1})(ad^{-1}ca^{-1})^{10n+2}(ad^{-1})(ba^{-1})(ac^{-1})(a^{-1})(ac^{-1})=1.
\label{16.A:aCdA}
\end{align}
But $ba^{-1}>1$ by Lemma \ref{lemma:case16:bA}, $a^{-1}>1$ in Case 16, $ad^{-1}>1$ in \Case{16}{1}{}, , and $ac^{-1}>1$ in \Case{16}{1}{}, so (\ref{16.A:aCdA}) shows that:
\begin{align*}
ad^{-1}ca^{-1}&<1\\
\Rightarrow{}ac^{-1}da^{-1}&>1.\qedhere
\end{align*}
\end{proof}

\begin{lemma} In \Case{16}{1}{}, $b^{-1}cd^{-1}a>1$.
\label{lemma:case16.A:BcDa}
\end{lemma}
\begin{proof} By Lemma \ref{lemma:eq5}, we have:
\begin{align}
dc^{-1}b^{2}a^{-2}=1\nonumber{}\\
\Rightarrow{}a^{-1}dc^{-1}b^{2}a^{-1}=1\nonumber{}\\
\Rightarrow{}(a^{-1}dc^{-1}b)(ba^{-1})=1.
\label{16.A:BcDa}
\end{align}
But $ba^{-1}>1$ by Lemma \ref{lemma:case16:bA}, so (\ref{16.A:BcDa}) shows that:
\begin{align*}
a^{-1}dc^{-1}b&<1\\
\Rightarrow{}b^{-1}cd^{-1}a&>1.\qedhere
\end{align*}
\end{proof}

\begin{lemma} In \Case{16}{1}{}, $ad^{-2}c>1$.
\label{lemma:16:A:aDDc}
\end{lemma}
\begin{proof} By Lemma~\ref{lemma:eq7}, we have:
\begin{align}
d^{2}a^{-1}d&=c^{2}b^{-1}c\nonumber{}\\
\Rightarrow{}bc^{-2}d^{2}a^{-1}dc^{-1}&=1\nonumber{}\\
\Rightarrow{}b(a^{-1}a)(d^{-1}d)c^{-2}d^{2}a^{-1}dc^{-1}&=1\nonumber{}\\
\Rightarrow{}(ba^{-1})(ad^{-1})(dc^{-1})(c^{-1}d^{2}a^{-1})(dc^{-1})&=1.\label{lemma:16:A:aDDc:contradiction}
\end{align}
But $ba^{-1}>1$ in Case 16 by Lemma~\ref{lemma:case16:bA}, $ad^{-1}>1$ in \Case{16}{1}{} by assumption, and $dc^{-1}>1$ in \Case{16}{1}{} by assumption. Therefore, (\ref{lemma:16:A:aDDc:contradiction}) shows that:
\begin{align*}
c^{-1}d^{2}a^{-1}&<1\\
\Rightarrow{}ad^{-2}c&>1.\qedhere
\end{align*}
\end{proof}

\begin{lemma} In \Case{16}{1}{}, $a^{-1}d^{-1}cd^{-1}c>1$.
\label{lemma:16:A:ADcDc}
\end{lemma}
\begin{proof} By Lemma~\ref{lemma:eq7}, we have:
\begin{align}
d^{2}a^{-1}d&=c^{2}b^{-1}c\nonumber{}\\
\Rightarrow{}bc^{-2}d^{2}a^{-1}dc^{-1}&=1\nonumber{}\\
\Rightarrow{}b(a^{-1}a)c^{-1}(da^{-1}ad^{-1})(d^{-1}cc^{-1}d)c^{-1}d(aa^{-1})da^{-1}dc^{-1}&=1\nonumber{}\\
\Rightarrow{}(ba^{-1})(ac^{-1}da^{-1})(ad^{-2}c)(c^{-1}dc^{-1}da)(a^{-1}da^{-1})(dc^{-1})&=1.\label{lemma:16:A:ADcDc:contradiction}
\end{align}
But $ba^{-1}>1$ in Case 16 by Lemma~\ref{lemma:case16:bA}, $ac^{-1}da^{-1}>1$ by Lemma~\ref{lemma:case16.A:aCdA}, $ad^{-2}c>1$ in \Case{16}{1}{} by Lemma~\ref{lemma:16:A:aDDc}, $a^{-1}da^{-1}>1$ in Case 16 since $a^{-1}da^{-1}=(ba^{-1})^{10n}$ (see (\ref{eq1:3})) which is positive by Lemma~\ref{lemma:case16:bA}, and $dc^{-1}>1$ in \Case{16}{1}{} by assumption. Therefore, (\ref{lemma:16:A:ADcDc:contradiction}) shows that:
\begin{align*}
c^{-1}dc^{-1}da&<1\\
\Rightarrow{}a^{-1}d^{-1}cd^{-1}c&>1.\qedhere
\end{align*}
\end{proof}

\begin{lemma} In \Case{16}{1}{}, $a^{-1}c^{-2}dca>1$.
\label{lemma:case16:A:ACCdca}
\end{lemma}
\begin{proof} Suppose that $G_n$ is left-orderable, and suppose (for contradiction), that $a<1$, $b<1$, $c<1$, and $d<1$. By the third group relation, we have: 
\begin{align}
(d^{-1}c)^{10n+3}c^{-1}bc^{-2}&=1\nonumber{}\\
\Rightarrow{}(d^{-1}c)^{10n+2}d^{-1}bc^{-2}&=1\nonumber{}\\
\Rightarrow{}d^{-1}bc^{-2}(d^{-1}c)^{10n+2}&=1\nonumber{}\\
\Rightarrow{}(d^{-1}bc^{-2})(c^{-1}aa^{-1}c^{-1})(d^{-1}c)^{10n-2}(caa^{-1}c^{-1})&\nonumber{}\\
(daa^{-1}d^{-1})(d^{-1}c)(cc^{-1}dd^{-1})(d^{-1}c)(bb^{-1}aa^{-1})(d^{-1}cd^{-1}c)&=1\nonumber{}\\
\Rightarrow{}(d^{-1})(bc^{-2})(c^{-1}a)(a^{-1}c^{-1}d^{-1}cca)^{10n-2}(a^{-1}c^{-1}da)&\nonumber{}\\
(a^{-1}d^{-1}d^{-1}cc)(c^{-1}d)(d^{-1}d^{-1}cb)(b^{-1}a)(a^{-1}d^{-1}cd^{-1}c)&=1\nonumber{}\\
\Rightarrow{}(d^{-1})(bc^{-2})(c^{-1}a)(a^{-1}c^{-1}d^{-1}c^{2}a)^{10n-2}(a^{-1}c^{-1}da)&\nonumber{}\\
(a^{-1}d^{-2}c^{2})(c^{-1}d)(d^{-2}cb)(b^{-1}a)(a^{-1}d^{-1}cd^{-1}c)&=1.
\label{16.A:ACCdca}
\end{align}
But $d^{-1}>1$ by Case 16, $bc^{-2}>1$ by \Case{16}{1}{}, $c^{-1}a>1$ by Lemma \ref{lemma:case16:Ca}, $a^{-1}c^{-1}da>1$ by Lemma \ref{lemma:case16:A:ACda}, $a^{-1}d^{-2}c^{2}>1$ by Lemma \ref{lemma:case16.A:ADDcc}, $c^{-1}d>1$ by Lemma \ref{lemma:case16:Cd}, $d^{-2}cb>1$ by Lemma \ref{lemma:case16.A:DDcb}, $b^{-1}a>1$ by general assumption, and $a^{-1}d^{-1}cd^{-1}c>1$ by Lemma \ref{lemma:16:A:ADcDc}, so (\ref{16.A:ACCdca}) shows that:
\begin{align*}
a^{-1}c^{-1}d^{-1}c^{2}a&<1\\
\Rightarrow{}a^{-1}c^{-2}dca&>1.\qedhere
\end{align*}
\end{proof}

\begin{proposition}  If $G_n$ is left-orderable, then \Case{16}{1}{} ($a<1$, $b<1$, $c<1$, and $d<1$) is impossible.
\label{proposition:case16.A}
\end{proposition}
\begin{proof} By Lemma \ref{lemma:eq8}, we have:
\begin{align}
b^{-1}c^{-2}d^{2}a&=1\nonumber{}\\
\Rightarrow{}b^{-1}(aa^{-1})c^{-2}d(caa^{-1}c^{-1})da&=1\nonumber{}\\
\Rightarrow{}(b^{-1}a)(a^{-1}c^{-2}dca)(a^{-1}c^{-1}da)&=1.
\label{proposition:case16.A:contradiction}
\end{align}
But $b^{-1}a>1$ by general assumption, $a^{-1}c^{-2}dca>1$ by Lemma \ref{lemma:case16:A:ACCdca}, and $a^{-1}c^{-1}da>1$ by Lemma \ref{lemma:case16:A:ACda}. Therefore, (\ref{proposition:case16.A:contradiction}) states that a product of positives is the identity, a contradiction.
\end{proof}

\subsection{\Case{16}{2}{}}

\noindent{} As a reminder, since we are working in a sub-case of Case 16, we know $a<1$, $b<1$, $c<1$, and $d<1$.

\begin{lemma} In \Case{16}{2}{}, $a^{-1}bc^{-1}b>1$.
\label{lemma:case16:AbCb}
\end{lemma}
\begin{proof}By Lemma~\ref{lemma:eq16}, we have:
\begin{align*}
bc^{-1}b=ad^{-1}a.
\end{align*}
Thus:
\begin{align*}
a^{-1}(bc^{-1}b)=a^{-1}(ad^{-1}a)=(d^{-1}a).
\end{align*}
But $d^{-1}a>1$ in Case 16 by Lemma~\ref{lemma:case16:Da}; therefore, we know:
\begin{align*}
a^{-1}bc^{-1}b=d^{-1}a&>1.\qedhere
\end{align*}
\end{proof}

\begin{corollary} In \Case{16}{2}{}, $a^{-1}bc^{-1}a>1$ and $a^{-1}bc^{-1}>1$.
\label{corollary:case16:AbC}
\label{corollary:case16:AbCa}
\end{corollary}
\begin{proof}By Lemma \ref{lemma:case16:AbCb}, we have
\begin{align*}
a^{-1}bc^{-1}b>1.
\end{align*}
Since $b^{-1}a>1$, we have:
\begin{align*}
b^{-1}a&>1\\
\Rightarrow{}a^{-1}bc^{-1}b(b^{-1}a)>a^{-1}bc^{-1}b&>1\\
\Rightarrow{}a^{-1}bc^{-1}a>a^{-1}bc^{-1}b&>1.
\end{align*}
By $b^{-1}>1$, we have:
\begin{align*}
b^{-1}&>1\\
\Rightarrow{}a^{-1}bc^{-1}b(b^{-1})>a^{-1}bc^{-1}b&>1\\
\Rightarrow{}a^{-1}bc^{-1}>a^{-1}bc^{-1}b&>1.\qedhere
\end{align*}
\end{proof}

\begin{lemma} In \Case{16}{2}{}, $c^{-1}d^{-1}c > 1 $.
\label{lemma:case16:CDc}
\end{lemma}
\begin{proof} By Lemma~\ref{lemma:eq8}, we have:
\begin{align}
b^{-1}c^{-2}d^{2}a &= 1\nonumber{}\\
\Rightarrow{}b^{-1}(b^{-1}b)c^{-2}d(cc^{-1})d(cc^{-1})a &= 1\nonumber{}\\
\Rightarrow{}b^{-1}b^{-1}(aa^{-1})bc^{-2}d(cc^{-1})d(cc^{-1})a &= 1\nonumber{}\\
\Rightarrow{}(b^{-1})(b^{-1}a)(a^{-1}bc^{-1})(c^{-1}dc)(c^{-1}dc)(c^{-1}a) &=1.\label{lemma:case16:CDc:contradiction}
\end{align}
Now $b^{-1}>1$ and $b^{-1}a>1$ by general assumption, $a^{-1}bc^{-1}>1$ in \Case{16}{2}{} by Corollary~\ref{corollary:case16:AbC}, and $c^{-1}a>1$ in Case 16 by Corollary~\ref{lemma:case16:Ca}. Therefore, (\ref{lemma:case16:CDc:contradiction}) shows that
\begin{align*}
c^{-1}dc&<1\\
\Rightarrow{}c^{-1}d^{-1}c&>1.\qedhere
\end{align*}
\end{proof}

\begin{lemma} In \Case{16}{2}{}, $b^{-1}d^{-1}b>1$.
\label{lemma:case16:BDb}
\end{lemma}
\begin{proof} By Lemma~\ref{lemma:eq8}, we have:
\begin{align}
d^{2}a&=c^{2}b\nonumber{}\\
\Rightarrow{}b^{-1}c^{-2}d^{2}a&=1\nonumber{}\\
\Rightarrow{}b^{-1}c^{-2}(bb^{-1})d(bb^{-1})d(bb^{-1})a&=1\nonumber{}\\
\Rightarrow{}(b^{-1})(c^{-1})(c^{-1}b)(b^{-1}db)(b^{-1}db)(b^{-1}a)&=1.\label{lemma:case16:BDb:contradiction}
\end{align}
Now $b^{-1}$, $c^{-1}$, and $b^{-1}a$ are positive by assumption in Case 16, and $c^{-1}b>1$ by Lemma~\ref{lemma:case16:Cb}. Therefore, (\ref{lemma:case16:BDb:contradiction}) shows that
\begin{align*}
b^{-1}db&<1\\
\Rightarrow{}b^{-1}d^{-1}b&>1.\qedhere
\end{align*}
\end{proof}

\begin{lemma} In \Case{16}{2}{}, $a^{-1}dc^{-1}a>1$.
\label{lemma:case16:AdCa}
\end{lemma}

\begin{proof} Starting from the third group relation, we have:
\begin{align}
(d^{-1}c)^{10n+3}c^{-1}bc^{-2}&=1\nonumber{}\\
\Rightarrow{}c^{-1}(cd^{-1})^{10n+3}bc^{-2}&=1\nonumber{}\\
\Rightarrow{}(c^{-1}a)(a^{-1}cd^{-1}a)^{10n+3}(a^{-1}bc^{-1})(c^{-1})&=1.\label{lemma:case16:AdCa:contradiction}
\end{align}
Now $c^{-1}>1$ in Case 16 by assumption, $c^{-1}a>1$ in Case 16 by Corollary~\ref{lemma:case16:Ca}, and $a^{-1}bc^{-1}>1$ in \Case{16}{2}{} by Corollary~\ref{corollary:case16:AbC}. Therefore, (\ref{lemma:case16:AdCa:contradiction}) shows that:
\begin{align*}
a^{-1}cd^{-1}a&<1\\
\Rightarrow{}a^{-1}dc^{-1}a&>1.\qedhere
\end{align*}
\end{proof}

\begin{lemma} In \Case{16}{2}{}, $d^{-2}cb>1$.
\label{lemma:case16:DDcb}
\end{lemma}

\begin{proof} By Lemma~\ref{lemma:eq8}, we have:
\begin{align}
c^{2}b&=d^{2}a\nonumber{}\\
\Rightarrow{}d^{-2}c^{2}&=ab^{-1}.\label{eq8:4}
\end{align}
By Corollary~\ref{corollary:eq6}, we have:
\begin{align}
a^{2}b^{-2}&=dc^{-1}\nonumber{}\\
\Rightarrow{}a^{2}b^{-2}cd^{-1}&=1.\label{eq6:4}
\end{align}
Combining (\ref{eq8:4}) and (\ref{eq6:4}), we find:
\begin{align}
ad^{-2}c^{2}b^{-1}cd^{-1}&=1\label{eq18}\\
\Rightarrow{}dc^{-1}bc^{-2}d^{2}a^{-1}&=1\nonumber{}\\
\Rightarrow{}a^{-1}dc^{-1}bc^{-2}d^{2}&=1\nonumber{}\\
\Rightarrow{}a^{-1}dc^{-1}(aa^{-1})bc^{-1}(bb^{-1})c^{-1}d^{2}&=1\nonumber{}\\
\Rightarrow{}(a^{-1}dc^{-1}a)(a^{-1}bc^{-1}b)(b^{-1}c^{-1}d^{2})&=1.\label{lemma:case16:DDcb:contradiction}
\end{align}
Now $a^{-1}dc^{-1}a>1$ in \Case{16}{2}{} by Lemma~\ref{lemma:case16:AdCa} and $a^{-1}bc^{-1}b>1$ in \Case{16}{2}{} by Lemma~\ref{lemma:case16:AbCb}. Therefore, (\ref{lemma:case16:DDcb:contradiction}) shows that
\begin{align*}
b^{-1}c^{-1}d^{2}&<1\\
\Rightarrow{}d^{-2}cb&>1.\qedhere
\end{align*}
\end{proof}

\begin{lemma} In \Case{16}{2}{}, $dc^{2}a^{-1}>1$.
\label{lemma:case16:dccA}
\end{lemma}
\begin{proof} By the third group relation, we have:
\begin{align*}
(d^{-1}c)^{10n+3}c^{-1}bc^{-2}=1\\
\Rightarrow{}d^{-1}c(d^{-1}c)^{10n+2}c^{-1}b(a^{-1}a)c^{-2}=1\\
\Rightarrow{}(cd^{-1})^{10n+2}(cc^{-1})(ba^{-1})(ac^{-2}d^{-1})=1\\
\Rightarrow{}(cd^{-1})^{10n+2}(ba^{-1})(ac^{-2}d^{-1})=1,
\end{align*}
where the last equality implies $ac^{-2}d^{-1}<1$, since $cd^{-1}>1$ in \Case{16}{2}{} by assumption, and $ba^{-1}>1$ by Lemma~\ref{lemma:case16:bA}. Therefore, we know $dc^{2}a^{-1}>1$.
\end{proof}

\begin{lemma} In \Case{16}{2}{}, $dab^{-1}a^{-1}>1$.
\label{lemma:case16:daBA}
\end{lemma}
\begin{proof} By Lemma~\ref{lemma:eq8}, we have:
\begin{align*}
c^{2}b&=d^{2}a\\
c^{2}ba^{-1}d^{-2}&=1\\
\Rightarrow{}d^{-1}c^{2}ba^{-1}d^{-1}&=1\\
\Rightarrow{}d^{-1}(d^{-1}d)c^{2}(a^{-1}a)ba^{-1}d^{-1}&=1\\
\Rightarrow{}(d^{-2})(dc^{2}a^{-1})(aba^{-1}d^{-1})&=1,
\end{align*}
where the last equality implies $aba^{-1}d^{-1}<1$, since $d^{-1}>1$ in Case 16 by assumption, and $dc^{2}a^{-1}>1$ in \Case{16}{2}{} by Lemma \ref{lemma:case16:dccA}. Therefore, we know $dab^{-1}a^{-1}>1$.
\end{proof}

\begin{lemma} In \Case{16}{2}{}, $b^{-1}cd^{-1}a>1$.
\label{lemma:case16:BcDa}
\end{lemma}
\begin{proof} By Lemma~\ref{lemma:eq5}, we have:
\begin{align*}
d^{-1}a^{2}&=c^{-1}b^{2}\\
\Rightarrow{}dc^{-1}b^{2}a^{-2}&=1\\
\Rightarrow{}a^{-1}dc^{-1}b^{2}a^{-1}&=1\\
\Rightarrow{}(a^{-1}dc^{-1}b)(ba^{-1})&=1,
\end{align*}
where the last equality implies $a^{-1}dc^{-1}b<1$, since $ba^{-1}>1$ in Case 16 by Lemma~\ref{lemma:case16:bA}. Therefore, we know $b^{-1}cd^{-1}a>1$.
\end{proof}

\begin{lemma} In \Case{16}{2}{}, $ab^{-1}d^{-1}a>1$.
\label{lemma:case16:aBDa}
\end{lemma}
\begin{proof} By Lemma~\ref{lemma:eq5}, we have:
\begin{align*}
d^{-1}a^{2}&=c^{-1}b^{2}\\
\Rightarrow{}dc^{-1}b^{2}a^{-2}&=1\\
\Rightarrow{}a^{-1}dc^{-1}(aa^{-1})b(c^{-1}bb^{-1}c)(d^{-1}aa^{-1}d)ba^{-1}&=1\\
\Rightarrow{}(a^{-1}dc^{-1}a)(a^{-1}bc^{-1}b)(b^{-1}cd^{-1}a)(a^{-1}dba^{-1})&=1,
\end{align*}
where the last equality implies $a^{-1}dba^{-1}<1$, since $a^{-1}dc^{-1}a>1$ in \Case{16}{2}{} by Lemma \ref{lemma:case16:AdCa}, $a^{-1}bc^{-1}b>1$ in \Case{16}{2}{} by Lemma \ref{lemma:case16:AbCb}, and $b^{-1}cd^{-1}a>1$ in \Case{16}{2}{} by Lemma \ref{lemma:case16:BcDa}. Therefore, we know $ab^{-1}d^{-1}a>1$.
\end{proof}

\begin{lemma} In \Case{16}{2}{}, $cda^{-1}>1$.
\label{lemma:case16:cdA}
\end{lemma}
\begin{proof} In \Case{16}{2}{}, $1>dc^{-1}$ by assumption, and so we have:
\begin{align*}
1>dc^{-1}&=(d^{-1}d)d(da^{-1}ad^{-1})c^{-1}\\
\Rightarrow{}1>dc^{-1}&=(d^{-1})(d^{3}a^{-1})(ad^{-1}c^{-1}),
\end{align*}
where the last equality implies $ad^{-1}c^{-1}<1$, since $d^{-1}>1$ and $d^{3}a^{-1}>1$ in \Case{16}{2}{} by assumption.
\end{proof}

\begin{lemma} In \Case{16}{2}{}, $a^{-1}dc^{-1}>1$.
\label{lemma:case16:AdC}
\end{lemma}
\begin{proof}
Starting from the third group relation, we have:
\begin{align}
(d^{-1}c)^{10n+3}c^{-1}bc^{-2}&=1\nonumber{}\\
\Rightarrow{}c^{-1}(cd^{-1})^{10n+3}bc^{-2}&=1\nonumber{}\\
\Rightarrow{}c^{-1}(cd^{-1})^{10n+2}(cd^{-1})(aa^{-1})(bc^{-1})c^{-1}&=1\nonumber{}\\
\Rightarrow{}c^{-1}(cd^{-1})^{10n+2}(cd^{-1}a)(a^{-1}bc^{-1})c^{-1}&=1.\label{lemma:case16:AdC:contradiction}
\end{align}
Now $c^{-1}>1$ in Case 16, $cd^{-1}>1$ in \Case{16}{2}{} by assumption, and $a^{-1}bc^{-1}>1$ in \Case{16}{2}{} by Corollary~\ref{corollary:case16:AbC}. Therefore, (\ref{lemma:case16:AdC:contradiction}) shows that:
\begin{align*}
cd^{-1}a&<1\\
\Rightarrow{}a^{-1}dc^{-1}&>1.\qedhere
\end{align*}
\end{proof}


\begin{lemma} In \Case{16}{2}{}, $ca^{-1}bc^{-1}>1$.
\label{lemma:case16:cAbC}
\end{lemma}
\begin{proof} By the first group relation, we have:
\begin{align*}
(b^{-1}a)^{10n}a^{-2}d&=1\\
\Rightarrow{}a^{-2}d(b^{-1}a)^{10n}&=1\\
\Rightarrow{}a^{-2}d(c^{-1}c)(b^{-1}a)^{10n-1}(c^{-1}c)(b^{-1}a)&=1\\
\Rightarrow{}(a^{-1}dc^{-1})(cb^{-1}ac^{-1})^{10n-1}(cb^{-1})&=1,
\end{align*}
where the last equality implies $cb^{-1}ac^{-1}<1$, since $a^{-1}dc^{-1}>1$ in \Case{16}{2}{} by Lemma \ref{lemma:case16:AdC}, and $cb^{-1}>1$ in \Case{16}{2}{} by assumption. Therefore, we know $ca^{-1}bc^{-1}>1$.
\end{proof}

\begin{lemma} In \Case{16}{2}{}, $b^{-1}c^{-1}db>1$.
\label{lemma:case16:BCdb}
\end{lemma}
\begin{proof} Starting from the third group relation, we have:
\begin{align}
(d^{-1}c)^{10n+3}c^{-1}bc^{-2}&=1\nonumber{}\\
\Rightarrow{}c^{-2}(d^{-1}c)^{10n+3}c^{-1}b&=1\nonumber{}\\
\Rightarrow{}c^{-2}(d^{-1}c)^{10n+2}d^{-1}b&=1\nonumber{}\\
\Rightarrow{}(c^{-1})(c^{-1}b)(b^{-1}d^{-1}cb)^{10n+2}(b^{-1}d^{-1}b)&=1.\label{lemma:case16:BCdb:contradiction}
\end{align}
Now $c^{-1}>1$ in Case 16 by assumption, $c^{-1}b>1$ in Case 16 by Lemma~\ref{lemma:case16:Cb}, and $b^{-1}d^{-1}b>1$ in \Case{16}{2}{} by Lemma~\ref{lemma:case16:BDb}. Therefore, (\ref{lemma:case16:BCdb:contradiction}) shows that:
\begin{align*}
b^{-1}d^{-1}cb&<1\\
\Rightarrow{}b^{-1}c^{-1}db&>1.\qedhere
\end{align*}
\end{proof}

\begin{lemma} In \Case{16}{2}{}, $c^{-2}dc>1$.
\label{lemma:case16:CCdc}
\end{lemma}
\begin{proof}
Starting from the third group relation, we have:
\begin{align}
(d^{-1}c)^{10n+3}c^{-1}bc^{-2}&=1\nonumber{}\\
\Rightarrow{}c^{-1}bc^{-2}(d^{-1}c)^{10n+3}&=1\nonumber{}\\
\Rightarrow{}c^{-1}bc^{-2}(d^{-1}c)^{10n}(d^{-1}c)^{3}&=1\nonumber{}\\
\Rightarrow{}(d^{-1}c)^{2}c^{-1}bc^{-2}(d^{-1}c)^{10n}(d^{-1}c)&=1\nonumber{}\\
\Rightarrow{}d^{-1}cd^{-1}bc^{-2}(d^{-1}c)^{10n}(d^{-1}c)&=1\nonumber{}\\
\Rightarrow{}d^{-1}cd^{-1}bc^{-1}(c^{-1}d^{-1}c^{2})^{10n}(c^{-1}d^{-1}c)&=1\nonumber{}\\
\Rightarrow{}d^{-1}(d^{-1}cbb^{-1}c^{-1}d)(bb^{-1})cd^{-1}(aa^{-1})bc^{-1}(c^{-1}d^{-1}c^{2})^{10n}(c^{-1}d^{-1}c)&=1\nonumber{}\\
\Rightarrow{}(d^{-2}cb)(b^{-1}c^{-1}db)(b^{-1}cd^{-1}a)(a^{-1}bc^{-1})(c^{-1}d^{-1}c^{2})^{10n}(c^{-1}d^{-1}c)&=1.\label{lemma:case16:CCdc:contradiction}
\end{align}
Now $d^{-2}cb>1$ in \Case{16}{2}{} by Lemma~\ref{lemma:case16:DDcb}, $b^{-1}c^{-1}db>1$ in \Case{16}{2}{} by Lemma~\ref{lemma:case16:BCdb}, $b^{-1}cd^{-1}a>1$ in \Case{16}{2}{} by Lemma~\ref{lemma:case16:BcDa}, $a^{-1}bc^{-1}>1$ in \Case{16}{2}{} by Corollary~\ref{corollary:case16:AbC}, and $c^{-1}d^{-1}c>1$ in \Case{16}{2}{} by Lemma~\ref{lemma:case16:CDc}. Therefore, (\ref{lemma:case16:CCdc:contradiction}) shows that
\begin{align*}
c^{-1}d^{-1}c^{2}&<1\\
\Rightarrow{}c^{-2}dc&>1.\qedhere
\end{align*}
\end{proof}

\begin{lemma} In \Case{16}{2}{}, $cab^{-1}c^{-1}>1$.
\label{lemma:case16:caBC}
\end{lemma}
\begin{proof}
By Lemma~\ref{lemma:eq8} we have:
\begin{align}
d^{2}a&=c^{2}b\nonumber{}\\
\Rightarrow{}d^{2}&=c^{2}ba^{-1}\nonumber{}\\
\Rightarrow{}d^{3}&=dc^{2}ba^{-1}.\label{eq8:5}
\end{align}
By Lemma~\ref{lemma:eq9} we have:
\begin{align}
c^{-3}d^{3}&=b^{-1}a\nonumber{}\\
\Rightarrow{}d^{3}&=c^{3}b^{-1}a.\label{eq9:2}
\end{align}
Combining (\ref{eq8:5}) and (\ref{eq9:2}), we find:
\begin{align}
c^{3}b^{-1}a&=dc^{2}ba^{-1}\nonumber{}\\
\Rightarrow{}c^{-1}d^{-1}c^{3}b^{-1}a^{2}b^{-1}c^{-1}&=1\nonumber{}\\
\Rightarrow{}cba^{-2}bc^{-3}dc&=1\label{eq12}\\
\Rightarrow{}cba^{-1}(c^{-1}c)a^{-1}bc^{-1}c^{-2}dc&=1\nonumber{}\\
\Rightarrow{}(cba^{-1}c^{-1})(ca^{-1}bc^{-1})(c^{-2}dc)&=1.\label{lemma:case16:caBC:contradiction}
\end{align}
Now $ca^{-1}bc^{-1}>1$ in \Case{16}{2}{} by Lemma~\ref{lemma:case16:cAbC} and $c^{-2}dc>1$ in \Case{16}{2}{} by Lemma~\ref{lemma:case16:CCdc}. Therefore, (\ref{lemma:case16:caBC:contradiction}) shows that:
\begin{align*}
cba^{-1}c^{-1}&<1\\
\Rightarrow{}cab^{-1}c^{-1}&>1.\qedhere
\end{align*}
\end{proof}

\begin{proposition} If $G_n$ is left-orderable, then \Case{16}{2}{} is impossible.
\label{proposition:case16.B}
\end{proposition}
\begin{proof} Suppose $G_n$ is left-orderable, and suppose (for contradiction) that the signs of elements are as in \Case{16}{2}{}. Starting from the first group relation, we have:
\begin{align*}
(a^{-1}b)^{10n}d^{-1}a^{2}&=1\\
\Rightarrow{}a^{-2}d(b^{-1}a)^{10n}&=1\\
\Rightarrow{}a^{-2}da^{-1}(ab^{-1})^{10n}a&=1\\
\Rightarrow{}da^{-1}(ab^{-1})^{10n}a^{-1}&=1\\
\Rightarrow{}db^{-1}(ab^{-1})^{10n-1}a^{-1}&=1\\
\Rightarrow{}db^{-1}(ab^{-1})^{10n-2}ab^{-1}a^{-1}&=1\\
\Rightarrow{}db^{-1}c^{-1}(cab^{-1}c^{-1})^{10n-2}cab^{-1}a^{-1}&=1\\
\Rightarrow{}d(a^{-1}a)b^{-1}(d^{-1}aa^{-1}d)c^{-1}(cab^{-1}c^{-1})^{10n-2}c(d^{-1}d)ab^{-1}a^{-1}&=1\\
\Rightarrow{}(da^{-1})(ab^{-1}d^{-1}a)(a^{-1}dc^{-1})(cab^{-1}c^{-1})^{10n-2}(cd^{-1})(dab^{-1}a^{-1})&=1.
\end{align*}
This is a contradiction, since $da^{-1}>1$ in \Case{16}{2}{} by assumption, $ab^{-1}d^{-1}a>1$ in \Case{16}{2}{} by Lemma~\ref{lemma:case16:aBDa}, $a^{-1}dc^{-1}>1$ in \Case{16}{2}{} by Lemma~\ref{lemma:case16:AdC}, $cab^{-1}c^{-1}>1$ in \Case{16}{2}{} by Lemma~\ref{lemma:case16:caBC}, $cd^{-1}>1$ in \Case{16}{2}{} by assumption, and $dab^{-1}a^{-1}>1$ in \Case{16}{2}{} by Lemma~\ref{lemma:case16:daBA}.
\end{proof}