\section{Case 1}
\label{section:case1}

\noindent{}It would be possible to disprove left-orderability in Cases 1 and 16 by finding a word $w$ that admits only positive occurrences of the generators $a$, $b$, $c$, and $d$ and that satisfies $w=1$ when considered as an element of $G_n$. We were not able to find such a word or disprove its existence. Sections \ref{section:case1} and \ref{section:case16} are presented in lieu of this more desirable solution.\\

\noindent{}In order to show Case 1 ($a,b,c,d>1$) is not possible if $G_n$ is left-orderable, we consider sub-cases (see Table~\ref{table:case1.-}).

\begin{table}[ht]
\begin{center}
\begin{tabular}{l | l | l | l }
Case \hspace{10 pt} & $a^{-1}d$\hspace{10 pt} & $b^{-1}d$\hspace{10 pt} & $b^{-1}c$\hspace{10 pt} \\\hline\hline
\case{1}{1}{} & $+$ & $+$ & $+$ \\\hline
\case{1}{2}{} & $+$ & $+$ & $-$ \\\hline
\case{1}{3}{} & $+$ & $-$ & $+$ \\\hline
\case{1}{4}{} & $+$ & $-$ & $-$ \\\hline
\case{1}{5}{} & $-$ & $+$ & $+$ \\\hline
\case{1}{6}{} & $-$ & $+$ & $-$ \\\hline
\case{1}{7}{} & $-$ & $-$ & $+$ \\\hline
\case{1}{8}{} & $-$ & $-$ & $-$ 
\end{tabular}
\end{center}
\caption{Eight sub-cases of Case 1, considering the signs of $a^{-1}d$, $b^{-1}d$, and $b^{-1}c$.}
\label{table:case1.-}
\end{table}

\noindent{}By Proposition~\ref{proposition:non-trivial}, we know that these sub-cases represent all possibilities if $G_n$ is left-orderable. We proceed by reaching a contradiction to left-orderability in each sub-case.
$\;$
\subsection{\Case{1}{2}{}-\case{1}{7}{}}

We start by showing that if $G_n$ is left-orderable, then \Cases{1}{2}{}-\case{1}{7}{} are not possible.

%begin{case1.ii-VII}

\begin{proposition} If $G_n$ is left-orderable, then \Cases{1}{3}{} and \case{1}{4}{} are impossible.
\end{proposition}
\begin{proof} $\;$ Suppose that $G_n$ is left-orderable, and suppose (for contradiction) that $a^{-1}d>1$ and $b^{-1}d<1$, then:
\begin{align*}
(a^{-1}d)(d^{-1}b)=a^{-1}b&>1\\
\Rightarrow{}b^{-1}a&<1,
\end{align*}
which contradicts the general assumption $b^{-1}a>1$.
\end{proof}

\begin{proposition} If $G_n$ is left-orderable, then \Cases{1}{5}{} and \case{1}{7}{} are impossible.
\end{proposition}
\begin{proof} $\;$ Suppose that $G_n$ is left-orderable, and suppose (for contradiction) that $b^{-1}c>1$ and $a^{-1}d<1$, then:
\begin{align*}
(c^{-1}b)(a^{-1}d)=b^{-1}(bc^{-1}b)a^{-1}d=b^{-1}(bc^{-1}b)(a^{-1}da^{-1})a=b^{-1}a.
\end{align*}
Where the last equality follows from Lemma~\ref{lemma:eq16}. This shows that \Cases{1}{5}{} and \case{1}{7}{} are impossible since $b^{-1}a>1$ by assumption, but $c^{-1}b<1$ and $a^{-1}d<1$ in \Cases{1}{5}{} and \case{1}{7}{}.
\end{proof}

\begin{proposition} If $G_n$ is left-orderable, then \Case{1}{2}{} is impossible.
\end{proposition}
\begin{proof} $\;$ Suppose that $G_n$ is left-orderable. By the third group relation, we have:
\begin{align}
(c^{-1}d)^{10n+3}c^2b^{-1}c&=1\nonumber{}\\
\Rightarrow{}c^2b^{-1}c(c^{-1}d)^{10n+3}&=1\nonumber{}\\
\Rightarrow{}c^2b^{-1}(dc^{-1})^{10n+3}c&=1\label{eq3:2}\\
\Rightarrow{}c^2b^{-1}(dc^{-1})^{10n+2}d&=1\nonumber{}\\
\Rightarrow{}c^2[b^{-1}(dc^{-1})b]^{10n+2}(b^{-1}d)&=1\label{eq3:3}\\
\Rightarrow{}(c^2)([b^{-1}a][a^{-1}d][c^{-1}b])^{10n+2}(b^{-1}a)(a^{-1}d)&=1.\nonumber{}
\end{align}
This shows that \Case{1}{2}{} is impossible, since $a^{-1}d>1$, $b^{-1}d>1$, $b^{-1}c<1$, and $c>1$ in \Case{1}{2}{}.
\end{proof}

\begin{proposition} If $G_n$ is left-orderable, then \Case{1}{6}{} is impossible.
\end{proposition}
\begin{proof} $\;$ Assume that $G_n$ is left-orderable. By (\ref{eq3:3}) we have:
\begin{align*}
c^2[b^{-1}(dc^{-1})b]^{10n+2}(b^{-1}d)&=1\\
\Rightarrow{}c^2[(b^{-1}d)(c^{-1}b)]^{10n+2}(b^{-1}d)&=1.
\end{align*}
This shows that \Case{1}{6}{} is impossible, since $b^{-1}d>1$, $b^{-1}c<1$, and $c>1$ in \Case{1}{6}{}.
\end{proof}
%end{case1.ii-VII}
%begin{case1.i}
%begin{case1.i.2-7}

\subsection{\Case{1}{1}{}}
\noindent{}We now show that if $G_n$ is left-orderable, then \Case{1}{1}{} is impossible. To accomplish this, we consider eight new sub-cases (see Table~\ref{table:case1.i.-}).

\begin{table}[ht]
\begin{center}
\begin{tabular}{l | l | l | l }
Case\hspace{10 pt} & $ca^{-1}$\hspace{10 pt} & $da^{-1}$\hspace{10 pt} & $cb^{-1}$\hspace{10 pt} \\\hline\hline
\case{1}{1}{1} & $+$ & $+$ & $+$ \\\hline
\case{1}{1}{2} & $+$ & $+$ & $-$ \\\hline
\case{1}{1}{3} & $+$ & $-$ & $+$ \\\hline
\case{1}{1}{4} & $+$ & $-$ & $-$ \\\hline
\case{1}{1}{5} & $-$ & $+$ & $+$ \\\hline
\case{1}{1}{6} & $-$ & $+$ & $-$ \\\hline
\case{1}{1}{7} & $-$ & $-$ & $+$ \\\hline
\case{1}{1}{8} & $-$ & $-$ & $-$ 
\end{tabular}
\end{center}
\caption{Eight sub-cases of \Case{1}{1}{}, considering the signs of $ca^{-1}$, $da^{-1}$, and $cb^{-1}$.}
\label{table:case1.i.-}
\end{table}

\noindent{}Since we are working in a sub-case of \Case{1}{1}{}, we also know $a>1$, $b>1$, $c>1$, $d>1$, and the following:
\begin{align} a^{-1}d>1\label{case1.i:inEq:Ad}\\
b^{-1}d>1\label{case1.i:inEq:Bd}\\
b^{-1}c>1
\end{align}

\noindent{}As before, Proposition~\ref{proposition:non-trivial} tells us that the cases shown in Table~\ref{table:case1.i.-} represent all possibilities if $G_n$ is left-orderable.

\subsubsection{\Case{1}{1}{2}-\case{1}{1}{7}}
\noindent{}We continue by disproving left-orderability in each case. We start by showing that if $G_n$ is left-orderable, then \Cases{1}{1}{2}-\case{1}{1}{7} are impossible.

\begin{proposition} If $G_n$ is left-orderable, then \Case{1}{1}{2} is impossible.
\end{proposition}
\begin{proof} $\;$ Suppose that $G_n$ is left-orderable, and suppose (for contradiction) that $ca^{-1}>1$, $da^{-1}>1$ and $bc^{-1}>1$, then: 
\begin{align*}
(ca^{-1})(da^{-1})(bc^{-1})(bc^{-1})>1\\
\Rightarrow{}c(a^{-1}da^{-1})(bc^{-1}b)c^{-1}>1.
\end{align*}
But by Lemma~\ref{lemma:eq16}, we have:
\begin{align*}
c(a^{-1}da^{-1})(bc^{-1}b)c^{-1}=cc^{-1}=1,
\end{align*}
a contradiction.
\end{proof}

\begin{lemma} In \Case{1}{1}{}, $d^{-1}c>1$.\label{case1.1:Dc}
\end{lemma}
\begin{proof} By the fourth group relation, we have:
\begin{align}
d^{2}a^{-1}d(c^{-1}d)^{10n+3}=1\nonumber{}\\
\Rightarrow{}(d^{2})(a^{-1}d)(c^{-1}d)^{10n+3}=1.\label{case1:1:Dc}
\end{align}
Now, $d^{2}>1$ by Case 1, and $a^{-1}d>1$ by \Case{1}{1}{}. Therefore (\ref{case1:1:Dc}) shows that:
\begin{align*}
c^{-1}d&<1\\
\Rightarrow{}d^{-1}c&>1.\qedhere
\end{align*}
\end{proof}

\begin{proposition} If $G_n$ is left-orderable, then \Case{1}{1}{3} is impossible.
\end{proposition}
\begin{proof} Suppose that $G_n$ is left-orderable, and suppose (for contradiction) that $ca^{-1}>1$, $da^{-1}<1$ and $cb^{-1}>1$. Then by Lemma \ref{case1.1:Dc}, $d^{-1}c>1$ and:
\begin{align}
(ad^{-1})(d^{-1}c)(cb^{-1})(ca^{-1})(ad^{-1})&>1\nonumber{}\\
\Rightarrow{}(a)(d^{-2}c^{2})(b^{-1})(cd^{-1})&>1\nonumber{}\\
\Rightarrow{}(a)(ab^{-1})(b^{-1})(cd^{-1})=(a^{2}b^{-2})(cd^{-1})&>1,\label{proposition:case1.i.3:contradiction}
\end{align}
where the last implication follows from Lemma~\ref{lemma:eq8}, which tells us that $d^{2}a=c^{2}b$, implying $d^{-2}c^{2}=ab^{-1}$. Now by Corollary~\ref{corollary:eq6} we know:
\begin{align*}
a^{2}b^{-2}&=dc^{-1}\\
\Rightarrow{}(a^{2}b^{-2})(cd^{-1})&=1.
\end{align*}
This contradicts (\ref{proposition:case1.i.3:contradiction}).
\end{proof}

\begin{proposition} If $G_n$ is left-orderable, then \Case{1}{1}{4} is impossible.
\end{proposition}
\begin{proof} Suppose $G_n$ is left-orderable, and suppose (for contradiction) that $ca^{-1}>1$, $da^{-1}<1$ and $cb^{-1}<1$, then
\begin{align*}
[(bc^{-1})(ca^{-1})]^{10n-1}(bc^{-1})(ca^{-1})(ad^{-1})(a)&>1\\
\Rightarrow{}[(ba^{-1})]^{10n-1}(bd^{-1}a)&>1\\
\Rightarrow{}aa^{-1}(ba^{-1})^{10n-1}(bd^{-1}a)&>1\\
\Rightarrow{}aa^{-1}b(a^{-1}b)^{10n-2}a^{-1}bd^{-1}a&>1\\
\Rightarrow{}a(a^{-1}b)^{10n}d^{-1}a>1.\\
\end{align*}
This contradicts the first group relation, which says $(a^{-1}b)^{10n}d^{-1}a^{2}=1$ or equivalently $a(a^{-1}b)^{10n}d^{-1}a=1$.
\end{proof}

\begin{proposition} If $G_n$ is left-orderable, then \Case{1}{1}{5} is impossible.
\end{proposition}
\begin{proof} Suppose that $G_n$ is left-orderable, and suppose (for contradiction) that $ca^{-1}<1$, $da^{-1}>1$ and $cb^{-1}>1$, then:
\begin{align*}
(c)(cb^{-1})[(da^{-1})(ac^{-1})]^{10n+2}(d)(c^{-1}c)&>1\\
\Rightarrow{}(c^{2}b^{-1})[(dc^{-1})]^{10n+2}(dc^{-1})(c)&>1\\
\Rightarrow{}c^{2}b^{-1}(dc^{-1})^{10n+3}c&>1.
\end{align*}
This contradicts (\ref{eq3:2}), which says that
\begin{align*}
c^{2}b^{-1}(dc^{-1})^{10n+3}c&=1\qedhere
\end{align*}
\end{proof}

\begin{proposition} If $G_n$ is left-orderable, then \Case{1}{1}{6} is impossible.
\end{proposition}
\begin{proof} Suppose $G_n$ is left-orderable, and suppose (for contradiction) that $ca^{-1}<1$ and $da^{-1}>1$, then
\begin{align}
[(da^{-1})(ac^{-1})]^{10n+3}(d^{2})(da^{-1})&>1\nonumber{}\\
\Rightarrow{}(dc^{-1})^{10n+3}d^{3}a^{-1}&>1,\label{proposition:case1.i.6:contradiction}
\end{align}
but by the fourth group relation, we have:
\begin{align}
d^{2}a^{-1}d(c^{-1}d)^{10n+3}&=1\nonumber{}\\
\Rightarrow{}d^{2}a^{-1}(dc^{-1})^{10n+3}d&=1\nonumber{}\\
\Rightarrow{}(dc^{-1})^{10n+3}d^{3}a^{-1}&=1,\label{eq4:2}
\end{align}
this contradicts (\ref{proposition:case1.i.6:contradiction}).
\end{proof}

\begin{proposition} If $G_n$ is left-orderable, then \Case{1}{1}{7} is impossible.
\end{proposition}
\begin{proof} Suppose that $G_n$ is left-orderable, and suppose (for contradiction) that $ca^{-1}<1$, $da^{-1}<1$, and $cb^{-1}>1$, then: 
\begin{align*}
(cb^{-1})(cb^{-1})(ad^{-1})(ac^{-1})&>1\\
\Rightarrow{}(c)(b^{-1}cb^{-1})(ad^{-1}a)(c^{-1})&>1\\
\Rightarrow{}cc^{-1}=1&>1,
\end{align*}
where the last implication follows from Lemma~\ref{lemma:eq16}.
\end{proof}

%end{case1.i.2-7}
%begin{case1.i.1}

\subsubsection{\protect\Case{1}{1}{1}}

\noindent{}We now show that if $G_n$ is left-orderable, then \Case{1}{1}{1} (see Table~\ref{table:case1.i.-}) is impossible.

\begin{lemma} In \Case{1}{1}{1}, $b^{-1}cd^{-1}b>1$.\label{case1.i.1:inEq:BcDb}
\end{lemma}
\begin{proof} By the third group relation, we have:
\begin{align}
(d^{-1}c)^{10n+3}c^{-1}bc^{-2}&=1\nonumber{}\\
\Rightarrow{}(cd^{-1})^{10n+3}bc^{-3}=(cd^{-1})^{10n+3}(bc^{-1})c^{-2}&=1\nonumber{}\\
\Rightarrow{}cd^{-1}&>1,\label{case1.i.1:inEq:cD}
\end{align}
where the last implication follows from the fact that $bc^{-1}<1$ and $c^{-1}<1$ in \Case{1}{1}{1}. Now by the third group relation, we have:
\begin{align}
(d^{-1}c)^{10n+3}c^{-1}bc^{-2}&=1\nonumber{}\\
\Rightarrow{}c^{2}b^{-1}c(c^{-1}d)^{10n+3}&=1\nonumber{}\\
\Rightarrow{}c^{2}b^{-1}cd^{-1}(dc^{-1})^{10n+3}d&=1\nonumber{}\\
\Rightarrow{}c(cb^{-1})(cd^{-1})b(b^{-1}dc^{-1}b)^{10n+3}(b^{-1}d)&=1.\label{lemmaBcDb:contradiction}
\end{align}
By (\ref{case1.i:inEq:Bd}) and (\ref{case1.i.1:inEq:cD}) it is easy to see that all expressions in parentheses in (\ref{lemmaBcDb:contradiction}) are positive except for $(b^{-1}dc^{-1}b)^{10n+3}$. This tells us that:
\begin{align*}
b^{-1}dc^{-1}b&<1\\
\Rightarrow{}b^{-1}cd^{-1}b&>1.\qedhere
\end{align*}
\end{proof}

\begin{lemma} In \Case{1}{1}{1}, $d^{-1}c^{-1}d^{2}>1$.\label{lemma:inEq:DCdd}
\end{lemma}
\begin{proof} $\;$ By (\ref{eq4:2}), we have:
\begin{align*}
(d^{-1}a)d^{-2}(d^{-1}c)^{10n+3}=1.
\end{align*}
However, by (\ref{case1.i:inEq:Ad}), $d^{-1}a<1$ (as is $d^{-2}$), so (\ref{eq4:2}) shows that:
\begin{align}
d^{-1}c>1.\label{case1.i:inEq:Dc}
\end{align}
Now consider:
\begin{align}
(d^{-2}cd)(d^{-1}c)(b^{-1}cd^{-1}b)&=d^{-2}c^{2}b^{-1}cd^{-1}b<1,\label{lemma:DCdd:contradiction}
\end{align}
where the last inequality follows from Lemma~\ref{lemma:inEq7}. By (\ref{case1.i:inEq:Dc}) and Lemma~\ref{case1.i.1:inEq:BcDb} we see that $(d^{-1}c)(b^{-1}cd^{-1}b)>1$, therefore (\ref{lemma:DCdd:contradiction}) shows that:
\begin{align*}
d^{-2}cd&<1\\
\Rightarrow{}d^{-1}c^{-1}d^{2}&>1.\qedhere
\end{align*}
\end{proof}

\begin{corollary} In \Case{1}{1}{1}, $d^{-1}c^{-1}dc>1$ and $b^{-1}c^{-1}d^{2}>1$.\label{corollary:inEq:DCdd}
\end{corollary}
\begin{proof} $\;$ These are immediate consequences of Lemma~\ref{lemma:inEq:DCdd} since $d^{-1}c>1$ in \Case{1}{1}{} (by (\ref{case1.i:inEq:Dc})) and $b^{-1}d>1$ in \Case{1}{1}{} (by (\ref{case1.i:inEq:Bd})).
\end{proof}

\begin{proposition} If $G_n$ is left-orderable, then \Case{1}{1}{1} ($ca^{-1}>1$, $da^{-1}>1$, and $cb^{-1}>1$) is impossible.
\end{proposition}
\begin{proof} $\;$ Suppose $G_n$ is left-orderable, and suppose (for contradiction) that $ca^{-1}>1$, $da^{-1}>1$, and $cb^{-1}>1$. By the third group relation we have:
\begin{align}
(d^{-1}c)^{10n+3}c^{-1}bc^{-2}&=1\nonumber{}\\
\Rightarrow{}cb^{-1}c(c^{-1}d)^{10n+3}c&=1\nonumber{}\\
\Rightarrow{}c(b^{-1}d)(c^{-1}d)^{10n+2}c&=1\nonumber{}\\
\Rightarrow{}(cb^{-1})d(c^{-1}d)(c^{-1}d)^{10n}(c^{-1}d)c&=1\nonumber{}\\
\Rightarrow{}(cb^{-1})d(bb^{-1})c^{-1}d(dd^{-1})(c^{-1}d)^{10n}(dd^{-1})c^{-1}dc&=1\nonumber{}\\
\Rightarrow{}(cb^{-1})(db)(b^{-1}c^{-1}d^{2})d^{-1}(c^{-1}d)^{10n}d(d^{-1}c^{-1}dc)&=1\nonumber{}\\
\Rightarrow{}(cb^{-1})(db)(b^{-1}c^{-1}d^{2})(d^{-1}c^{-1}d^{2})^{10n}(d^{-1}c^{-1}dc)&=1.\label{proposition:1.i.1:contradiction}
\end{align}
Now $cb^{-1}>1$ by assumption in \Case{1}{1}{1}. Similarly, $d>1$ and $b>1$ by assumption in Case 1, thus $db>1$. The remaining terms in parentheses in (\ref{proposition:1.i.1:contradiction}) are positive by Lemma~\ref{lemma:inEq:DCdd} and Corollary~\ref{corollary:inEq:DCdd}. We have therefore reached a contradiction, proving that if $G_n$ is left-orderable, then \Case{1}{1}{1} is impossible.
\end{proof}

%end{case1.i.1}
%begin{case1.i.8}

\subsubsection{\Case{1}{1}{8}}

\noindent{}Next we show that if $G_n$ is left-orderable, then \Case{1}{1}{8} is impossible. After Proposition~\ref{proposition:case1.i.8}, all sub-cases of \Case{1}{1}{} will have been eliminated, showing that \Case{1}{1}{} is impossible if $G_n$ is left-orderable.

\begin{lemma} In \Case{1}{1}{8}, $ab^{-1}>1$.
\label{eq8iL}
\end{lemma}
\begin{proof} By the second group relation, we have:
\begin{align*}
b^{-2}c(b^{-1}a)^{10n}&=1\\
\Rightarrow{}(b^{-1})(cb^{-1})(ab^{-1})^{10n}&=1\\
\Rightarrow{}ab^{-1}&>1,
\end{align*}
where the last implication follows from $b^{-1}<1$ (in Case 1), and $cb^{-1}<1$ (in \Case{1}{1}{8}).
\end{proof}

\begin{proposition} If $G_n$ is left-orderable, then \Case{1}{1}{8} ($ca^{-1}<1$, $da^{-1}<1$, and $cb^{-1}<1$) is impossible.\label{proposition:case1.i.8}
\end{proposition}
\begin{proof} $\;$Suppose $G_n$ is left-orderable and suppose (for contradiction), that $ca^{-1}<1$, $da^{-1}<1$, and $cb^{-1}<1$. By Corollary~\ref{corollary:eq6}, we have:
\begin{align}
a^{2}b^{-2}&=dc^{-1}\nonumber{}\\
\Rightarrow{}(b^{-1})(cd^{-1})(a^{2}b^{-1})&=1\nonumber{}\\
\Rightarrow{}(b^{-1}a)(a^{-1}c)(d^{-1}a)(ab^{-1})&=1\nonumber{}\\
\Rightarrow{}(a^{-1}c)(d^{-1}a)&<1,\label{eqproposition8I}
\end{align}
where the last implication follows from the general assumption $(b^{-1}a)>1$, and since $ab^{-1}>1$ by Lemma~\ref{eq8iL}. Nevertheless, by (\ref{eq3:2}):
\begin{align*}
(c^{2}b^{-1})(dc^{-1})^{10n+3}(c)&=1\\
\Rightarrow{}(c^{2})(b^{-1}a)(a^{-1}dc^{-1}a)^{10n+2}(a^{-1}d)(c^{-1}c)&=1\\
\Rightarrow{}(a^{-1}d)(c^{-1}a)&<1,
\end{align*}
where the last implication follows from $b^{-1}a>1$, $c^{2}>1$ (in Case 1), and $a^{-1}d>1$ (in \Case{1}{1}{}), i.e. $(a^{-1}c)(d^{-1}a)>1$, which contradicts (\ref{eqproposition8I}). Therefore if $G_n$ is left-orderable \Case{1}{1}{8} is impossible.
\end{proof}

%end{case1.i.8}
%end{case1.i}
%begin{case1.viii}

\subsection{\Case{1}{8}{}}

\noindent{}We will now show that if $G_n$ is left-orderable, then \Case{1}{8}{} ($d^{-1}a>1$, $d^{-1}b>1$, $c^{-1}b>1$)is impossible.

\begin{lemma} In \Case{1}{8}{}, $c^{-1}d > 1$.
\label{lemma:inEq:Cd}
\end{lemma}

\begin{proof} Starting from the first group relation, we have:
\begin{align}
(a^{-1}b)^{10n}d^{-1}a^{2}&=1\nonumber{}\\
\Rightarrow{}a(a^{-1}b)^{10n}d^{-1}a&=1\nonumber{}\\
\Rightarrow{}(ba^{-1})^{10n-1}bd^{-1}a&=1\label{eq3:4}\\
\Rightarrow{}(ba^{-1})^{10n-2}ba^{-1}bd^{-1}a&=1\nonumber{}\\
\Rightarrow{}(cc^{-1})(ba^{-1}(dd^{-1}))^{10n-2}(cc^{-1})ba^{-1}(dd^{-1})bd^{-1}a&=1\nonumber{}\\
\Rightarrow{}c([c^{-1}ba^{-1}d][d^{-1}c])^{10n-2}(c^{-1}ba^{-1}d)(d^{-1}b)(d^{-1}a)&=1,\label{lemma:inEq:Cd:contradiction}
\end{align}
but $d^{-1}b>1$, $d^{-1}a>1$, and $c>1$ in \Case{1}{8}{}. Further, we know by Lemma~\ref{lemma:inEq:CbAd} that $c^{-1}ba^{-1}d>1$. Therefore, (\ref{lemma:inEq:Cd:contradiction}) shows that:
\begin{align*}
d^{-1}c&<1\\
\Rightarrow{}c^{-1}d&>1.\qedhere
\end{align*}
\end{proof}

\begin{lemma} In \Case{1}{8}{}, $ab^{-1} > 1$.
\label{lemma:inEq:aB}
\end{lemma}

\begin{proof} By (\ref{eq3:4}):
\begin{align}
(ba^{-1})^{10n-1}b(a^{-1}a)(d^{-1}a)&=1\nonumber{}\\
(ba^{-1})^{10n}a(d^{-1}a)&=1,\label{lemma:inEq:aB:contradiction}
\end{align}
but $d^{-1}a>1$ in \Case{1}{8}{} and $a>1$ in Case 1, so (\ref{lemma:inEq:aB:contradiction}) shows that:
\begin{align*}
ba^{-1}&<1\\
\Rightarrow{}ab^{-1}&>1.\qedhere
\end{align*}
\end{proof}

\begin{lemma} In \Case{1}{8}{}, $a^{-1}dc^{-1}a > 1$. \label{lemma:inEq:AdCa}
\end{lemma}
\begin{proof} By Corollary~\ref{corollary:eq6}, we have:
\begin{align}
a^{2}b^{-2}&=dc^{-1}\nonumber{}\\
\Rightarrow{}b^{-2}cd^{-1}a^{2} &= 1\nonumber{}\\
\Rightarrow{}b^{-1}cd^{-1}a^{2}b^{-1} &= 1\nonumber{}\\
\Rightarrow{}(b^{-1}a)(a^{-1}cd^{-1}a)(ab^{-1}) &= 1.\label{lemma:inEq:AdCa:contradiction}
\end{align}
But $b^{-1}a>1$ by assumption and $ab^{-1}>1$ in \Case{1}{8}{} by Lemma~\ref{lemma:inEq:aB} so (\ref{lemma:inEq:AdCa:contradiction}) shows that:
\begin{align*}
a^{-1}cd^{-1}a&<1\\
\Rightarrow{}a^{-1}dc^{-1}a&>1.\qedhere
\end{align*}
\end{proof}

\begin{lemma} In \Case{1}{8}{}, $c^{-1}d^{-2}a>1$.
\label{lemma:inEq:CDDa}
\end{lemma}
\begin{proof} Starting from the fourth group relation:
\begin{align}
d^{2}a^{-1}d(c^{-1}d)^{10n+3} &= 1\nonumber{}\\
\Rightarrow{}a^{-1}d(c^{-1}d)^{10n+3}d^{2} &= 1\nonumber{}\\
\Rightarrow{}a^{-1}(dc^{-1})^{10n+3}d^{3}&=1\nonumber{}\\
\Rightarrow{}a^{-1}(dc^{-1})^{10n+3}(aa^{-1})d^{2}(cc^{-1})d&=1\nonumber{}\\
\Rightarrow{}(a^{-1}dc^{-1}a)^{10n+3}(a^{-1}d^{2}c)(c^{-1}d)&=1.\label{lemma:inEq:CDDa:contradiction}
\end{align}
But $a^{-1}dc^{-1}a>1$ in \Case{1}{8}{} by Lemma~\ref{lemma:inEq:AdCa} and $c^{-1}d>1$ in \Case{1}{8}{} by Lemma~\ref{lemma:inEq:Cd}, so (\ref{lemma:inEq:CDDa:contradiction}) shows that:
\begin{align*}
a^{-1}d^{2}c&<1\\
\Rightarrow{}c^{-1}d^{-2}a&>1.\qedhere
\end{align*}
\end{proof}

\begin{lemma} In \Case{1}{8}{}, $d^{-1}ad^{-1}>1$.
\label{lemma:inEq:DaD}
\end{lemma}
\begin{proof}
Starting from the fourth group relation, we have:
\begin{align}
(c^{-1}d)^{10n+3}d^{2}a^{-1}d&=1\nonumber{}\\
\Rightarrow{}(c^{-1}d)^{10n+3}(cc^{-1})d^{2}a^{-1}d&=1\nonumber{}\\
\Rightarrow{}(c^{-1}d)^{10n+3}(c)(c^{-1}d)(da^{-1}d)&=1.\label{lemma:inEq:DaD:contradiction}
\end{align}
We know that $c^{-1}d>1$ in \Case{1}{8}{} by Lemma~\ref{lemma:inEq:Cd} and we know that $c>1$ in \Case{1}{8}{}, so (\ref{lemma:inEq:DaD:contradiction}) shows that:
\begin{align*}
da^{-1}d&<1\\
\Rightarrow{}d^{-1}ad^{-1}&>1.\qedhere
\end{align*}
\end{proof}

\begin{lemma} In \Case{1}{8}{}, $cba^{-1}>1$.
\label{lemma:inEq:cbA}
\end{lemma}

\begin{proof} By Lemma~\ref{lemma:eq8}, we have:
\begin{align}
c^{2}b&=d^{2}a\nonumber{}\\
\Rightarrow{}c^{-1}d^{2}ab^{-1}c^{-1}&= 1\nonumber{}\\
\Rightarrow{}(c^{-1}d)d(ab^{-1}c^{-1})&=1.\label{lemma:inEq:cbA:contradiction}
\end{align}
We know that $c^{-1}d>1$ in \Case{1}{8}{} by Lemma~\ref{lemma:inEq:Cd} and we know that $d>1$ in \Case{1}{8}{}, so (\ref{lemma:inEq:cbA:contradiction}) shows that:
\begin{align*}
ab^{-1}c^{-1}&<1\\
\Rightarrow{}cba^{-1}&>1.\qedhere
\end{align*}
\end{proof}

\begin{lemma} In \Case{1}{8}{}, $da^{-1}bc^{-1}>1$.
\label{lemma:inEq:dAbC}
\end{lemma}
\begin{proof} By Lemma~\ref{lemma:eq16}, we have:
\begin{align*}
b^{-1}cb^{-1}ad^{-1}a &= 1\\
\Rightarrow{}(cb^{-1}ad^{-1})(ab^{-1}) &=1\\
\Rightarrow{}ab^{-1}&=da^{-1}bc^{-1}\\
\Rightarrow{}da^{-1}bc^{-1}&>1,
\end{align*}
where the last implication follows from Lemma~\ref{lemma:inEq:aB}.
\end{proof}

\begin{lemma} In \Case{1}{8}{}, $cb^{-1}ac^{-1}>1$.
\label{lemma:inEq:cBaC}
\end{lemma}
\begin{proof} By (\ref{eq3:4}), we have:
\begin{align}
(ba^{-1})^{10n}bd^{-1}a&=1\nonumber{}\\
\Rightarrow{}b(a^{-1}b)^{10n}d^{-1}a&=1\nonumber{}\\
\Rightarrow{}b(a^{-1}b)(a^{-1}b)^{10n-1}d^{-1}a&=1\nonumber{}\\
\Rightarrow{}(ba^{-1})(bc^{-1})(ca^{-1}bc^{-1})^{10n-1}cd^{-1}a&=1\nonumber{}\\
\Rightarrow{}(cc^{-1})(ba^{-1})(dd^{-1})(ad^{-1}da^{-1})(bc^{-1})(ca^{-1}bc^{-1})^{10n-1}cd^{-1}a&=1\nonumber{}\\
\Rightarrow{}c(c^{-1}ba^{-1}d)(d^{-1}ad^{-1})(da^{-1}bc^{-1})(ca^{-1}bc^{-1})^{10n-1}c(d^{-1}a)&=1.\label{lemma:inEq:cBaC:contradiction}
\end{align}
We know $c^{-1}ba^{-1}d>1$ in \Case{1}{8}{} by Lemma~\ref{lemma:inEq:CbAd}; we know $d^{-1}ad^{-1}>1$ in \Case{1}{8}{} by Lemma~\ref{lemma:inEq:DaD}; and we know $da^{-1}bc^{-1}>1$ by Lemma~\ref{lemma:inEq:dAbC}. Furthermore, we know $d^{-1}a>1$ and $c>1$ by assumption in \Case{1}{8}{}. Therefore, (\ref{lemma:inEq:cBaC:contradiction}) shows that:
\begin{align*}
ca^{-1}bc^{-1}&<1\\
\Rightarrow{}cb^{-1}ac^{-1}&>1.\qedhere
\end{align*}
\end{proof}

\begin{lemma} In \Case{1}{8}{}, $cba^{-1}c^{-1}>1$.
\label{lemma:inEq:cbAC}
\end{lemma}
\begin{proof} By Lemma~\ref{lemma:eq8}, we have:
\begin{align}
c^{2}b&=d^{2}a\nonumber{}\\
\Rightarrow{}c^{-1}d^{2}ab^{-1}c^{-1}&=1\nonumber{}\\
\Rightarrow{}c^{-1}d^{2}a^{-1}a^{2}b^{-1}c^{-1}&=1\nonumber{}\\
\Rightarrow{}c^{-1}d^{2}a^{-1}(bc^{-1}cb^{-1})a(c^{-1}c)ab^{-1}c^{-1}&=1\nonumber{}\\
\Rightarrow{}(c^{-1}d)(da^{-1}bc^{-1})(cb^{-1}ac^{-1})(cab^{-1}c^{-1})&=1.\label{lemma:inEq:cbAC:contradiction}
\end{align}
We know that $c^{-1}d>1$ in \Case{1}{8}{} by Lemma~\ref{lemma:inEq:Cd}; we know that $da^{-1}bc^{-1}>1$ in \Case{1}{8}{} by Lemma~\ref{lemma:inEq:dAbC}; and we know that $cb^{-1}ac^{-1}>1$ in \Case{1}{8}{} by Lemma~\ref{lemma:inEq:cBaC}. Therefore, (\ref{lemma:inEq:cbAC:contradiction}) shows that:
\begin{align*}
cab^{-1}c^{-1}&<1\\
\Rightarrow{}cba^{-1}c^{-1}&>1.\qedhere
\end{align*}
\end{proof}

\begin{proposition}
If $G_{n}$ is left-orderable then \Case{1}{8}{} ($d^{-1}a > 1$, $d^{-1}b>1$, and $c^{-1}b>1$) is not possible.
\label{proposition:case1.viii}
\end{proposition}
\begin{proof} By (\ref{eq3:4}), we have:
\begin{align}
(ba^{-1})^{10n}bd^{-1}a&=1\nonumber{}\\
\Rightarrow{}(ba^{-1})(bd^{-1}a)(ba^{-1})^{10n-2}(ba^{-1})&=1\nonumber{}\\
\Rightarrow{}(ba^{-1})(bd^{-1}a)(c^{-1}c)(ba^{-1})^{10n-2}(c^{-1}c)(ba^{-1})&=1\nonumber{}\\
\Rightarrow{}(ba^{-1})(bd^{-1})(ac^{-1})(cba^{-1}c^{-1})^{10n-2}(c)(ba^{-1})&=1\nonumber{}\\
\Rightarrow{}(cc^{-1})(ba^{-1})(dd^{-1})(bd^{-1})(ad^{-1}da^{-1})(bc^{-1}cb^{-1})\nonumber{}\\
(ac^{-1})(cba^{-1}c^{-1})^{10n-1}(cba^{-1})&=1\nonumber{}\\
\Rightarrow{}c(c^{-1}ba^{-1}d)(d^{-1}b)(d^{-1}ad^{-1})(da^{-1}bc^{-1})\nonumber{}\\
(cb^{-1}ac^{-1})(cba^{-1}c^{-1})^{10n-1}(cba^{-1})&=1.\label{proposition:case1.viii:contradiction}
\end{align}
We know that $c^{-1}ba^{-1}d$, $d^{-1}ad^{-1}$, $da^{-1}bc^{-1}$, $cb^{-1}ac^{-1}$, $cba^{-1}c^{-1}$, and  $cba^{-1}$ are all positive in \Case{1}{8}{} by Lemmas~\ref{lemma:inEq:CbAd}, \ref{lemma:inEq:DaD}, \ref{lemma:inEq:dAbC}, \ref{lemma:inEq:cBaC}, \ref{lemma:inEq:cbAC}, and \ref{lemma:inEq:cbA} respectively. Also, $c$ is positive in \Case{1}{8}{} by assumption. Therefore (\ref{proposition:case1.viii:contradiction}) shows that if $G_n$ is left-orderable, then \Case{1}{8}{} is not possible.
\end{proof}

    %end{case1.viii}
%end{case1}

%begin{case16}

\noindent{}With Proposition~\ref{proposition:case1.viii}, we have eliminated the one remaining sub-case of Case 1. Thus, we have shown that if $G_n$ is left-orderable, then the only option for the signs of the four generators is Case 16. That is, if $G_n$ is left-orderable then $a<1$, $b<1$, $c<1$, and $d<1$.