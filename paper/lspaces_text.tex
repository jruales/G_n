
%\begin{definition}
%A 3-manifold that can be obtained as a union of two solid tori associated by a homeomorphism of their boundaries is called a \emph{lens space}
%\end{definition}
%}

\subsection{Heegaard Floer Homology, L-Spaces, and Lens Spaces}
Heegard Floer homology is an invariant of 3-manifolds, introduced by Ozsv\'{a}th and Szab\'{o} \cite{OzsvathSzabo}. It associates a closed 3-manifold $Y$ a graded vector space over $\mathbb{F}_2$, denoted $\reallywidehat{HF}(Y)$.
\begin{definition}An \emph{L-space} is a rational homology sphere $Y$ with simplest possible Heegaard Floer homology, that is, with
\begin{align*}
rk \reallywidehat{HF}(Y) = |H_1(Y ; \mathbb{Z})|.
\end{align*}
All lens spaces have this property.
\end{definition}